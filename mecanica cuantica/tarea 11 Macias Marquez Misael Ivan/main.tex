 \documentclass[12pt,a4paper]{article}

\usepackage{graphicx}% Include figure files
\usepackage{dcolumn}% Align table columns on decimal point
\usepackage{bm}% bold math
%\usepackage{hyperref}% add hypertext capabilities
%\usepackage[mathlines]{lineno}% Enable numbering of text and display math
%\linenumbers\relax % Commence numbering lines

%\usepackage[showframe,%Uncomment any one of the following lines to test 
%%scale=0.7, marginratio={1:1, 2:3}, ignoreall,% default settings
%%text={7in,10in},centering,
%%margin=1.5in,
%%total={6.5in,8.75in}, top=1.2in, left=0.9in, includefoot,
%%height=10in,a5paper,hmargin={3cm,0.8in},
%]{geometry}

\usepackage{multicol}%Para hacer varias columnas
\usepackage{multicol,caption}
\usepackage{multirow}
\usepackage{cancel}
\usepackage{hyperref}
\hypersetup{
    colorlinks=true,
    linkcolor=blue,
    filecolor=magenta,      
    urlcolor=cyan,
}

\setlength{\topmargin}{-1.0in}
\setlength{\oddsidemargin}{-0.3pc}
\setlength{\evensidemargin}{-0.3pc}
\setlength{\textwidth}{6.75in}
\setlength{\textheight}{9.5in}
\setlength{\parskip}{0.5pc}

\usepackage[utf8]{inputenc}
\usepackage{expl3,xparse,xcoffins,titling,kantlipsum}
\usepackage{graphicx}
\usepackage{xcolor} 
\usepackage{siunitx}
\usepackage{nopageno}
\usepackage{lettrine}
\usepackage{caption}
\renewcommand{\figurename}{Figura}
\usepackage{float}
\renewcommand\refname{Bibliograf\'ia}
\usepackage{amssymb}
\usepackage{amsmath}
\usepackage[rightcaption]{sidecap}
\usepackage[spanish]{babel}

\providecommand{\abs}[1]{\lvert#1\rvert}
\providecommand{\norm}[1]{\lVert#1\rVert}
\newcommand{\dbar}{\mathchar'26\mkern-12mu d}

\usepackage{mathtools}
\DeclarePairedDelimiter\bra{\langle}{\rvert}
\DeclarePairedDelimiter\ket{\lvert}{\rangle}
\DeclarePairedDelimiterX\braket[2]{\langle}{\rangle}{#1 \delimsize\vert #2}

% CABECERA Y PIE DE PÁGINA %%%%%
\usepackage{fancyhdr}
\pagestyle{fancy}
\fancyhf{}

\begin{document}

Macías Márquez Misael Iván

\begin{enumerate}


\item



%%%2%%%



\item \begin{enumerate}
    \item Demuestre que el momento angular orbital relativo y del centro de masa son independientes en términos de las coordenadas del centro de masa $\hat{R}$ y relativa $\hat{r}$ en la forma
    
    \begin{equation*}
        \hat{L} = \hat{R} \times \hat{P} + \hat{r} \times \hat{p}
    \end{equation*}
    
    \textbf{Sol:}
    
    \item También muestre que el momento angular orbital relativo y del centro de masa son independientes, y sus componentes satisfacen las relaciones de conmutación usuales.
    
    \textbf{Sol:}
    
\end{enumerate}



%%%3%%%



\item Considere una partícula de masa $m$, con el número cuántico de momento angular $l = 0$, dentro de un pozo de potencial central dado por

\begin{equation*}
    V(r) = \left\{\begin{array}{cc}
        -V_0 & r \leq a > 0  \\
         0 & r < a
    \end{array} \right.
\end{equation*}

con $V_0 > 0$ y $a$ constantes. Para este sistema ¿Siempre existen estados ligados? Encuentre la condición que debe cumplir $V_0$ para que esto sea así. Además, encuentre la función de onda y el valor propio de la energía para el estado base (nivel de más baja energía).

\textbf{Sol:}

\begin{equation*}
    n= \sqrt{\left(\frac{\sin{(\delta + \alpha - \theta_{I1})}+\sin{(\theta_{I1})}\cos{\alpha}}{\sin{\alpha}}\right)^{2} + \sin^2{\theta_{I1}}}
\end{equation*}
    
    
\end{enumerate}

\end{document}
