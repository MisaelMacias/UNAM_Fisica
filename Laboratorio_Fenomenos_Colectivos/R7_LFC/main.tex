\documentclass[DIV=calc, paper=a4, fontsize=11pt]{scrartcl}


\usepackage{makeidx}
\usepackage{graphicx}
\usepackage{flushend}

\usepackage{lmodern}
\usepackage[left=1.5cm,right=1.5cm,top=2.5cm,bottom=2cm]{geometry}
\usepackage{float}		
\bibliographystyle{plain} 
\pagestyle{plain} 
\pagenumbering{arabic}
\usepackage{fancyhdr} 	


\usepackage[T1]{fontenc}
\usepackage[utf8]{inputenc}
\usepackage[spanish]{babel}
\usepackage{hyperref}
\usepackage{graphicx}

\usepackage{lipsum}
\usepackage[protrusion=true,expansion=true]{microtype}
\usepackage{amsmath,amsfonts,amsthm}
\usepackage[svgnames]{xcolor}
\usepackage[svgnames]{xcolor}
\usepackage{booktabs}
\usepackage{fix-cm}
\usepackage{multicol}
\newenvironment{Figura}
  {\par\medskip\noindent\minipage{\linewidth}}
  {\endminipage\par\medskip}

\usepackage{sectsty}
\allsectionsfont{\usefont{OT1}{phv}{b}{n}}

\usepackage{fancyhdr}
\pagestyle{fancy}
\usepackage{siunitx}
\usepackage{lastpage}

\lhead{}
\chead{}
\rhead{}

\lfoot{}
\cfoot{}
\rfoot{\footnotesize Page \thepage\ of \pageref{LastPage}}

\renewcommand{\headrulewidth}{0.0pt}
\renewcommand{\footrulewidth}{0.4pt}

\usepackage{lettrine}
\newcommand{\initial}[1]{\lettrine[lines=3,lhang=0.3,nindent=0em]{
\color{DarkGoldenrod}{\textsf{#1}}}{}}

\usepackage{titling}

\newcommand{\HorRule}{\color{DarkGoldenrod} \rule{\linewidth}{1pt}}

\pretitle{\vspace{-120pt} \begin{flushleft} \HorRule \fontsize{22}{35} \usefont{OT1}{phv}{b}{n} \color{DarkRed} \selectfont}

\title{Principio de Arquímedes\\ %Aquí va el nombre de la práctica 
Práctica 7} %Numero de la práctica 

\posttitle{\par\end{flushleft}\vskip 0.5em}

\preauthor{\begin{flushleft}\large \lineskip 0.5em \usefont{OT1}{phv}{b}{sl} \color{DarkRed}}

\author{Misael Iván Macías Márquez\\
misaelmacias@ciencias.unam.mx}

\postauthor{\footnotesize \usefont{OT1}{phv}{m}{sl} \color{Black}

\vspace*{0.1cm} Facultad de Ciencias, UNAM

\par\end{flushleft}\HorRule}

\date{Fecha\\Semestre 2022-2}


\begin{document}

\maketitle


\begin{abstract}
\textbf{Resumen:} 


jlj
\end{abstract}

\begin{multicols}{2}




\section*{Introducción}





\section*{Desarrollo experimental}
\begin{enumerate}
\item Con ayuda del dens\'imetro obtener la densidad del fluido de la primer probeta.
\item Usando el vernier, medir los diámetros de las 3 pelotas y con ayuda de la báscula digital pesar cada una de ellas.
\item Graduar la probeta con ayuda de cinta adhesiva cada 5 cm.
\item Dejar caer cada una de las pelotas en el fluido de la probeta, y con un cronómetro medir el tiempo que tarda en recorrer cada marca, repetir este procedimiento 5 veces para cada una de las pelotas, figura (2).
\item Repetir el paso anterior con las otras dos pelotas.
\item Repetir todos los pasos anteriores para cada probeta.
\end{enumerate}


\section*{Resultados y Análisis}
\begin{itemize}
\item Hacer una tabla mostrando: cada pelota, su di\'ametro, su radio, su masa, su volumen y su densidad.
\item Hacer una gr\'afica de distancia vs tiempo para poder apreciar y obtener la velocidad final de cada pelota en cada fluido (de preferencia un esquema por pelota, donde se aprecien todos los fluidos).
\item Obtener y comparar con los valores de la literatura el coeficiente de viscosidad para cada fluido.
\item \textbf{PUNTO EXTRA} ¿C\'omo cambia la ley de Stokes si en lugar de ser una esfera el objeto que se sumerge, fuera un disco?
\end{itemize}

\section*{Conclusiones}
  
\begin{thebibliography}{99}
\bibitem{1} Sears, Zemansky \emph{Física universitaria} Vol. 1, Gpo. Edit. Pearson Educación, 12 ed.l, México, 2009.

\end{thebibliography}

\section*{Apéndices}


\end{multicols}
\end{document}