\documentclass[12pt,a4paper]{article}

\usepackage{graphicx}% Include figure files
\usepackage{dcolumn}% Align table columns on decimal point
\usepackage{bm}% bold math
%\usepackage{hyperref}% add hypertext capabilities
%\usepackage[mathlines]{lineno}% Enable numbering of text and display math
%\linenumbers\relax % Commence numbering lines

%\usepackage[showframe,%Uncomment any one of the following lines to test 
%%scale=0.7, marginratio={1:1, 2:3}, ignoreall,% default settings
%%text={7in,10in},centering,
%%margin=1.5in,
%%total={6.5in,8.75in}, top=1.2in, left=0.9in, includefoot,
%%height=10in,a5paper,hmargin={3cm,0.8in},
%]{geometry}

\usepackage{multicol}%Para hacer varias columnas
\usepackage{multicol,caption}
\usepackage{multirow}
\usepackage{cancel}
\usepackage{hyperref}
\hypersetup{
    colorlinks=true,
    linkcolor=blue,
    filecolor=magenta,      
    urlcolor=cyan,
}

\setlength{\topmargin}{-1.0in}
\setlength{\oddsidemargin}{-0.3pc}
\setlength{\evensidemargin}{-0.3pc}
\setlength{\textwidth}{6.75in}
\setlength{\textheight}{9.5in}
\setlength{\parskip}{0.5pc}

\usepackage[utf8]{inputenc}
\usepackage{expl3,xparse,xcoffins,titling,kantlipsum}
\usepackage{graphicx}
\usepackage{xcolor} 
\usepackage{nopageno}
\usepackage{lettrine}
\usepackage{caption}
\renewcommand{\figurename}{Figura}
\usepackage{float}
\renewcommand\refname{Bibliograf\'ia}
\usepackage{amssymb}
\usepackage{amsmath}
\usepackage[rightcaption]{sidecap}
\usepackage[spanish]{babel}

\providecommand{\abs}[1]{\lvert#1\rvert}
\providecommand{\norm}[1]{\lVert#1\rVert}
\newcommand{\dbar}{\mathchar'26\mkern-12mu d}

% CABECERA Y PIE DE PÁGINA %%%%%
\usepackage{fancyhdr}
\pagestyle{fancy}
\fancyhf{}

\begin{document}

\begin{enumerate}
    \item \textbf{Sea $(V,<,>)$ e.v.p.i / $C$. Dada $T \in \mathcal{L}(V)$ autoadjunto. Demostrar que:}
    \begin{enumerate}
        \item $\norm{u+iTu}=\norm{u-iTu}$  $\forall u \in V$
        
        \textbf{SOLUCIÓN:}
        
        Sea $u \in V$
        
        \begin{equation*}
            \norm{u+iTu}^2=<u+iTu,u+iTu>
        \end{equation*}
        
        y por propiedades del producto interno
        
        \begin{equation*}
            \norm{u+iTu}^2=<u,u+iTu>+<iTu,u+iTu>
        \end{equation*}
        
        \begin{equation*}
            =<u,u>+<u,iTu>+<iTu,u>+<iTu,iTu>
        \end{equation*}
        
        \begin{equation*}
            =\norm{u}^2+(-i)<u,Tu>+i<Tu,u>+i(-i)\norm{Tu}^2
        \end{equation*}
        
        \begin{equation*}
            =\norm{u}^2-i<T^*u,u>+i<Tu,u>+\norm{Tu}^2
        \end{equation*}
        
        por hipótesis T es autoadjunto, entonces
        
        \begin{equation*}
            \norm{u+iTu}^2=\norm{u}^2\cancel{-i<Tu,u>+i<Tu,u>}+\norm{Tu}^2
        \end{equation*}
        
        \begin{equation*}
            =\norm{u}^2+\norm{Tu}^2+i<u,Tu>-i<u,Tu>
        \end{equation*}
        
        \begin{equation*}
            =\norm{u}^2+\norm{Tu}^2+i<u,Tu>-i<T^*u,u>
        \end{equation*}
        
        \begin{equation*}
            =\norm{u}^2+\norm{Tu}^2+i<u,Tu>-i<Tu,u>
        \end{equation*}
        
        \begin{equation*}
            =<u,u>+<-iTu,u>+<u,-iTu>+i(-i)<Tu,Tu>
        \end{equation*}
        
        \begin{equation*}
            =<u-iTu,u-iTu>=\norm{u-iTu}^2
        \end{equation*}
        
        \begin{equation*}
            \therefore \norm{u+iTu}=\norm{u-iTu}
        \end{equation*}
        
        $\hspace{15cm} \blacksquare$
        
        \item $u+iTu=v+iTu$  $\leftrightarrow$  $u=v$
        
        \textbf{SOLUCIÓN:}
        
        $\leftarrow$) es inmediato que si $u=v$, entonces $u+iTu=v+iTu$
        
        $\rightarrow$) $u+iTu=v+iTu$ $\rightarrow$ $u=v+i(Tu-Tu)$ $\rightarrow$ $u=v+i(0)=v$
        
        \begin{equation*}
            \therefore u+iTu=v+iTu \hspace{1cm}  \leftrightarrow  \hspace{1cm} u=v
        \end{equation*}
        
        $\hspace{15cm} \blacksquare$
        
        \item $I+iT$ \textbf{es no singular.  obs $T\in \mathcal{L}(V)$ es no singular}
        
        \textbf{SOLUCIÓN:}
        
        Sea $u,v \in V$. Supongamos que $I+iT$ es singular, entonces $I+iT$ es no invertible, entonces $(I+iT)(u)$no es inyectiva, entonces $\exists u,v\in V $ tal que si $(I+iT)(u)=(I+iT)(v)$ $\rightarrow$ $u \neq v$ pero por el inciso b) sabemos que eso es falso, por lo tanto $I+iT$ es no singular 
        
        $\hspace{15cm} \blacksquare$
        
        \item $I-iT$ \textbf{es no singular}   $\leftrightarrow$  $\text{Ker}(T)={0_V}$
        
        \textbf{SOLUCIÓN:}
        
        $\rightarrow$) si $I-iT$ es no singular, entonces $I-iT$ es invertible, entonces $I-iT$ es inyectiva y es inyectiva $\leftrightarrow$ $\text{Ker}(I-iT)=0_V$, lo que implica que $\text{Ker}(T)=0_V$
        
        $\leftarrow$) $\text{Ker}(T)=0_V$ $\rightarrow$  $T(0_v)=0_v$ $\rightarrow$ $iT(0_V)=0_V$ $\rightarrow$ $0_V-iT(0_V)=0_V$ $\rightarrow$ $(I-iT)(0_V)=0_V$ $\rightarrow$ $\text{Ker}(I-iT)=0_V$ $\leftrightarrow$ $I-iT$ es inyectiva, entonces $I-iT$ es invertible y por lo tanto $I-iT$ es no singular
        
        $\hspace{15cm} \blacksquare$
        
        \item \textbf{Sea $V$ de dimensión finita. Demostrar que $U=(I-iT)(I+iT)^{-1}$ es un operador unitario, llamado la transformación de Cayley de T}
        
        \textbf{SOLUCIÓN:} 
        
        Primero necesitamos demostrar que $((I+iT)^{-1})^{*}= (I-iT)^{-1}$, así que, sea $u,v \in V$
        
        \begin{equation*}
            <u,((I+iT)^{-1})^{*}(I-iT)v>=<(I+iT)^{-1}u,(I-iT)v>
        \end{equation*}
        
        \begin{equation*}
            =<(I+iT)^{-1}u,(I-iT^{*})v>=<(I+iT)^{-1}u,(I+iT)^{*}v>
        \end{equation*}
        
        \begin{equation*}
            =<\cancel{(I+iT)(I+iT)^{-1}}u,v>=<u,v>
        \end{equation*}
        
        ahora si, veamos que U es unitario
        
        \begin{equation*}
            UU^{*}=(I-iT)(I+iT)^{-1}((I-iT)(I+iT)^{-1})^{*}
        \end{equation*}
        
        \begin{equation*}
            =(I-iT)(I+iT)^{-1}(I-iT)^{*}((I+iT)^{-1})^{*}
        \end{equation*}
        
        \begin{equation*}
            =(I-iT)\cancel{(I+iT)^{-1}(I+iT)}(I-iT)^{-1}
        \end{equation*}
        
        \begin{equation*}
            =\cancel{(I-iT)(I-iT)^{-1}}=I
        \end{equation*}
        
        \begin{equation*}
            U^{*}U=((I-iT)(I+iT)^{-1})^{*}(I-iT)(I+iT)^{-1}
        \end{equation*}
        
        \begin{equation*}
            =(I-iT)^*((I+iT)^{-1})^{*}(I-iT)(I+iT)^{-1}
        \end{equation*}
        
        \begin{equation*}
            =(I+iT)\cancel{(I-iT)^{-1}(I-iT)}(I+iT)^{-1}
        \end{equation*}
        
        \begin{equation*}
            =\cancel{(I+iT)(I+iT)^{-1}}=I
        \end{equation*}
        
        $\therefore$ $U$ es unitario 
        
        $\hspace{15cm}$  $\blacksquare$
        
         
    \end{enumerate}
    
    \item \textbf{Sea $(V,<,>)$/K  finito dimensional. Probar}
    
    \begin{enumerate}
        \item $H_u:V \rightarrow V$ \textbf{definido por} $H_u(x)=x-2<x,u>u$ \textbf{es lineal} $\forall x \in V$
        
        \textbf{SOLUCIÖN:}
        
        Para que $H_u$ sea lineal, debemos mostrar que abre sumas y saca escalares, entonces, sea $x,y \in  V$ y $c$ un escalar
        
        \begin{equation*}
            H_u(x+cy)=(x+cy)-2<(x+cy),u>u
        \end{equation*}
        
        que por propiedades del producto interno
        
        \begin{equation*}
            H_u(x+cy):=x+cy-2(<x,u>u+c<y,u>u)
        \end{equation*}
        
        \begin{equation*}
            =x+cy-2<x,u>u-2c<y,u>u
        \end{equation*}
        
        \begin{equation*}
            =(x-2<x,u>u)+(cy-2c<y,u>u)
        \end{equation*}
        
        \begin{equation*}
            =(x-2<x,u>u)+c(y-2<y,u>u)=:H_u(x)+cH_u(y)
        \end{equation*}
        
        \begin{equation*}
            \therefore H_u(x)  \text{ es lineal } \forall x \in V 
        \end{equation*}
        
        $\hspace{15cm} \blacksquare$
        
        \item $H_u(x)=x$   $\leftrightarrow$   $x \perp u$
        
        \textbf{SOLUCIÓN:}
        
        $\rightarrow$) por hipótesis, $H_u(x):=x-2<x,u>u=x$, y entonces, $2<x,u>u=0_V$, o bien, $<x,u>=0$ y por lo tanto $x \perp u$
        
        $\leftarrow$) como $x \perp u$ $\rightarrow$ $<x,u>=0$, que a su vez se puede escribir como 
        
        \begin{equation*}
            2<x,u>u=0
        \end{equation*}
        
        \begin{equation*}
            -2<x,u>u=0
        \end{equation*}
        
        \begin{equation*}
            H_u(x):=x-2<x,u>u=x
        \end{equation*}
        
        \begin{equation*}
            \therefore \hspace{1cm} H_u(x)=x \hspace{0.1cm}   \leftrightarrow \hspace{0.1cm}   x \perp u
        \end{equation*}
        
        $\hspace{15cm} \blacksquare$
        
        \item $H_u(u)=-u$
        
        \textbf{SOLUCIÓN:}
        
        Supongamos que u es unitario, entonces ($<u,u>=1$)
        
        \begin{equation*}
            H_u(u):= u-2<u,u>u=u-2u=-u
        \end{equation*}
        
        $\hspace{15cm} \blacksquare$
        
        \item $H_u^* =H_u$ \textbf{y}  $H_u^2=I$ ($\therefore H$ \textbf{es unitario})
        
        \textbf{SOLUCIÓN:}
        
        Sea $x,y \in V$
        
        \begin{equation*}
            <y,H_u(x)>= <y,x-2<x,u>u>
        \end{equation*}
        
        que por propiedades del producto interno
        
        \begin{equation*}
            <y,H_u(x)>= <y,x>-2<y,<x,u>u>
        \end{equation*}
        
        \begin{equation*}
            =<y,x>-2\overline{<x,u>}<y,u>
        \end{equation*}
        
        \begin{equation*}
            =<y,x>-2<u,x><y,u>
        \end{equation*}
        
        \begin{equation*}
            <y,x>-2<y,u><u,x>
        \end{equation*}
        
        \begin{equation*}
            <y-2<y,u>u,x>=<H_u(y),x>=<y,H_u^*(x)>
        \end{equation*}
        
        \begin{equation*}
            \therefore H_u^*(x)=H_u(x)
        \end{equation*}
        
        \begin{equation*}
            H_u^2(x)=H_u(x-2<x,u>u)
        \end{equation*}
        
        y como $H_u$ es lineal
        
        \begin{equation*}
            H_u^2(x)=H_u(x)-2<x,u>H_u(u)
        \end{equation*}
        
        y por el inciso c)
        
        \begin{equation*}
            H_u^2(x)=(x-2<x,u>u)-2<x,u>(-u)
        \end{equation*}
        
        \begin{equation*}
            x\cancel{-2<x,u>u+2<x,u>u}=x
        \end{equation*}
        
        por lo que $H_u^2$ es la identidad, de lo demostrado anteriormente, podemos ver que $H_u^2=H_uH_u^*=H_u^*H_u=I$, o bien $H_u$ es unitario
        
        $\hspace{15cm} \blacksquare$
        
        
        
        
        
        
    \end{enumerate}
\end{enumerate}

\end{document}
