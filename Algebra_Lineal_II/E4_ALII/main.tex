\documentclass[12pt,a4paper]{article}

\usepackage{graphicx}% Include figure files
\usepackage{dcolumn}% Align table columns on decimal point
\usepackage{bm}% bold math
%\usepackage{hyperref}% add hypertext capabilities
%\usepackage[mathlines]{lineno}% Enable numbering of text and display math
%\linenumbers\relax % Commence numbering lines

%\usepackage[showframe,%Uncomment any one of the following lines to test 
%%scale=0.7, marginratio={1:1, 2:3}, ignoreall,% default settings
%%text={7in,10in},centering,
%%margin=1.5in,
%%total={6.5in,8.75in}, top=1.2in, left=0.9in, includefoot,
%%height=10in,a5paper,hmargin={3cm,0.8in},
%]{geometry}

\usepackage{multicol}%Para hacer varias columnas
\usepackage{multicol,caption}
\usepackage{multirow}
\usepackage{cancel}
\usepackage{hyperref}
\hypersetup{
    colorlinks=true,
    linkcolor=blue,
    filecolor=magenta,      
    urlcolor=cyan,
}

\setlength{\topmargin}{-1.0in}
\setlength{\oddsidemargin}{-0.3pc}
\setlength{\evensidemargin}{-0.3pc}
\setlength{\textwidth}{6.75in}
\setlength{\textheight}{9.5in}
\setlength{\parskip}{0.5pc}

\usepackage[utf8]{inputenc}
\usepackage{expl3,xparse,xcoffins,titling,kantlipsum}
\usepackage{graphicx}
\usepackage{xcolor} 
\usepackage{nopageno}
\usepackage{lettrine}
\usepackage{caption}
\renewcommand{\figurename}{Figura}
\usepackage{float}
\renewcommand\refname{Bibliograf\'ia}
\usepackage{amssymb}
\usepackage{amsmath}
\usepackage[rightcaption]{sidecap}
\usepackage[spanish]{babel}

\providecommand{\abs}[1]{\lvert#1\rvert}
\providecommand{\norm}[1]{\lVert#1\rVert}
\newcommand{\dbar}{\mathchar'26\mkern-12mu d}

% CABECERA Y PIE DE PÁGINA %%%%%
\usepackage{fancyhdr}
\pagestyle{fancy}
\fancyhf{}

\begin{document}

\begin{enumerate}
    \item \textbf{Sea $(V,<,>)$ e.v.p.i finito dimensional. Dado $T \in \mathcal{L}(V)$ normal, demostrar que  $T$ es autoadjunto $\leftrightarrow$ $TT^{*}= T^{*}T$ i.e. $T$ es normal}
    
    $\rightarrow)$ por hipótesis $TT = T T^{*}$ y también $TT=T^{*}T$, por lo que $TT^{*}= T^{*}T$ i.e. por definición T es normal  
    
    $\leftarrow)$  Supongamos que $T^{2}=T$ y $TT^{*}=T^{*}T$, entonces
    
    \begin{equation*}
        (T-T^{*}T)^{*} (T-T^{*}T)= (T^{*}-T^{*}^{*}T^{*}) (T-T^{*}T) 
    \end{equation*}
    
    \begin{equation*}
        (T^{*}-TT^{*})(T-T^{*}T) = T^{*}T-T^{*}T^{*}T-TT^{*}T+TT^{*}T^{*}T
    \end{equation*}
    
    \begin{equation*}
        T^{*}T-T^{*}^{2}T-TT^{*}T+TT^{*}^{2}T
    \end{equation*}
    
    y por hipótesis $T^{2}^{*}=T^{*} \rightarrow T^{*}^{2}=T^{*}$
    
    \begin{equation*}
        T^{*}T-T^{*}^{2}T-TT^{*}T+TT^{*}^{2}T=T^{*}T-T^{*}T-TT^{*}T+TT^{*}T = 0
    \end{equation*}
    
    por lo tanto $T-T^{*}T=0$ o bien, $T=T^{*}T$ y también $T^{*}=T^{*}^{*}T^{*}= TT^{*}=T^{*}T$
    \begin{equation*}
        \therefore T=T^{*}
    \end{equation*}
    
    $\hspace{15cm} \blacksquare$
    
    \item \textbf{Sea $(V,<,>)$ e.v.p.i finito dimensional. Dado $T \in \mathcal{L}(V)$ normal. Probar que:}
    \begin{enumerate}
        \item $\text{Ker}(T)= \text{Ker}(T^{*})$
        
        
        $\subseteq)$ Sea $v \in \text{Ker}(T)$, entonces $T(v)= 0_V$, que en su representación matricial
        
        \begin{equation*}
            Tv = 0_V
        \end{equation*}
        
        \begin{equation*}
            T^{*}T v = T^{*}0_V = 0_V
        \end{equation*}
        
        y como T es normal, entonces por definición $TT^{*}= T^{*}T$
        
        \begin{equation*}
            TT^{*} v = 0_V
        \end{equation*}
        
        \begin{equation*}
            (T^{-1}T)T^{*}v = T^{-1} 0_V = 0_V
        \end{equation*}
        
        \begin{equation*}
            IT^{*}v = T^{*}v = 0_V
        \end{equation*}
        
        \begin{equation*}
            T^{*}(v)= 0_V
        \end{equation*}
        
        entonces
        
        \begin{equation*}
            v \in \text{Ker}(T^{*}) \therefore \text{Ker}(T) \subset \text{Ker}(T^{*})
        \end{equation*}
        
$\supseteq)$ Sea $v \in \text{Ker}(T^{*})$, entonces $T^{*}(v)= 0_V$, que en su representación matricial

        \begin{equation*}
            T^{*}v = 0_V
        \end{equation*}
        
        \begin{equation*}
            TT^{*} v = T0_V = 0_V
        \end{equation*}
        
        y como T es normal
        
        \begin{equation*}
            T^{*}T v = 0_V
        \end{equation*}
        
        \begin{equation*}
            ((T^{*})^{-1}T^{*})T^v = (T^{*})^{-1} 0_V = 0_V
        \end{equation*}
        
        \begin{equation*}
            ITv = Tv = 0_V
        \end{equation*}
        
        \begin{equation*}
            T(v)= 0_V
        \end{equation*}

        entonces
        
        \begin{equation*}
            v \in \text{Ker}(T) \therefore \text{Ker}(T^{*}) \subset \text{Ker}(T)
        \end{equation*}
        
        y así, $\text{Ker}(T) = \text{Ker}(T^{*})$ 
        
        $\hspace{15cm} \blacksquare$
        
        
        
        \item $\text{R}(T) = \text{R}(T^{*})$
        
        Por el teorema 2.5 y como $T:V \rightarrow V$, $T^{*}:V \rightarrow V$, 
        
        \begin{equation*}
            N(T) + R(T) = \text{dim}(V) = N(T^{*})+R(T^{*})
        \end{equation*}
        
        pero por el inciso anterior
        
        \begin{equation*}
            \text{Ker}(T)= \text{Ker}(T^{*}) \rightarrow N(T)= N(T^{*})
        \end{equation*}
        
        \begin{equation*}
            \therefore R(T)= R(T^{*})
        \end{equation*}
        
        $\hspace{15cm} \blacksquare$
        
        
    \end{enumerate}
\end{enumerate}

\end{document}
