\documentclass[12pt,a4paper]{article}

\usepackage{graphicx}% Include figure files
\usepackage{dcolumn}% Align table columns on decimal point
\usepackage{bm}% bold math
%\usepackage{hyperref}% add hypertext capabilities
%\usepackage[mathlines]{lineno}% Enable numbering of text and display math
%\linenumbers\relax % Commence numbering lines

%\usepackage[showframe,%Uncomment any one of the following lines to test 
%%scale=0.7, marginratio={1:1, 2:3}, ignoreall,% default settings
%%text={7in,10in},centering,
%%margin=1.5in,
%%total={6.5in,8.75in}, top=1.2in, left=0.9in, includefoot,
%%height=10in,a5paper,hmargin={3cm,0.8in},
%]{geometry}

\usepackage{multicol}%Para hacer varias columnas
\usepackage{multicol,caption}
\usepackage{multirow}
\usepackage{cancel}
\usepackage{hyperref}
\hypersetup{
    colorlinks=true,
    linkcolor=blue,
    filecolor=magenta,      
    urlcolor=cyan,
}

\setlength{\topmargin}{-1.0in}
\setlength{\oddsidemargin}{-0.3pc}
\setlength{\evensidemargin}{-0.3pc}
\setlength{\textwidth}{6.75in}
\setlength{\textheight}{9.5in}
\setlength{\parskip}{0.5pc}

\usepackage[utf8]{inputenc}
\usepackage{expl3,xparse,xcoffins,titling,kantlipsum}
\usepackage{graphicx}
\usepackage{xcolor} 
\usepackage{nopageno}
\usepackage{lettrine}
\usepackage{caption}
\renewcommand{\figurename}{Figura}
\usepackage{float}
\renewcommand\refname{Bibliograf\'ia}
\usepackage{amssymb}
\usepackage{amsmath}
\usepackage[rightcaption]{sidecap}
\usepackage[spanish]{babel}

\providecommand{\abs}[1]{\lvert#1\rvert}
\providecommand{\norm}[1]{\lVert#1\rVert}
\newcommand{\dbar}{\mathchar'26\mkern-12mu d}

% CABECERA Y PIE DE PÁGINA %%%%%
\usepackage{fancyhdr}
\pagestyle{fancy}
\fancyhf{}

\begin{document}

\begin{enumerate}
    \item \textbf{Sea $(V,<,>)$ con $x,y \in V$ dados, definimos $T:V \rightarrow V$  $T(v)=<v,x>y$}
    
    \begin{enumerate}
        \item \textbf{Probar que $T \in \mathcal{L} (V)$}
        
        \textbf{SOLUCIÓN:}
        
        Dado que $\mathcal{L}(V)$ es el espacio de transformaciones lineales de V a V, entonces debemos demostrar que T es una transformación lineal:
        
        Sea $v,w \in  V$ y $c \in K$
        
        \begin{equation*}
            \hspace{1cm} T(v+w) := <v+w,x>y
        \end{equation*}
        
        \begin{equation*}
            \hspace{1cm} T(cv) := <cv,x>y
        \end{equation*}
        
        y por propiedades del producto interno
        
        \begin{equation*}
            T(v+w) := (<v,x> + <w,x>)y = <v,x>y + <w,x>y =: T(v) + T(w)
        \end{equation*}
        
        \begin{equation*}
            T(cv) := c(<v,x>y) =: c T(v)  
        \end{equation*}
        
        y así T es una transformación lineal en V y por lo tanto $T \in \mathcal{L}(V)$ $\hspace{3cm} \blacksquare$
        
        \item \textbf{Demostrar  que $\exists T^{*} \in \mathcal{L}(V)$ y dar explícitamente a $T^{*}$}
        
        \textbf{SOLUCIÓN:}
        
        Por el teorema 6.9 del Friedberg podemos asegurar que existe la transformación lineal de V a V $T^{*}$ $\hspace{12.5cm} \blacksquare$
        
        Tambien nos puede ayudar a encontrar explicitamente $T^{*}$, entonces
        
        \begin{equation*}
            <T(v), w> = <v, T^{*}(w)>
        \end{equation*}
        
        y por definición
        
        \begin{equation*}
            <<v,x>y,w> = <v, T^{*}(w)>
        \end{equation*}
        
        ahora como el producto interno saca escalares
        
        \begin{equation*}
            <v,x><y,w> = <v,T^{*}(w)>
        \end{equation*}
        
        y también los mete en el segundo término pero conjugados
        
        \begin{equation*}
            <v,\overline{<y,w>}x> = <v,T^{*}(w)>
        \end{equation*}
        
        \begin{equation*}
            \therefore  T^{*}(w) = \overline{<y,w>}x
        \end{equation*}
    \end{enumerate}
    
    
    
    \item \textbf{Sean $(V,<,>)$ e.v.p.i finito dimensional, dado $T \in \mathcal{L}(V)$. Probar que}
    \begin{enumerate}
        \item $\text{Ker} (T^{*}) = \text{Img}(T)^{\bot}$
        
        \textbf{SOLUCIÓN:}
        
        Sea $v,w \in V$, por el teorema 6.9 se tiene que
        
        \begin{equation*}
            <T(v), w> = <v, T^{*}(w)>
        \end{equation*}
        
        ahora. supongamos que $w \in Ker (T^{*})$, entonces 
        
        \begin{equation*}
            <v,T^{*}(w)> = <v,0_V>
        \end{equation*}
        
        y por propiedades del producto interior
        
        \begin{equation*}
            <v,T^{*}(w)> = 0
        \end{equation*}
        
        pero entonces por el teorema 6.9
        
        \begin{equation*}
            <T(v),w> = 0
        \end{equation*}
        
        y así, por definición, $w \in \text{Img}(T)^{\perp}$, o bien $\text{Ker} (T^{*}) \subset \text{Img}(T)^{\bot}$
        
        ahora, supongamos que $w \in \text{Img}(T)^{\bot}$, entonces, por definición
        
        \begin{equation*}
            <T(v),w> = 0
        \end{equation*}
        
        pero por el teorema 6.9
        
        \begin{equation*}
            <v,T^{*}(w)>= 0
        \end{equation*}
        
        y como $v$ es arbitrario,necesariamente $T^{*}(w) = 0_V$, o bien $\text{Img}(T)^{\bot} \subset \text{Ker}(T^{*})$
        
        \begin{equation*}
            \therefore  \text{Ker} (T^{*}) = \text{Img}(T)^{\bot}
        \end{equation*}
        
        $\hspace{15cm} \blacksquare$
        
        
        
        \item $\text{Ker} (T) = \text{Img}(T^{*})^{\bot}$
        
        \textbf{SOLUCIÓN:}
        
        Sea $v,w \in V$, por el teorema 6.9 se tiene que
        
        \begin{equation*}
            <T(v), w> = <v, T^{*}(w)>
        \end{equation*}
        
        ahora. supongamos que $v \in Ker (T)$, entonces 
        
        \begin{equation*}
            <T(v),w> = <0_V,w>
        \end{equation*}
        
        y por propiedades del producto interior
        
        \begin{equation*}
            <T(v),w> = 0
        \end{equation*}
        
        pero entonces por el teorema 6.9
        
        \begin{equation*}
            <v,T^{*}(w)> = 0
        \end{equation*}
        
        y así, por definición, $v \in \text{Img}(T^{*})^{\perp}$, o bien $\text{Ker} (T) \subset \text{Img}(T^{*})^{\bot}$
        
        ahora, supongamos que $v \in \text{Img}(T^{*})^{\bot}$, entonces, por definición
        
        \begin{equation*}
            <v,T^{*}(w)> = 0
        \end{equation*}
        
        pero por el teorema 6.9
        
        \begin{equation*}
            <T(v),w>= 0
        \end{equation*}
        
        y como $w$ es arbitrario,necesariamente $T(v) = 0_V$, o bien $\text{Img}(T^{*}^{\bot}) \subset \text{Ker}(T)$
        
        \begin{equation*}
            \therefore     \text{Ker} (T) = \text{Img}(T^{*})^{\bot}  
        \end{equation*}
        
        $\hspace{15cm} \blacksquare$
        
        \item $\text{Img} (T) = \text{Ker}(T^{*})^{\bot}$
        
        \textbf{SOLUCIÓN:}
        
        Sea $v,w \in V$, por el teorema 6.9 se tiene que
        
        \begin{equation*}
            <T(v), w> = <v, T^{*}(w)>
        \end{equation*}
        
        ahora, supongamos que $w \in \text{Ker}(T^{*})$, entonces por el teorema 6.9
        
        \begin{equation*}
            <v,T^{*}(w)> =<v,0_V>=<T(v),w>
        \end{equation*}
        
        y por propiedades del producto interior
        
        \begin{equation*}
            <T(v),w> = 0
        \end{equation*}
        
        entonces $w \in \text{Img}(T)^{\bot}$, o bien $\text{Ker}(T^{*}) \subset \text{Img}(T)^{\bot} \rightarrow \text{Ker}(T^{*})^{\bot} \subset \text{Img}(T)$
        
        ahora, supongamos que $w \in \text{Img}(T)^{\bot}$, entonces por el teorema 6.9
        
        \begin{equation*}
            <T(v),w> = <v,T^{*}(w)>= 0
        \end{equation*}
        
        y como $v$ es arbitrario, entonces $T^{*}(w) = 0$, lo que significa $w \in \text{Ker}(T^{*})$
        
        o bien $\text{Img}(T)^{\bot} \subset \text{Ker}(T^{*}) \rightarrow \text{Img}(T) \subset \text{Ker}(T^{*})^{\bot}$
        
        \begin{equation*}
            \therefore \text{Img}(T) = \text{Ker}(T^{*})^{\bot}
        \end{equation*}
        
        $\hspace{15cm} \blacksquare$
        
        \item $\text{Img} (T^{*}) = \text{Ker}(T)^{\bot}$
        
        \textbf{SOLUCIÓN:}
        
        Sea $v,w \in V$, por el teorema 6.9 se tiene que
        
        \begin{equation*}
            <T(v), w> = <v, T^{*}(w)>
        \end{equation*}
        
        ahora. supongamos que $w \in Ker (T)$, entonces por el teorema 6.9
        
        \begin{equation*}
            <T(v),w> = <v,T^{*}(w)> = 0
        \end{equation*}
        
        entonces, $v \in \text{Img}(T^{*})^{\bot}$, o bien 
        
        $\text{Ker}(T) \subset \text{Img}(T^{*})^{\bot} \rightarrow \text{Ker}(T)^{\bot} \subset (\text{Img}(T^{*})^{\bot})^{\bot}= \text{Img}(T^{*})$
        

        
        ahora, supongamos que $v \in \text{Img}(T^{*})^{\bot}$, entonces por el teorama 6.9
        
        \begin{equation*}
            <v,T^{*}(w)> = <T(v), w> = 0
        \end{equation*}
        
        y como $w$ es arbitrario, entonces $T(v)= 0$, lo que significa $v \in \text{Ker}(T)$,o bien
        
        $\text{Img}(T^{*})^{\bot} \subset \text{Ker}(T)$ $\rightarrow$ $\text{Img}(T^{*}) \subset \text{Ker}(T)^{\bot}$
        
        \begin{equation*}
            \therefore  \text{Img} (T^{*}) = \text{Ker}(T)^{\bot}
        \end{equation*}
        
        $\hspace{15cm} \blacksquare$
        
    \end{enumerate}
\end{enumerate}

\end{document}