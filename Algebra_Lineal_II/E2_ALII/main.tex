\documentclass[12pt,a4paper]{article}

\usepackage{graphicx}% Include figure files
\usepackage{dcolumn}% Align table columns on decimal point
\usepackage{bm}% bold math
%\usepackage{hyperref}% add hypertext capabilities
%\usepackage[mathlines]{lineno}% Enable numbering of text and display math
%\linenumbers\relax % Commence numbering lines

%\usepackage[showframe,%Uncomment any one of the following lines to test 
%%scale=0.7, marginratio={1:1, 2:3}, ignoreall,% default settings
%%text={7in,10in},centering,
%%margin=1.5in,
%%total={6.5in,8.75in}, top=1.2in, left=0.9in, includefoot,
%%height=10in,a5paper,hmargin={3cm,0.8in},
%]{geometry}

\usepackage{multicol}%Para hacer varias columnas
\usepackage{multicol,caption}
\usepackage{multirow}
\usepackage{cancel}
\usepackage{hyperref}
\hypersetup{
    colorlinks=true,
    linkcolor=blue,
    filecolor=magenta,      
    urlcolor=cyan,
}

\setlength{\topmargin}{-1.0in}
\setlength{\oddsidemargin}{-0.3pc}
\setlength{\evensidemargin}{-0.3pc}
\setlength{\textwidth}{6.75in}
\setlength{\textheight}{9.5in}
\setlength{\parskip}{0.5pc}

\usepackage[utf8]{inputenc}
\usepackage{expl3,xparse,xcoffins,titling,kantlipsum}
\usepackage{graphicx}
\usepackage{xcolor} 
\usepackage{nopageno}
\usepackage{lettrine}
\usepackage{caption}
\renewcommand{\figurename}{Figura}
\usepackage{float}
\renewcommand\refname{Bibliograf\'ia}
\usepackage{amssymb}
\usepackage{amsmath}
\usepackage[rightcaption]{sidecap}
\usepackage[spanish]{babel}

\providecommand{\abs}[1]{\lvert#1\rvert}
\providecommand{\norm}[1]{\lVert#1\rVert}

% CABECERA Y PIE DE PÁGINA %%%%%
\usepackage{fancyhdr}
\pagestyle{fancy}
\fancyhf{}

\begin{document}

\begin{enumerate}



%%%1%%%



    \item \textbf{Sean $A,B \in M_{n \times n} (K)$. Demostrar que si $(I-AB)$ es invertible entonces $I-BA$ es invertible (hint:$(I-BA)^{-1} = I + B(I-AB)^{-1} A$).}
    
    \textbf{Probar que AB y BA tienen los mismos eigenvalores en K}
    
    \textbf{SOLUCIÓN:}
    
    Para demostrar que $I-BA$ es invertible, basta con mostrar una matriz que si se multiplica por la derecha o izquierda con $I-BA$ se obtiene la identidad
    
    Supongamos que $(I-BA)^{-1} = I + B(I-AB)^{-1} A$(sabemos por hipótesis que $(I-AB)^{-1}$ existe), entonces
    
    \begin{equation*}
        (I-BA)(I-BA)^{-1} = (I-BA)(I + B(I-AB)^{-1} A)
    \end{equation*}
    
    desarrollando el producto
    
    \begin{equation*}
        = I + B(I-AB)^{-1} A -BA -BAB(I-AB)^{-1} A
    \end{equation*}
    
    ahora factorizando la B por la derecha y A por la izquierda
    
    \begin{equation*}
        = I + B((I-AB)^{-1} -I -AB(I-AB)^{-1} )A
    \end{equation*}
    
    despues reordenando terminso y factorizando $(I-AB)^{-1}$
    
    \begin{equation*}
        = I + B(-I + (I-AB)^{-1} -AB(I-AB)^{-1} )A
    \end{equation*}
    
    \begin{equation*}
        = I + B(-I + \cancel{(I-AB)(I-AB)^{-1}} )A
    \end{equation*}
    
    \begin{equation*}
        = I +B(-I+I)A
    \end{equation*}
    
    pero como $-I+I$ es la matriz 0, entonces
    
    \begin{equation*}
        = I + B(0)A = I \hspace{1cm} \therefore (I-BA)(I-BA)^{-1} = I
    \end{equation*}
    
    ahora veamos por la izquierda
    
    \begin{equation*}
        (I-BA)^{-1} (I-BA) = (I + B(I-AB)^{-1} A) (I-BA)
    \end{equation*}
    
    que igual desarrollando
    
    \begin{equation*}
        = I -BA + B(I-AB)^{-1} A -BAB(I-AB)^{-1} A
    \end{equation*}
    
    hagamos exactamente lo mismo que en el caso anterior
    
    \begin{equation*}
        = I+ B(-I + (I-AB)^{-1} -AB(I-AB)^{-1} )A
    \end{equation*}
    
    \begin{equation*}
        = I+ B(-I + \cancel{(I-AB)(I-AB)^{-1}})A
    \end{equation*}
    
    \begin{equation*}
        = I+ B(-I + I)A = I+B(0)A = I \hspace{1cm} \therefore (I-BA)^{-1} (I-BA) = I
    \end{equation*}
    
    y así confirmamos que existe una inversa y por lo tanto $I-BA$ es invertible bajo el supuesto de que $I-AB$ lo sea $\hspace{13cm} \blacksquare$
    
    Sea $\lambda$ un valor propio de $AB$, con vector propio $v$, entonces
    
    \begin{equation*}
        AB v= \lambda v
    \end{equation*}
    
    \begin{equation*}
        BAB v = B \lambda v
    \end{equation*}
    
    \begin{equation*}
        BA(Bv) = \lambda (Bv)
    \end{equation*}
    
    sea $v' = Bv$
    
    \begin{equation*}
        BAv' = \lambda v'
    \end{equation*}
    
    entonces, $\lambda$ es un valor propio de $BA$ 
    $\hspace{8cm}\blacksquare$
    
    
    
    
    
%%%2%%%



    \item \textbf{Sea $T:R^3 \rightarrow R^3$ el operador lineal dado por }
    
    \begin{equation*}
        T ([x, y, z]) = [2x-y-z,x-z,-x+y+2z]
    \end{equation*}
    
    \textbf{Hallar una base de $R^3$ con relación a la cual la matriz de  T sea diagonal}
    
    \textbf{SOLUCIÓN:}
    
    La matriz de dicha transformación es
    
    \begin{equation*}
        [T]_\beta =\left( \begin{array}{lcc}
            2 & -1 & -1 \\
            \\ 1 & 0 & -1 \\
            \\ -1 & 1 & 2 \\
        \end{array}
        \right)
    \end{equation*}
    
    Con $\beta$ como la base canónica de $R^3$
    
    Ahora calculemos los valores y vectores propios para encontrar una base que diagonalize la matriz de la transformación 
    
    \begin{equation*}
        \text{det}([T]_\beta - \lambda I) = \left|\begin{array}{lcc}
             2- \lambda & -1 & -1 \\
             1 & -\lambda & -1 \\
             -1 & 1 & 2-\lambda \\
        \end{array}
        \right| = (2-\lambda)(-\lambda(2-\lambda)+1)+((2-\lambda)-1) -(1-\lambda)
    \end{equation*}
    
    \begin{equation*}
        = (2-\lambda)(-2\lambda+\lambda^2+1)-\lambda+1 +\lambda-1 =-4\lambda+2\lambda^2+2 +2\lambda^2 -\lambda^3- \lambda 
    \end{equation*}
    
    \begin{equation*}
        =-\lambda^3+4\lambda^2-5\lambda +2 = -(\lambda-1)^2(\lambda-2) = 0
    \end{equation*}
    
    por lo que los valores propios son $\lambda_1 =1$ con multiplicidad 2 y $\lambda_2 = 2$, que sustituyendo
    
    \begin{equation*}
        ([T]_\beta - \lambda_1 I)v = \left(\begin{array}{lcc}
             2- 1 & -1 & -1 \\
             1 & -1 & -1 \\
             -1 & 1 & 2-1 \\
        \end{array}\right)\left(\begin{array}{lcc}
             x  \\
             y \\
             z \\
        \end{array}\right) = \left(\begin{array}{lcc}
             x-y-z \\
             x-y-z \\
             -x+y+z \\
        \end{array}\right) =\left(\begin{array}{lcc}
             0  \\
             0 \\
             0 \\
        \end{array} \right)
    \end{equation*}
    
    \begin{equation*}
        \longrightarrow  \left\{\begin{array}{cc}
             x-y-z = 0  \\
             x-y-z = 0 \\
             -x+y+z = 0 \\
        \end{array}\right. \longrightarrow
        \left\{\begin{array}{cc}
             x = y+z  \\
             y = x-z \\
        \end{array}\right.
    \end{equation*}
    
    que de aquí podemos sacar los vectores propios $\left(\begin{array}{cc}
         1  \\
         1 \\
         0 \\
    \end{array}\right)$ y $\left(\begin{array}{cc}
         1 \\
         0 \\
         1 \\
    \end{array}\right)$
    
    ahora para $\lambda_2$
    
    \begin{equation*}
        ([T]_\beta - \lambda_2 I)v = \left(\begin{array}{lcc}
             2- 2 & -1 & -1 \\
             1 & -2 & -1 \\
             -1 & 1 & 2-2 \\
        \end{array}\right)\left(\begin{array}{lcc}
             x  \\
             y \\
             z \\
        \end{array}\right) = \left(\begin{array}{lcc}
             -y-z \\
             x-2y-z \\
             -x+y \\
        \end{array}\right) =\left(\begin{array}{lcc}
             0  \\
             0 \\
             0 \\
        \end{array} \right)
    \end{equation*}
    
    \begin{equation*}
        \longrightarrow  \left\{\begin{array}{cc}
             -y-z = 0  \\
             x-2y-z = 0 \\
             -x+y = 0 \\
        \end{array}\right. \longrightarrow
        \left\{\begin{array}{cc}
             y = -z  \\
             z = -2y+x \\
             x = y \\
        \end{array}\right. \longrightarrow  \left\{\begin{array}{cc}
             y = -z  \\
             z = -y \\
             x = y \\
        \end{array}\right.
    \end{equation*}
    
    y así, en la base $\beta' =\left\{ \left(\begin{array}{cc}
         1  \\
         1 \\
         0 \\
    \end{array}\right),\left(\begin{array}{cc}
         1 \\
         0 \\
         1 \\
    \end{array}\right),\left(\begin{array}{cc}
         -1 \\
         -1 \\
         1 \\
    \end{array}\right)\right\}$
    
    comprobemos que la matriz $[T]_{\beta'}$ es diagonal
    
    Sea 
    
    \begin{equation*}
        Q = \left(\begin{array}{lcc}
            1 & 1 & -1 \\
            1 & 0 & -1 \\
            0 & 1 & 1 \\
        \end{array}\right)
    \end{equation*}
    
    \begin{equation*}
        Q^{-1} = \left(\begin{array}{rcc}
            -1 & 2 & 1 \\
            1&-1 & 0 \\
            -1 & 1 & 1 \\
        \end{array}\right)
    \end{equation*}
    
    entonces
    
    \begin{equation*}
        [T]_{\beta'} = Q^{-1}[T]_\beta Q =\left(\begin{array}{rcc}
            -1 & 2 & 1 \\
            1&-1 & 0 \\
            -1 & 1 & 1 \\
        \end{array}\right)\left( \begin{array}{lcc}
            2 & -1 & -1 \\
            \\ 1 & 0 & -1 \\
            \\ -1 & 1 & 2 \\
        \end{array}
        \right)\left(\begin{array}{lcc}
            1 & 1 & -1 \\
            1 & 0 & -1 \\
            0 & 1 & 1 \\
        \end{array}\right) 
    \end{equation*} 
    
    \begin{equation*}
        =\left( \begin{array}{lcc}
            1 & 0 & 0 \\
            \\ 0 & 1 & 0 \\
            \\ 0 & 0 & 2 \\
        \end{array}
        \right)
    \end{equation*}
\end{enumerate}


\end{document}