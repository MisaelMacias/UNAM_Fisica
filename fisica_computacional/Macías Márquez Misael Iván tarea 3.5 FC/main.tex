\documentclass[12pt,a4paper]{article}

\usepackage{graphicx}% Include figure files
\usepackage{dcolumn}% Align table columns on decimal point
\usepackage{bm}% bold math
%\usepackage{hyperref}% add hypertext capabilities
%\usepackage[mathlines]{lineno}% Enable numbering of text and display math
%\linenumbers\relax % Commence numbering lines

%\usepackage[showframe,%Uncomment any one of the following lines to test 
%%scale=0.7, marginratio={1:1, 2:3}, ignoreall,% default settings
%%text={7in,10in},centering,
%%margin=1.5in,
%%total={6.5in,8.75in}, top=1.2in, left=0.9in, includefoot,
%%height=10in,a5paper,hmargin={3cm,0.8in},
%]{geometry}

\usepackage{multicol}%Para hacer varias columnas
\usepackage{multicol,caption}
\usepackage{multirow}
\usepackage{cancel}
\usepackage{hyperref}
\hypersetup{
    colorlinks=true,
    linkcolor=blue,
    filecolor=magenta,      
    urlcolor=cyan,
}

\setlength{\topmargin}{-1.0in}
\setlength{\oddsidemargin}{-0.3pc}
\setlength{\evensidemargin}{-0.3pc}
\setlength{\textwidth}{6.75in}
\setlength{\textheight}{9.5in}
\setlength{\parskip}{0.5pc}

\usepackage[utf8]{inputenc}
\usepackage{expl3,xparse,xcoffins,titling,kantlipsum}
\usepackage{graphicx}
\usepackage{xcolor} 
\usepackage{nopageno}
\usepackage{lettrine}
\usepackage{caption}
\renewcommand{\figurename}{Figura}
\usepackage{float}
\renewcommand\refname{Bibliograf\'ia}
\usepackage{amssymb}
\usepackage{amsmath}
\usepackage[rightcaption]{sidecap}
\usepackage[spanish]{babel}

\providecommand{\abs}[1]{\lvert#1\rvert}
\providecommand{\norm}[1]{\lVert#1\rVert}
\newcommand{\dbar}{\mathchar'26\mkern-12mu d}

% CABECERA Y PIE DE PÁGINA %%%%%
\usepackage{fancyhdr}
\pagestyle{fancy}
\fancyhf{}

\begin{document}

\begin{verbatim}
import numpy as np
import math as mh

temp = "s"
op = ""
k = 0
x = 0.
y = 0
z = 0.

sen = 0.
senr = 0.
cos = 0.
cosr = 0.
epsilon = 0.
lista = np.zeros([300,])

while ((temp == "S") or (temp == "s")):
#bucle para mantener el programa corriendo hasta que se requiera
    op = input("¿Qué función quieres aproximar?(sen/cos) \n")
    
    if ((op != "sen") and (op != "cos")) :
                        
                        print("Esa opción no está disponible :C")
                
    else:
        
            try:
            #excepción para mantener el flujo del programa si se ingresan valores inválidos
            
                    k = int(input("¿Cuántos términos quieres?\n"))
                    #print(type(k))
                    x = float(input("¿Cuál es el valor a evaluar?(en radianes)\n"))
                    #print(type(x))
            
            except :
            
                print("valor inválido :C")
            
            else : 
            
                    y = mh.floor(x / np.pi)  % 2
                
                    if (np.absolute(x) > np.pi):
                    #declaración para evitar problemas de rango sobre el cálculo de los elementos de la serie 
                
                        z = x - (((mh.floor(x / np.pi)) + (y)) * np.pi)
                    
                    else :
                        
                        z = x
                    
                    if (k > 0) :
                    #declaración para comprobar validez de parámetros
                
                        if op == "sen":
                        #declaración para comprobar validez de la opción elegida
                    
                            for i in range(k):
                            #bucle para producir los elementos de la serie
                    
                                sen += (-1)**i/(np.math.factorial(2*i + 1)) * z**(2*i + 1)
            
                            senr = np.sin(x)
            
                            print("El valor aproximado del seno en " + str(x) + " es : " + str(sen))
                            #print(lista)
                            print("El valor \"real\"(obtenido por la función nativa de numpy [np.sin(" + str(x) + ")]) es : " + str(senr))
                            print("El error porcentual es :" + str(np.abs((senr/sen)-1)))
                            
                
                        elif op == "cos":
                        #declaración para comprobar validez de la opción elegida
                    
                            for i in range(k):
                            #bucle para producir los elementos de la serie
                
                                cos += ((-1)**i/(np.math.factorial(2*i))) * z**(2*i)
                    
            
                            cosr = np.cos(x)
            
                            print("El valor aproximado del cos en " + str(x) + " es : " + str(cos))
                            #print(lista)
                            print("El valor \"real\"(obtenido por la función nativa de numpy [np.cos(" + str(x) + ")]) es : " + str(cosr))
                            print("El error porcentual es :" + str(np.abs((cosr/cos) - 1)))
            
                        
                    

    temp = input("¿Quieres intentarlo de nuevo?(S/N) \n")
    

   
print("bye C:")
\end{verbatim}

\end{document}