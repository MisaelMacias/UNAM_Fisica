 \documentclass[12pt,a4paper]{article}

\usepackage{graphicx}% Include figure files
\usepackage{dcolumn}% Align table columns on decimal point
\usepackage{bm}% bold math
%\usepackage{hyperref}% add hypertext capabilities
%\usepackage[mathlines]{lineno}% Enable numbering of text and display math
%\linenumbers\relax % Commence numbering lines

%\usepackage[showframe,%Uncomment any one of the following lines to test 
%%scale=0.7, marginratio={1:1, 2:3}, ignoreall,% default settings
%%text={7in,10in},centering,
%%margin=1.5in,
%%total={6.5in,8.75in}, top=1.2in, left=0.9in, includefoot,
%%height=10in,a5paper,hmargin={3cm,0.8in},
%]{geometry}

\usepackage{multicol}%Para hacer varias columnas
\usepackage{multicol,caption}
\usepackage{multirow}
\usepackage{cancel}
\usepackage{hyperref}
\hypersetup{
    colorlinks=true,
    linkcolor=blue,
    filecolor=magenta,      
    urlcolor=cyan,
}

\setlength{\topmargin}{-1.0in}
\setlength{\oddsidemargin}{-0.3pc}
\setlength{\evensidemargin}{-0.3pc}
\setlength{\textwidth}{6.75in}
\setlength{\textheight}{9.5in}
\setlength{\parskip}{0.5pc}

\usepackage[utf8]{inputenc}
\usepackage{expl3,xparse,xcoffins,titling,kantlipsum}
\usepackage{graphicx}
\usepackage{xcolor} 
\usepackage{siunitx}
\usepackage{nopageno}
\usepackage{lettrine}
\usepackage{caption}
\renewcommand{\figurename}{Figura}
\usepackage{float}
\renewcommand\refname{Bibliograf\'ia}
\usepackage{amssymb}
\usepackage{amsmath}
\usepackage[rightcaption]{sidecap}
\usepackage[spanish]{babel}

\providecommand{\abs}[1]{\lvert#1\rvert}
\providecommand{\norm}[1]{\lVert#1\rVert}
\newcommand{\dbar}{\mathchar'26\mkern-12mu d}

\usepackage{mathtools}
\DeclarePairedDelimiter\bra{\langle}{\rvert}
\DeclarePairedDelimiter\ket{\lvert}{\rangle}
\DeclarePairedDelimiterX\braket[2]{\langle}{\rangle}{#1 \delimsize\vert #2}

% CABECERA Y PIE DE PÁGINA %%%%%
\usepackage{fancyhdr}
\pagestyle{fancy}
\fancyhf{}

\begin{document}

Macías Márquez Misael Iván

\begin{enumerate}



%%%1%%%



\item ¿Qué es una transformada de Legendre y por qué se usa en termodinámica?

\textbf{Res:}

La transformada de Legendre es un método para que cierta función que cumple algunas condiciones cambie su dependencia a otras variables.

En termodinámica se usa para cambiar de las variables extensivas a las intensivas principalmente en los potenciales termodinámicos.



%%%2%%%



\item ¿Los potenciales termodinámicos tiene interpretación física o solamente son un artilugio matemático para facilitar algunas cosas?

\textbf{Res:}

Tienen una interpretación física y también son artilugios matemáticos xd, supongo que la interpretación de cada potencial la debo dar en la siguiente pregunta.



%%%3%%%



\item Escriba lo que entiende por energía libre de Helmholtz, Entalpía y energía de Gibbs.

\textbf{Res:}

La energía libre de Hemholtz nos da el trabajo útil de un sistema en un proceso iscórico e isotérmico.

La entalpía es el calor que se puede mover en un sistema que sufre un proceso adiabático.

La energía de Gibbs es lo máximo de trabajo que se puede sacar de un proceso químico, además es útil para determinar que tan espontáneo será el proceso.



%%%4%%%



\item Las relaciones de Maxwell sientan su principio en un importante teorema sobre segundas derivadas parciales, físicamente ¿qué estamos suponiendo sobre la teoría termodinámica, y sus variable en juego, que nos permite construir estas relaciones?

\textbf{Res:}

Estamos suponiendo que cumplen con el teorema de Schwarz, o bien que la función que se deriva es continua y que por supuesto existen.





%%%5%%%



\item Escriba y explique la Tercera Ley de la Termodinámica

\textbf{Res:}

Existen varios enunciados de la tercera ley de la termodinámica así que solo pondré el dado por Nerst:

'Cerca del cero absoluto, todas las reacciones en un sistema en equilibrio interno tienen lugar sin cambio en la entropía'

Esto quiere decir que cuando la temperatura de un sistema en equilibrio interno tiende a $0 K$ la entropía también tiende a $0$ .







%%%6%%%



\item La versión popular del principio de incertidumbre de Heisenberg establece que, en un sistema cuántico, es imposible conocer simultáneamente la posición y el momento lineal de un cierto objeto ¿cómo se relaciona este principio con la tercera ley?

\textbf{Res:}

Creo que su relación se puede ver desde el punto de vista de la física estadística donde la entropía se define como $S = k_b \ln{\Omega}$, ya que si la entropía tiende a cero la cantidad de microestados se ve forzada a ser uno lo que estaría determinando por completo nuestro sistema violando así el principio de incertidumbre de Heisenberg.


    
    
\end{enumerate}

\end{document}