 \documentclass[12pt,a4paper]{article}

\usepackage{graphicx}% Include figure files
\usepackage{dcolumn}% Align table columns on decimal point
\usepackage{bm}% bold math
%\usepackage{hyperref}% add hypertext capabilities
%\usepackage[mathlines]{lineno}% Enable numbering of text and display math
%\linenumbers\relax % Commence numbering lines

%\usepackage[showframe,%Uncomment any one of the following lines to test 
%%scale=0.7, marginratio={1:1, 2:3}, ignoreall,% default settings
%%text={7in,10in},centering,
%%margin=1.5in,
%%total={6.5in,8.75in}, top=1.2in, left=0.9in, includefoot,
%%height=10in,a5paper,hmargin={3cm,0.8in},
%]{geometry}

\usepackage{multicol}%Para hacer varias columnas
\usepackage{multicol,caption}
\usepackage{multirow}
\usepackage{cancel}
\usepackage{hyperref}
\hypersetup{
    colorlinks=true,
    linkcolor=blue,
    filecolor=magenta,      
    urlcolor=cyan,
}

\setlength{\topmargin}{-1.0in}
\setlength{\oddsidemargin}{-0.3pc}
\setlength{\evensidemargin}{-0.3pc}
\setlength{\textwidth}{6.75in}
\setlength{\textheight}{9.5in}
\setlength{\parskip}{0.5pc}

\usepackage[utf8]{inputenc}
\usepackage{expl3,xparse,xcoffins,titling,kantlipsum}
\usepackage{graphicx}
\usepackage{xcolor} 
\usepackage{siunitx}

\usepackage{nopageno}
\usepackage{lettrine}
\usepackage{caption}
\renewcommand{\figurename}{Figura}
\usepackage{float}
\renewcommand\refname{Bibliograf\'ia}
\usepackage{amssymb}
\usepackage{amsmath}
\usepackage[rightcaption]{sidecap}
\usepackage[spanish]{babel}

\providecommand{\abs}[1]{\lvert#1\rvert}
\providecommand{\norm}[1]{\lVert#1\rVert}
\newcommand{\dbar}{\mathchar'26\mkern-12mu d}

\usepackage{mathtools}
\DeclarePairedDelimiter\bra{\langle}{\rvert}
\DeclarePairedDelimiter\ket{\lvert}{\rangle}
\DeclarePairedDelimiterX\braket[2]{\langle}{\rangle}{#1 \delimsize\vert #2}

% CABECERA Y PIE DE PÁGINA %%%%%
\usepackage{fancyhdr}
\pagestyle{fancy}
\fancyhf{}
\spanishdecimal{.}

\begin{document}

Macías Márquez Misael Iván

\begin{enumerate}



%%%1%%%



\item El ángulo de incidencia $\theta$ y el ángulo de refracción $\theta'$, entre dos medios con índices de refracción $n$ y $n'$ respectivamente, se relacionan mediante la formula $6.1$. Linealizar dicha ecuación de la forma más sencilla posible cuando la variable independiente es $\theta$ y la dependiente es $\theta'$. Indicar la pendiente y la ordenada al origen.

\textbf{Sol:}

La ecuación $6.1$ es:

\begin{equation*}
    n\sin{\theta}=n'\sin{\theta'}
\end{equation*}

o bien

\begin{equation*}
    \sin{\theta'}=\frac{n}{n'}\sin{\theta}
\end{equation*}

donde $y=\sin{\theta'}$, $x=\sin{\theta}$, $m=\frac{n}{n'}$ y $b= 0$.



%%%2%%%




\item Calcular la fórmula del error para el índice de refracción $n_T$ en la fórmula $6.3$  asumiendo que el medio en el que está inmerso el material es aire ($n_{aire}=1$) y que la incertidumbre del ángulo crítico es $\Delta \theta_C$.

\textbf{Sol:}

La ecuación $6.3$ es:

\begin{equation*}
    \theta_C = \sin^{-1}{\frac{n_T}{n_I}}
\end{equation*}

y despejando el índice de refracción $n_I$ y sustituyendo $n_T$:

\begin{equation*}
    n_I =\frac{n_T}{\sin{\theta_C}}=\frac{n_{aire}}{\sin{\theta_C}} =\frac{1}{\sin{\theta_C}}
\end{equation*}

propagando la incertidumbre para $n_T$:

\begin{equation*}
    \Delta n_I= \sqrt{\left(\frac{\partial n_I}{\partial \theta_C}\right)^{2}\Delta \theta_{C}^{2}}= \frac{\cos{\theta_C}\Delta \theta_C}{\sin^{2}{\theta_C}}
\end{equation*}


%%%3%%%



\item En la fórmula para el método de Pfund (ecuación $6.4$), asumiendo que la placa está rodeada de aire, y que $D$ y $h$ son mediciones con incertidumbres $\Delta D$ y $\Delta h$ respectivamente. Deducir la formula de propagación del error para el índice de refracción $n_p$.

\textbf{Sol:}

La ecuación $6.3$ es:

\begin{equation*}
    \theta_C = \sin^{-1}{\frac{n_T}{n_I}}
\end{equation*}

y despejando el índice de refracción $n_P$ y sustituyendo $n_T$:

\begin{equation*}
    n_P = n_T \frac{\sqrt{D^2 + 16h^2}}{D}= n_{aire} \frac{\sqrt{D^2 + 16h^2}}{D}= \frac{\sqrt{D^2 + 16h^2}}{D}
\end{equation*}

propagando la incertidumbre para $n_P$:

\begin{equation*}
    \Delta n_P=\sqrt{\left(\frac{\partial n_P}{\partial D} \Delta D\right)^{2}+ \left(\frac{\partial n_P}{\partial h}\Delta h\right)^{2}}
\end{equation*}



%%%4%%%




\item De la ecuación $6.5$, calcular la fórmula de la propagación de error para esta fórmula asumiendo que $r$ y $a$ son mediciones experimentales.

\textbf{Sol:}

La ecuación $6.5$ es:

\begin{equation*}
    n_I = \frac{r}{a}n_T
\end{equation*}

propagando la incertidumbre para $n_I$:

\begin{equation*}
    \Delta n_I = \sqrt{\left(\frac{\partial n_I}{\partial r}\Delta r\right)^{2} + \left(\frac{\partial n_I}{\partial a}\Delta a\right)^{2}}
\end{equation*}

\begin{equation*}
    = \sqrt{\left(\frac{n_T\Delta r}{a}\right)^{2} + \left(-\frac{rn_T\Delta a}{a^2}\right)^{2}}
\end{equation*}

\begin{equation*}
    = \frac{n_T}{a^2}\sqrt{a^2\Delta r^{2}+r^2\Delta a^2}
\end{equation*}



%%%5%%%



\item Calcular la fórmula de propagación para la ecuación $6.7$ tomando $\alpha$ constante.

\textbf{Sol:}

La ecuación $6.7$ es:

\begin{equation*}
    n= \frac{\sin{\frac{\delta_m + \alpha}{2}}}{\sin{\frac{\alpha}{2}}}
\end{equation*}

propagando la incertidumbre para $n$:

\begin{equation*}
    \Delta n = \sqrt{\left(\frac{\partial n}{\partial\delta_m}\Delta \delta_m\right)^{2}} = \left|\frac{\partial n}{\partial \delta_m}\right|\Delta \delta_m
\end{equation*}

\begin{equation*}
    = \frac{1}{2}\left|\frac{\cos{\frac{\delta_m + \alpha}{2}}}{\sin{\frac{\alpha}{2}}} \right| \Delta \delta_m
\end{equation*}
    
\end{enumerate}

\end{document}