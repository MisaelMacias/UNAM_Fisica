 \documentclass[12pt,a4paper]{article}

\usepackage{graphicx}% Include figure files
\usepackage{dcolumn}% Align table columns on decimal point
\usepackage{bm}% bold math
%\usepackage{hyperref}% add hypertext capabilities
%\usepackage[mathlines]{lineno}% Enable numbering of text and display math
%\linenumbers\relax % Commence numbering lines

%\usepackage[showframe,%Uncomment any one of the following lines to test 
%%scale=0.7, marginratio={1:1, 2:3}, ignoreall,% default settings
%%text={7in,10in},centering,
%%margin=1.5in,
%%total={6.5in,8.75in}, top=1.2in, left=0.9in, includefoot,
%%height=10in,a5paper,hmargin={3cm,0.8in},
%]{geometry}

\usepackage{multicol}%Para hacer varias columnas
\usepackage{multicol,caption}
\usepackage{multirow}
\usepackage{cancel}
\usepackage{hyperref}
\hypersetup{
    colorlinks=true,
    linkcolor=blue,
    filecolor=magenta,      
    urlcolor=cyan,
}

\setlength{\topmargin}{-1.0in}
\setlength{\oddsidemargin}{-0.3pc}
\setlength{\evensidemargin}{-0.3pc}
\setlength{\textwidth}{6.75in}
\setlength{\textheight}{9.5in}
\setlength{\parskip}{0.5pc}

\usepackage[utf8]{inputenc}
\usepackage{expl3,xparse,xcoffins,titling,kantlipsum}
\usepackage{graphicx}
\usepackage{xcolor} 
\usepackage{siunitx}

\usepackage{nopageno}
\usepackage{lettrine}
\usepackage{caption}
\renewcommand{\figurename}{Figura}
\usepackage{float}
\renewcommand\refname{Bibliograf\'ia}
\usepackage{amssymb}
\usepackage{amsmath}
\usepackage[rightcaption]{sidecap}
\usepackage[spanish]{babel}

\providecommand{\abs}[1]{\lvert#1\rvert}
\providecommand{\norm}[1]{\lVert#1\rVert}
\newcommand{\dbar}{\mathchar'26\mkern-12mu d}

\usepackage{mathtools}
\DeclarePairedDelimiter\bra{\langle}{\rvert}
\DeclarePairedDelimiter\ket{\lvert}{\rangle}
\DeclarePairedDelimiterX\braket[2]{\langle}{\rangle}{#1 \delimsize\vert #2}

% CABECERA Y PIE DE PÁGINA %%%%%
\usepackage{fancyhdr}
\pagestyle{fancy}
\fancyhf{}
\spanishdecimal{.}

\begin{document}

Macías Márquez Misael Iván

\begin{enumerate}


%%%1%%%



\item Luz incide en agua ($n_{\text{agua}}=1.33$) desde el aire ($n_{\text{aire}}=1$). ¿Cuál es el valor del ángulo de Brewster?

\textbf{Sol}:

Sustituyendo en la ecuación 9.9:

\begin{equation*}
    \theta_B = \tan^{-1}{\left(\frac{n_{\text{agua}}}{n_{\text{aire}}}\right)} = \tan^{-1}{\left(\frac{1.33}{1}\right)} = 53.06 ^{\circ}
\end{equation*}



%%%2%%%




\item ¿Qué ocurre cuando la luz con PCI incide en un polarizador lineal con ET a un ángulo $\theta$ arbitrario?, Físicamente, ¿qué significa esto para la intensidad cuando se gira un polarizador? (Utilizar matrices para justificar).

\textbf{Sol}:

Pues creo que solo la componente sobre el ET pasará y oscilará sobre el mismo.

La matriz de polarización lineal PL es:

\begin{equation*}
    PL=E_0 \left(\begin{array}{cc}
         \cos{\alpha}  \\
         \sin{\alpha} 
    \end{array}\right)
\end{equation*}

y la matriz de rotación en el plano es:

\begin{equation*}
    \left(\begin{array}{cc}
         \cos{\theta} & -\sin{\theta}  \\
         \sin{\theta} & \cos{\theta}
    \end{array}\right)
\end{equation*}

que al multiplicarse:

\begin{equation*}
    E_0 \cdot \left(\begin{array}{cc}
         \cos{\theta} & -\sin{\theta}  \\
         \sin{\theta} & \cos{\theta}\end{array}\right) \cdot \left(\begin{array}{cc}
         \cos{\alpha}  \\
         \sin{\alpha} \end{array}\right) = E_0 \left(\begin{array}{cc}
         \cos{\theta}\cos{\alpha}-\sin{\theta}\sin{\alpha}  \\
         \sin{\theta}\cos{\alpha}+\cos{\theta}\sin{\alpha}   \end{array}\right) 
\end{equation*}

\begin{equation*}
    = E_0 \left(\begin{array}{cc}
         \cos{(\theta +\alpha)}  \\
         \sin{(\theta +\alpha)} \end{array}\right)
\end{equation*}







%%%3%%%



\item Luz PL a $45^{\circ}$ e intensidad $I_0$ incide en un polarizador lineal con ET a $0^{\circ}$ seguido de un analizador con ET a $45^{\circ}$. ¿Cuál es la polarización e intensidad de la luz de salida?(Usar matrices).

\textbf{Sol}:

\begin{equation*}
    \left(\begin{array}{cc}
        \cos^{2}{\theta} & \sin{\theta} \cos{\theta}  \\
        \sin{\theta}\cos{\theta} & \sin^{2}{\theta} 
    \end{array}\right) \cdot E_0  \left(\begin{array}{c}
          \cos{\beta} \\
          \sin{\beta}
    \end{array}\right)=E_0 \cos{(\theta - \beta)} \left(\begin{array}{c}
         \cos{\theta} \\
         \sin{\theta} 
    \end{array}\right)
\end{equation*}

sustituyendo $\theta= 45^{\circ}$ y $\beta = 0^{\circ}$:

\begin{equation*}
    E_{0} \cos{(45^{\circ} - 0 ^{\circ})}\left(\begin{array}{c}
         \cos{45^{\circ}}  \\
         \sin{45^{\circ}}
    \end{array}\right)=\frac{E_0}{2} \left(\begin{array}{c}
         1  \\
          1
    \end{array}\right)
\end{equation*}





%%%4%%%



\item Si dos polarizadores tienen sus ejes a $90^{\circ}$ (ortogonales), la ley de Malus predice que la luz a la salida será cero. ¿Qué ocurriría si un tercer polarizador se coloca entre los dos polarizadores ortogonales con su eje a $45^{\circ}$ del primer polarizador? (Utilizar matrices para justificar la respuesta y luz PL paralela al primer polarizador).

\textbf{Sol}:

por la ley de Malus sabemos que para el primer polarizador la intensidad de salida es $\frac{I_0}{2}$ o bien: 

\begin{equation*}
    I_{0} \cos{(45^{\circ} - 0 ^{\circ})}\left(\begin{array}{c}
         \cos{45^{\circ}}  \\
         \sin{45^{\circ}}
    \end{array}\right)=\frac{I_0}{2} \left(\begin{array}{c}
         1  \\
          1
    \end{array}\right)
\end{equation*}

entonces aplicando de nuevo la ley de malus 2 veces:

\begin{equation*}
    \frac{I_{0}}{2} \cos{(45^{\circ} - 0 ^{\circ})}\left(\begin{array}{c}
         \cos{45^{\circ}}  \\
         \sin{45^{\circ}}
    \end{array}\right)=\frac{I_0}{4} \left(\begin{array}{c}
         1  \\
          1
    \end{array}\right)
\end{equation*}


\begin{equation*}
    \frac{I_{0}}{4} \cos{(45^{\circ} - 0 ^{\circ})}\left(\begin{array}{c}
         \cos{45^{\circ}}  \\
         \sin{45^{\circ}}
    \end{array}\right)=\frac{I_0}{8} \left(\begin{array}{c}
         1  \\
          1
    \end{array}\right)
\end{equation*}
    
    
\end{enumerate}

\end{document}