 \documentclass[12pt,a4paper]{article}

\usepackage{graphicx}% Include figure files
\usepackage{dcolumn}% Align table columns on decimal point
\usepackage{bm}% bold math
%\usepackage{hyperref}% add hypertext capabilities
%\usepackage[mathlines]{lineno}% Enable numbering of text and display math
%\linenumbers\relax % Commence numbering lines

%\usepackage[showframe,%Uncomment any one of the following lines to test 
%%scale=0.7, marginratio={1:1, 2:3}, ignoreall,% default settings
%%text={7in,10in},centering,
%%margin=1.5in,
%%total={6.5in,8.75in}, top=1.2in, left=0.9in, includefoot,
%%height=10in,a5paper,hmargin={3cm,0.8in},
%]{geometry}

\usepackage{multicol}%Para hacer varias columnas
\usepackage{multicol,caption}
\usepackage{multirow}
\usepackage{cancel}
\usepackage{hyperref}
\hypersetup{
    colorlinks=true,
    linkcolor=blue,
    filecolor=magenta,      
    urlcolor=cyan,
}

\setlength{\topmargin}{-1.0in}
\setlength{\oddsidemargin}{-0.3pc}
\setlength{\evensidemargin}{-0.3pc}
\setlength{\textwidth}{6.75in}
\setlength{\textheight}{9.5in}
\setlength{\parskip}{0.5pc}

\usepackage[utf8]{inputenc}
\usepackage{expl3,xparse,xcoffins,titling,kantlipsum}
\usepackage{graphicx}
\usepackage{xcolor} 
\usepackage{siunitx}

\usepackage{nopageno}
\usepackage{lettrine}
\usepackage{caption}
\renewcommand{\figurename}{Figura}
\usepackage{float}
\renewcommand\refname{Bibliograf\'ia}
\usepackage{amssymb}
\usepackage{amsmath}
\usepackage[rightcaption]{sidecap}
\usepackage[spanish]{babel}

\providecommand{\abs}[1]{\lvert#1\rvert}
\providecommand{\norm}[1]{\lVert#1\rVert}
\newcommand{\dbar}{\mathchar'26\mkern-12mu d}

\usepackage{mathtools}
\DeclarePairedDelimiter\bra{\langle}{\rvert}
\DeclarePairedDelimiter\ket{\lvert}{\rangle}
\DeclarePairedDelimiterX\braket[2]{\langle}{\rangle}{#1 \delimsize\vert #2}

% CABECERA Y PIE DE PÁGINA %%%%%
\usepackage{fancyhdr}
\pagestyle{fancy}
\fancyhf{}
\spanishdecimal{.}

\begin{document}

Macías Márquez Misael Iván

\begin{enumerate}



%%%1%%% 



\item Calcule la energía del primer estado excitado del átomo de helio, usando perturbaciones independientes del tiempo, para estados degenerados.

\textbf{Sol:}

La primera configuración excitada del helio es $1s^{1}2s^{1}$


%%%2%%%



\item Una base de vectores propios para $\hat{J}^{2}= (\hat{J}_1 + \hat{J}_2)^{2}$ y su proyección $\hat{J}_{z}= \hat{J}_{z1} + \hat{J}_{z2}$ puede representarse como $|\ket{jm}$, de modo que $\hat{J}^2|\ket{jm} = j(j+1)|\ket{jm}$ y $\hat{J}_{z} |\ket{jm} =m\hbar |\ket{jm}$. El valor propio máximo de la proyección en unidades de $\hbar$ es $m_{max}=(m_1 + m_2)_{max}=j_1 + j_2$.

\begin{enumerate}
    \item ¿Cuál es el valor máximo de $j$?
    
    \textbf{Sol:}
    
    
    \item Encontrar el valor mínimo de $j$ utilizando el hecho de que el número de estados en la representación de estados propios de $\hat{J}_1$ y $\hat{J}_2$ representado por |$\ket{j_1 m_1}$, $|\ket{j_2 m_2}$ y en la representación $|\ket{jm}$ se conserva; esto es
    
    \begin{equation*}
        \sum_{j_{min}}^{j_{max}}(2j+1)=(2j_1 + 1)(2j_2 +1)
    \end{equation*}
    
    \textbf{Sol:}
    
    
\end{enumerate}



%%%3%%%



\item Deduzca el símbolo de término para el estado base para los siguientes átomos:

\begin{enumerate}
    \item Zirconio cuya configuración es $[Kr](4d)^2(5s)^2$.
    
    \textbf{Sol:}
    
    Para el zirconio se tiene la configuración electrónica $nd^2$ y por lo visto en la tarea anterior, el símbolo de término para el estado base es $^{3}F_{2}$.
    
    \item Paladio cuya configuración electrónica es $[Kr](4d)^{10}$
    
    \textbf{Sol:}
    
    Dado que todas las subcapas del paladio están llenas se tiene que el simbolo de termino para el estado base es $^{1}S_{0}$.
\end{enumerate}



%%%4%%%



\item Sugiera justificadamente con qué átomos podría realizarse un experimento que usando efecto Zeeman normal permita determinar la razón $\frac{e}{m}$ para el electrón.

\textbf{Sol:}

Se pueden usar aquellos átomos que cumplan con el efecto Zeeman anómalo o normal como el hidrógeno, zinc y sodio.



%%%5%%%



\item En el experimento de Stern-Gerlach, átomos de plata se calientan a $1000K$. Después un haz delgado de dichos átomos se pasan a través de un campo que varía en el espacio con un gradiente de $0.75 Weber/cm$ en una distancia de $10cm$, en la dirección del eje $z$. El haz después de salir del campo viaja $12cm$, antes de golpear una placa fotográfica.

\begin{enumerate}
    \item Calcule la separación de los dos componentes del haz sobre la pantalla. Use la velocidad promedio del haz de la relación $\frac{1}{2}m <v>^2 = \frac{3}{2}K_{B}T$.
    
    \textbf{Sol:}
    
    
    \item Si el experimento se realiza con los siguientes átomos: $Ca(^{1}S_0)$, $Ti(^3F_2)$, $As(^4S_{3/2})$;¿Cuántas líneas o manchas se observan en la pantalla?
    
    \textbf{Sol:}
\end{enumerate}




%%%6%%%




\item Calcule el factor de Landé, $g(L,S,J)$, y el momento magnético $\overline{\mu}_{J}$, de átomos en los estados $^{2}D_{5/2}$, $^{2}D_{3/2}$ y $^{2}F_{7/2}$.


\textbf{Sol:}

El factor de Landé es:

\begin{equation*}
    g_L = 1 + \frac{j(j+1)+s(s+1) - l(l+1)}{2j(j+1)}
\end{equation*}

y el momento magnético:

\begin{equation*}
    \overline{\mu}_{J} = \frac{e}{2 m_e} g_L \overline{J}
\end{equation*}

entonces para $^2D_{5/2}$ ($s=1/2$, $j=5/2$, $l =2$):

\begin{equation*}
    g_L =1+ \frac{(5/2)(7/2)+ (1/2)(3/2)- (2)(3)}{2(5/2)(7/2)} = \frac{6}{5}
\end{equation*}

\begin{equation*}
    \overline{\mu}_{J} = \frac{e}{2 m_e} g_L \overline{J} = \frac{e}{10 m_e} \overline{J} 
\end{equation*}

para $^2D_{3/2}$ ($s=1/2$, $j=3/2$, $l=1$):

\begin{equation*}
    g_L =1+ \frac{(3/2)(5/2)+(1/2)(3/2) - (1)(2)}{2(3/2)(5/2)} = \frac{4}{3}
\end{equation*}

\begin{equation*}
    \overline{\mu}_{J} = \frac{e}{2 m_e} g_L \overline{J} =\frac{e}{6 m_e} \overline{J}
\end{equation*}


para $^2F_{7/2}$ ($s=1/2$, $j=7/2$, $l=3$):

\begin{equation*}
    g_L =1+ \frac{(7/2)(9/2)+(1/2)(3/2)-(3)(4)}{2(7/2)(9/2)} = \frac{8}{7}
\end{equation*}

\begin{equation*}
    \overline{\mu}_{J} = \frac{e}{2 m_e} g_L \overline{J} =\frac{e}{14 m_e} g_L \overline{J}
\end{equation*}


%%%7%%%



\item Demuestre que cuando un átomo con momento angular total $\overline{J}$, se coloca en un campo magnético constante $\overline{B}=B_0 \hat{e}_z$, el momento angular precesa alrededor del campo magnético con una frecuencia $\omega = \frac{eB_0}{2m}g(L,S,J)$.

\textbf{Sol:}




%%%8%%%




\item En los siguientes 3 casos el átomo tiene una transición entre 2 estados cuánticos y se encuentra sometido a un campo magnético constante. Encontrar en cada caso, el número de lineas que se observan, en presencia de dicho campo.

\begin{enumerate}
    \item Cadmio en la transición $^{1}P-^{1}D$ cuya longitud de onda, sin campo es $\lambda=6438.47 angstroms$
    
    \textbf{Sol:}
    
    \item Doblete del sodio con longitudes de onda $\lambda_1=5889.96 angstroms$ y $\lambda_2 =5895.93 angstronms$, correspondientes a las transiciones $^{2}P_{1/2}-^{2}S_{1/2}$ y $^{2}P_{1/2}-^{2}S_{1/2}$.
    
    \textbf{Sol:}
    
    \item La línea con $\lambda=4722.16 angstroms$ del zinc correspondiente a la transición $^{3}P_1 - ^{3}S_1$.
    
    \textbf{Sol:}

\end{enumerate}




%%%9%%%




\item Determine la separación Zeeman de los niveles de energía que corresponden al término $^{4}F_{5/2}$ y $^{4}G_{7/2}$ en un campo magnético de $1000G(0.1\frac{Weber}{m^2})$, Utilice su respuesta para predecir el efecto de tal campo magnético sobre la línea de $5700 A$ del $Sc$ (escandio).

\textbf{Sol:}

El cambio de energía por el efecto Zeeman es:

\begin{equation*}
    \Delta E = g_L \mu_B m_j B
\end{equation*}

entonces

\begin{equation*}
    g_L = 1+\frac{(5/2)(7/2)+(3/2)(5/2)-(1)(2)}{2(5/2)(7/2)} = \frac{8}{5}
\end{equation*}

\begin{equation*}
    \Delta E =  \frac{8}{5} \left(9.27\times 10^{-24}\frac{J}{T}\right) \left(\frac{3}{2}\right) (0.1T) =  2.22 \times 10^{-24} J
\end{equation*}



%%%10%%%




\item ¿Cuántos estados están permitidos para tres electrones $p$ equivalentes?

\textbf{Sol:}

$3$ que son $^4S_{3/2}$, $^{2}D_{5/2}$ y $^{2}P_{3/2}$.


    
    
\end{enumerate}

\end{document}