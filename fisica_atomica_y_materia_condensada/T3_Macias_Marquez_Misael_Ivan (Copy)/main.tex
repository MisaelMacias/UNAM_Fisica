 \documentclass[12pt,a4paper]{article}

\usepackage{graphicx}% Include figure files
\usepackage{dcolumn}% Align table columns on decimal point
\usepackage{bm}% bold math
%\usepackage{hyperref}% add hypertext capabilities
%\usepackage[mathlines]{lineno}% Enable numbering of text and display math
%\linenumbers\relax % Commence numbering lines

%\usepackage[showframe,%Uncomment any one of the following lines to test 
%%scale=0.7, marginratio={1:1, 2:3}, ignoreall,% default settings
%%text={7in,10in},centering,
%%margin=1.5in,
%%total={6.5in,8.75in}, top=1.2in, left=0.9in, includefoot,
%%height=10in,a5paper,hmargin={3cm,0.8in},
%]{geometry}

\usepackage{multicol}%Para hacer varias columnas
\usepackage{multicol,caption}
\usepackage{multirow}
\usepackage{cancel}
\usepackage{hyperref}
\hypersetup{
    colorlinks=true,
    linkcolor=blue,
    filecolor=magenta,      
    urlcolor=cyan,
}

\setlength{\topmargin}{-1.0in}
\setlength{\oddsidemargin}{-0.3pc}
\setlength{\evensidemargin}{-0.3pc}
\setlength{\textwidth}{6.75in}
\setlength{\textheight}{9.5in}
\setlength{\parskip}{0.5pc}

\usepackage[utf8]{inputenc}
\usepackage{expl3,xparse,xcoffins,titling,kantlipsum}
\usepackage{graphicx}
\usepackage{xcolor} 
\usepackage{siunitx}
\usepackage{nopageno}
\usepackage{lettrine}
\usepackage{caption}
\renewcommand{\figurename}{Figura}
\usepackage{float}
\renewcommand\refname{Bibliograf\'ia}
\usepackage{amssymb}
\usepackage{amsmath}
\usepackage[rightcaption]{sidecap}
\usepackage[spanish]{babel}

\providecommand{\abs}[1]{\lvert#1\rvert}
\providecommand{\norm}[1]{\lVert#1\rVert}
\newcommand{\dbar}{\mathchar'26\mkern-12mu d}

\usepackage{mathtools}
\DeclarePairedDelimiter\bra{\langle}{\rvert}
\DeclarePairedDelimiter\ket{\lvert}{\rangle}
\DeclarePairedDelimiterX\braket[2]{\langle}{\rangle}{#1 \delimsize\vert #2}

% CABECERA Y PIE DE PÁGINA %%%%%
\usepackage{fancyhdr}
\pagestyle{fancy}
\fancyhf{}

\begin{document}

Macías Márquez Misael Iván

\begin{enumerate}




%%%1%%%




\item Demuestre que todas las componentes del momento angular orbital para una partícula satisfacen 

\begin{equation*}
    [ \hat{L}_{\alpha}, f(\overline{r})] = \hat{L}_{\alpha} f(\overline{r}) 
\end{equation*}

donde $\alpha$ toma valores del conjunto  $\{x,y,x\}$ y $f(\overline{r})$ es una función de coordenadas derivable.

\textbf{Sol:}

Sea $\beta$, $\gamma$ $\in$ $\{x,y,z\}$, con $\alpha \neq \beta \neq  \gamma$

\begin{equation*}
    [\hat{L}_\alpha, f(\overline{r})] = \hat{L}_\alpha f(\overline{r}) - f(\overline{r}) \hat{L}_\alpha
\end{equation*}

por definición

\begin{equation*}
    [\hat{L}_\alpha , f(\overline{r})] = \hat{L}_\alpha f(\overline{r}) + i \hbar f(\overline{r}) (\beta\frac{\partial }{\partial \gamma} - \gamma \frac{\partial}{\partial \beta})
\end{equation*}

y agregando un par de ceros

\begin{equation*}
    [\hat{L}_\alpha , f(\overline{r})] = \hat{L}_\alpha f(\overline{r})- f(\overline{r}) (- \i \hbar\beta\frac{\partial }{\partial \gamma} + i \hbar \gamma \frac{\partial}{\partial \beta} + (-i\hbar\gamma \frac{\partial}{\partial \beta}  + i \hbar \frac{\partial}{\partial \beta} \gamma))
\end{equation*}

\begin{equation*}
    =\hat{L}_\alpha f(\overline{r}) - f(\overline{r}) ([\hat{x}_\beta,\hat{p}_\gamma] + [\hat{x}_\gamma, \hat{p}_\beta] )
\end{equation*}

y usando $[\hat{x}_i, \hat{p}_j] = i \hbar \delta_{ij}$

\begin{equation*}
    [\hat{L}_\alpha , f(\overline{r})] = \hat{L}_\alpha f(\overline{r}) \cancel{-i \hbar f(\overline{r}) (\delta_{\beta \gamma} + \delta_{\gamma \beta} )}
\end{equation*}

\begin{equation*}
    = \hat{L}_\alpha f(\overline{r})
\end{equation*}




%%%2%%%




\item Demuestre que el operador de momento angular orbital $\overline{L} = \sum_{j=1}^{N} \overline{L}_j$ conmuta con el hamiltoniano sin interacción de $N$ partículas

\begin{equation*}
    \hat{H} = \frac{1}{2m} \sum_{j=1}^{N} \left(\frac{\hat{p}_{j}^{2}}{2m} - \frac{kZ e^2}{r_j}\right)
\end{equation*}

\textbf{Sol:}

Usando notación de índices

\begin{equation*}
    [\hat{L}, \hat{H}] = [\hat{L}_i, \hat{H}_j]
\end{equation*}

por definición del momento angular

\begin{equation*}
    [\hat{L}_i, \hat{H}_j] =\left[\epsilon_{ijk} \hat{x}_i \hat{p}_j,\frac{1}{2m} \sum_{j=1}^{N} \left(\frac{\hat{p}_{j}^{2}}{2m} - \frac{kZ e^2}{r_j}\right)\right]
\end{equation*}

como el conmutador abre sumas, saca escalares y se cumple que $[xy,z] = [x,z]y + x[y,z] $


\begin{equation*}
    [\hat{L}_i, \hat{H}_j] =\frac{\epsilon_{ijk}}{2m} \sum_{j=1}^{N}\left( \left[\hat{x}_i \hat{p}_j,\frac{\hat{p}_{j}^{2}}{2m} \right] - \left[\hat{x}_i \hat{p}_j,\frac{kZ e^2}{r_j}\right]\right)
\end{equation*}


\begin{equation*}
    =\frac{\epsilon_{ijk}}{2m} \sum_{j=1}^{N}\left(\frac{1}{4m^2} \left(\left[\hat{x}_i ,\hat{p}_{j}^{2} \right] \hat{p}_j +\hat{x}_i  \left[\hat{p}_j,\hat{p}_{j}^{2} \right] \right) -kZe^2\left( \left[\hat{x}_i ,\frac{1}{r_j}\right]\hat{p}_j +\hat{x}_i \left[ \hat{p}_j,\frac{1}{r_j}\right]\right)\right)
\end{equation*}

además como $[x,yz] = [x,y]z + y[x,z]$, $[\hat{x}_i, \hat{p}_j] = i \hbar \delta_{ij}$, $[\hat{x}_i,\hat{x_j}] =0$ , $[\hat{p}_i, \hat{p}_j] = 0$ y $[x,y^{-1}] = [x,y]^{y^{-1}}$
 

\begin{equation*}
    [\hat{L}_i, \hat{H}_j] = \sum_{j=1}^{N} \frac{\epsilon_{ijk}}{2m}\left(\frac{1}{4m^2} \left(([\hat{x}_i, \hat{p}_j] \hat{p}_j + \hat{p}_j[\hat{x_i}, \hat{p}_j]) \hat{p}_j +\hat{x}_i  ([\hat{p}_i,\hat{p}_j]\hat{p}_j + \hat{p}_j [\hat{p}_i,\hat{p}_j]) \right) \right.
\end{equation*}

\begin{equation*}
    \left. -kZe^2\left( \left[\hat{x}_i ,\frac{1}{r_j}\right]\hat{p}_j +\hat{x}_i \left[ \hat{p}_j,\frac{1}{r_j}\right]\right)\right)
\end{equation*}

\begin{equation*}
    = \sum_{j=1}^{N} \frac{\epsilon_{ijk}}{2m}\left(\frac{1}{4m^2} \left((i\hbar \delta_{ij} \hat{p}_j + \hat{p}_ji\hbar \delta_{ij}) \hat{p}_j +\hat{x}_i  (\cancel{[\hat{p}_i,\hat{p}_j]}\hat{p}_j + \hat{p}_i \cancel{[\hat{p}_j,\hat{p}_j]}) \right) \right.
\end{equation*}

\begin{equation*}
    \left. -kZe^2\left( \left[r_j ,\hat{x}_i\right]^{r_{j}^{-1}}\hat{p}_j +\hat{x}_i \left[ r_j,\hat{p}_j\right]^{r_{j}^{-1}}\right)\right)
\end{equation*}

\begin{equation*}
    =\sum_{j=1}^{N}\frac{\epsilon_{ijk}}{2m}\left( \frac{i\hbar\delta_{ij}}{2m^2}\hat{p}_{j}^{2}  -kZe^2\left( \cancel{\left[r_j ,\hat{x}_i\right]^{r_{j}^{-1}}}\hat{p}_j +\hat{x}_i i\hbar \delta_{ij}\right)\right)
\end{equation*}

\begin{equation*}
    =\sum_{j=1}^{N}\frac{i \hbar\epsilon_{ijk}\delta_{ij}}{2m}\left( \frac{1}{2m^2}\hat{p}_{j}^{2}  -kZe^2\hat{x}_i \right)
\end{equation*}

y para $j \neq i$

\begin{equation*}
    [\hat{L}, \hat{H}] = 0
\end{equation*}


%%%3%%%




\item Demuestre que el operador de momento angular orbital $\overline{L}_{jk} = \overline{L}_j + \overline{L}_k$ conmuta con el operador de interacción electrón-electrón de $N$ electrones ($j \neq k$)

\begin{equation*}
    \hat{V} = \frac{1}{2} \sum_{j,k}^{2} \frac{ke^2}{|\overline{r}_{j} - \overline{r}_k|}
\end{equation*}

\textbf{Sol:}

\begin{equation*}
    [\hat{L}_{jk},\hat{V}] =\left[\hat{L}_j + \hat{L}_k,\frac{1}{2} \sum_{j,k}^{2} \frac{ke^2}{|\overline{r}_{j} - \overline{r}_k|}\right]
\end{equation*}


\begin{equation*}
    =\frac{ke^2}{2}\left[\hat{L}_j + \hat{L}_k, \frac{1}{|\overline{r}_{1} - \overline{r}_2|} +\frac{1}{|\overline{r}_{2} - \overline{r}_1|}\right]
\end{equation*}

\begin{equation*}
    =\frac{ke^2}{2}\left[\hat{L}_{x1} + \hat{L}_{y1} + \hat{L}_{z1} + \hat{L}_{x2}+ \hat{L}_{y2} + \hat{L}_{z2}, \frac{1}{|\overline{r}_{1} - \overline{r}_2|} +\frac{1}{|\overline{r}_{2} - \overline{r}_1|}\right]
\end{equation*}

\begin{equation*}
    =\frac{ke^2}{2}\left([\hat{L}_{x1},\frac{1}{|\overline{r}_{1} - \overline{r}_2|}] + [\hat{L}_{y1},\frac{1}{|\overline{r}_{1} - \overline{r}_2|}] + [\hat{L}_{z1},\frac{1}{|\overline{r}_{1} - \overline{r}_2|}] + [\hat{L}_{x2},\frac{1}{|\overline{r}_{1} - \overline{r}_2|}]+ [\hat{L}_{y2},\frac{1}{|\overline{r}_{1} - \overline{r}_2|}]  \right.
\end{equation*}

\begin{equation*}
    + [\hat{L}_{z2},\frac{1}{|\overline{r}_{2} - \overline{r}_1|}] +[\hat{L}_{x1},\frac{1}{|\overline{r}_{2} - \overline{r}_1|}] + [\hat{L}_{y1},\frac{1}{|\overline{r}_{2} - \overline{r}_1|}] + [\hat{L}_{z1},\frac{1}{|\overline{r}_{2} - \overline{r}_1|}] + [\hat{L}_{x2},\frac{1}{|\overline{r}_{2} - \overline{r}_1|}]+ [\hat{L}_{y2},\frac{1}{|\overline{r}_{2} - \overline{r}_1|}] 
\end{equation*}

\begin{equation*}
    \left.+ [\hat{L}_{z2},\frac{1}{|\overline{r}_{2} - \overline{r}_1|}] \right)
\end{equation*}

y usando el problema 1

\begin{equation*}
    =\frac{ke^2}{2}\left(\hat{L}_{x1}\frac{1}{|\overline{r}_{1} - \overline{r}_2|} + \hat{L}_{y1}\frac{1}{|\overline{r}_{1} - \overline{r}_2|} + \hat{L}_{z1}\frac{1}{|\overline{r}_{1} - \overline{r}_2|} + \hat{L}_{x2}\frac{1}{|\overline{r}_{1} - \overline{r}_2|}+ \hat{L}_{y2}\frac{1}{|\overline{r}_{1} - \overline{r}_2|}  \right.
\end{equation*}

\begin{equation*}
    + \hat{L}_{z2}\frac{1}{|\overline{r}_{2} - \overline{r}_1|} +\hat{L}_{x1}\frac{1}{|\overline{r}_{2} - \overline{r}_1|} + \hat{L}_{y1}\frac{1}{|\overline{r}_{2} - \overline{r}_1|} + \hat{L}_{z1}\frac{1}{|\overline{r}_{2} - \overline{r}_1|} + \hat{L}_{x2}\frac{1}{|\overline{r}_{2} - \overline{r}_1|}+ \hat{L}_{y2}\frac{1}{|\overline{r}_{2} - \overline{r}_1|} 
\end{equation*}

\begin{equation*}
    \left.+ \hat{L}_{z2}\frac{1}{|\overline{r}_{2} - \overline{r}_1|} \right)
\end{equation*}

y por definición

\begin{equation*}
    =\frac{ke^2}{2}\left(\epsilon_{123}\hat{x}_{21} \hat{p}_{31} \frac{1}{|\overline{r}_{1} - \overline{r}_2|} + \epsilon_{213}\hat{x}_{12} \hat{p}_{32}\frac{1}{|\overline{r}_{1} - \overline{r}_2|} +\epsilon_{321}\hat{x}_{21} \hat{p}_{12}\frac{1}{|\overline{r}_{1} - \overline{r}_2|} + \epsilon_{123}\hat{x}_{22} \hat{p}_{32}\frac{1}{|\overline{r}_{1} - \overline{r}_2|}+ \right.
\end{equation*}

\begin{equation*}
     \epsilon_{213}\hat{x}_{12} \hat{p}_{32}\frac{1}{|\overline{r}_{1} - \overline{r}_2|} + \epsilon_{312}\hat{x}_{12} \hat{p}_{22}\frac{1}{|\overline{r}_{2} - \overline{r}_1|} +\epsilon_{123}\hat{x}_{21} \hat{p}_{31}\frac{1}{|\overline{r}_{2} - \overline{r}_1|} + \epsilon_{213}\hat{x}_{11} \hat{p}_{31}\frac{1}{|\overline{r}_{2} - \overline{r}_1|} 
\end{equation*}

\begin{equation*}
    \left.+\epsilon_{312} \hat{x}_{11} \hat{p}_{21}\frac{1}{|\overline{r}_{2} - \overline{r}_1|} +\epsilon_{123} \hat{x}_{22} \hat{p}_{32}\frac{1}{|\overline{r}_{2} - \overline{r}_1|}+\epsilon_{213}\hat{x}_{12} \hat{p}_{32}\frac{1}{|\overline{r}_{2} - \overline{r}_1|} + \epsilon_{312}\hat{x}_{12} \hat{p}_{22}\frac{1}{|\overline{r}_{2} - \overline{r}_1|} \right)
\end{equation*}






\begin{equation*}
    =\frac{ke^2}{2}\left(\cancel{y_{1} \frac{\partial}{\partial z}_1 \frac{1}{|\overline{r}_{1} - \overline{r}_2|}} -\cancel{x_{2} \frac{\partial}{\partial z}_2\frac{1}{|\overline{r}_{1} - \overline{r}_2|}} -\cancel{y_{2} \frac{\partial}{\partial x}_2\frac{1}{|\overline{r}_{1} - \overline{r}_2|}} + \cancel{y_{2} \frac{\partial}{\partial z}_2\frac{1}{|\overline{r}_{1} - \overline{r}_2|}} \right.
\end{equation*}

\begin{equation*}
     -\cancel{x_{2} \frac{\partial}{\partial z}_2\frac{1}{|\overline{r}_{1} - \overline{r}_2|}} -\cancel{x_{2} \frac{\partial}{\partial y}_2\frac{1}{|\overline{r}_{2} - \overline{r}_1|}} +\cancel{y_{1} \frac{\partial}{\partial z}_1\frac{1}{|\overline{r}_{2} - \overline{r}_1|}} -\cancel{x_{1} \frac{\partial}{\partial z}_1\frac{1}{|\overline{r}_{2} - \overline{r}_1|}} 
\end{equation*}

\begin{equation*}
    \left.+\cancel{x_{1} \frac{\partial}{\partial z}_1\frac{1}{|\overline{r}_{2} - \overline{r}_1|} }-\cancel {y_{2} \frac{\partial}{\partial z}_2\frac{1}{|\overline{r}_{2} - \overline{r}_1|}}+\cancel{x_{2} \frac{\partial}{\partial z}_2\frac{1}{|\overline{r}_{2} - \overline{r}_1|}} +\cancel{ x_{2} \frac{\partial}{\partial y}_2 \frac{1}{|\overline{r}_{2} - \overline{r}_1|}} \right) = 0
\end{equation*}









%%%%4%%%%




\item Utilizando los resultados de los problemas (2) y (3)  demuestre que el operador de momento angular orbital total para un átomo de $N$ electrones conmuta con el hamiltoniano electrostático para dicho sistema dado por ($j\neq k$)

\begin{equation*}
    \hat{H}^{(0)} = \sum_{j=1}^{N} \left(\frac{\hat{p}_{j}^{2}}{2m} - \frac{kZe^2}{r_j}\right) + \frac{1}{2} \sum_{j,k}^{N} \frac{ke^2}{|\overline{r}_j - \overline{r}_k|}
\end{equation*}

\textbf{Sol:}

\begin{equation*}
    [\hat{L}, \hat{H}^{(0)}] = [\hat{L},\hat{H} + \hat{V}] = [\hat{L},\hat{H}] + [\hat{L}, \hat{V}] 
\end{equation*}

por el problema 2 y 3  $[\hat{L}, \hat{H}] = [\hat{L}, \hat{V}] = 0$

\begin{equation*}
    \therefore [\hat{L}, \hat{H}^{(0)}] = 0
\end{equation*}



%%%%5%%%




\item Demuestre que para un átomo de $N$ electrones el operador de momento angular total $\overline{J} = \overline{L} + \overline{S}$, donde $\overline{L} = \sum_{j=1}^{N} \overline{L}_{j}$ y $\hat{S} = \sum_{j=1}^{N} \overline{S}_j$, conmuta con el hamiltoniano de interacción espín-orbital dado por

\begin{equation*}
    \hat{H}_{J,L} = \sum_{j=1}^{N} F(r_j) \overline{L}_j  \cdot \overline{S}_j
\end{equation*}

\textbf{Sol:}

\begin{equation*}
    [\hat{J}, \hat{H]_{J,L}}] = [\hat{L} + \hat{S}, \sum_{j=1}^{N} F(r_j) \hat{L}_j \cdot  \hat{S}_j]
\end{equation*}

y como $\hat{J}_{j}^{2} = L_{j}^{2}+ 2 \hat{L}_j \cdot \hat{S}_j + \hat{S}_{j}^2 $

\begin{equation*}
    [\hat{J}, \hat{H]_{J,L}}]=  [\hat{L} + \hat{S},\frac{1}{2} \sum_{j=1}^{N} F(r_j) (\hat{J}_{j}^{2} -\hat{S}_{j}^2 - \hat{L}_{j}^{2})]
\end{equation*}

\begin{equation*}
    = \frac{1}{2}\left([\hat{L},\sum_{j=1}^{N} F(r_j) (\hat{J}_{j}^{2} -\hat{S}_{j}^2 - \hat{L}_{j}^{2})] + [\hat{S},\sum_{j=1}^{N} F(r_j) (\hat{J}_{j}^{2} -\hat{S}_{j}^2 - \hat{L}_{j}^{2})]\right)
\end{equation*}

\begin{equation*}
    = \frac{1}{2}\left([\sum_{j=1}^{N}\hat{L}_j,\sum_{j=1}^{N} F(r_j) (\hat{J}_{j}^{2} -\hat{S}_{j}^2 - \hat{L}_{j}^{2})] + [\sum_{j=1}^{N}\hat{S}_j,\sum_{j=1}^{N} F(r_j) (\hat{J}_{j}^{2} -\hat{S}_{j}^2 - \hat{L}_{j}^{2})]\right)
\end{equation*}

\begin{equation*}
    = \frac{1}{2}\left(\sum_{j,i=1}^{N}[\hat{L}_i,F(r_j) (\hat{J}_{j}^{2} -\hat{S}_{j}^2 - \hat{L}_{j}^{2})] + \sum_{j,i=1}^{N}[\hat{S}_i,F(r_j) (\hat{J}_{j}^{2} -\hat{S}_{j}^2 - \hat{L}_{j}^{2})]\right)
\end{equation*}

\begin{equation*}
    = \frac{1}{2}\left(\sum_{j,i=1}^{N}[\hat{L}_i,F(r_j) \hat{J}_{j}^{2}] -[\hat{L}_i,F(r_j)\hat{S}_{j}^2] - [\hat{L}_i,F(r_j)\hat{L}_{j}^{2}] + \right.
\end{equation*}

\begin{equation*}
     \left.\sum_{j,i=1}^{N}[\hat{S}_i,F(r_j) \hat{J}_{j}^{2}] -[\hat{S}_i,F(r_j)\hat{S}_{j}^2] -[\hat{S}_i,F(r_j) \hat{L}_{j}^{2}]\right)
\end{equation*}

\begin{equation*}
    = \frac{1}{2}\left(\sum_{j,i=1}^{N}F(r_j)[\hat{L}_i, \hat{J}_{j}^{2}] -F(r_j)[\hat{L}_i,\hat{S}_{j}^2] - F(r_j)[\hat{L}_i,\hat{L}_{j}^{2}] + \right.
\end{equation*}

\begin{equation*}
     \left.\sum_{j,i=1}^{N}F(r_j)[\hat{S}_i, \hat{J}_{j}^{2}] -F(r_j)[\hat{S}_i,\hat{S}_{j}^2] -F(r_j)[\hat{S}_i, \hat{L}_{j}^{2}]\right)
\end{equation*}

y como el cuadrado del modulo del momento angular conmuta con las componentes de la misma,


\begin{equation*}
    [\hat{J}, \hat{H]_{J,L}}] = \frac{1}{2}\left(\sum_{j,i=1}^{N}F(r_j)\cancel{[\hat{L}_i, \hat{J}_{j}^{2}]} -F(r_j)\cancel{[\hat{L}_i,\hat{S}_{j}^2]}
    - F(r_j)\cancel{[\hat{L}_i,\hat{L}_{j}^{2}] }+ \right.
\end{equation*}

\begin{equation*}
     \left.\sum_{j,i=1}^{N}F(r_j)\cancel{[\hat{S}_i, \hat{J}_{j}^{2}]
     }-F(r_j)\cancel{[\hat{S}_i,\hat{S}_{j}^2]} -F(r_j)\cancel{[\hat{S}_i, \hat{L}_{j}^{2}]}\right) = 0
\end{equation*}



%%%6%%%




\item Demuestre que $\overline{J}$, conmuta con el hamiltoniano de un átomo de $N$ electrones que incluye la interacción espín-órbita.

\textbf{Sol:}

\begin{equation*}
    [\hat{J},\hat{H}]= [\hat{J}, \hat{H}^{(0)} + \hat{H}_{J.L}]
\end{equation*}

y por los problemas 4 y 5

\begin{equation*}
    [\hat{J},\hat{H}]= [\hat{J}, \hat{H}^{(0)} + \hat{H}_{J.L}] = \cancel{[\hat{J},\hat{H}^{(0)}]} + \cancel{[\hat{J},\hat{H}_{J,L}]} = 0
\end{equation*}



%%%7%%%
\newpage




\item Determine los símbolos de términos que representan a las siguientes configuraciones electrónicas:

\begin{enumerate}
    \item $np^1$
    
    \textbf{Sol:}
    
    Dado que $L_z = \sum_i l_{zi} = \sum_i m_{li} = M_L$, $S_z = \sum_i s_{zi} = \sum_i m_{si} = M_S$ y $J_z = L_z + S_z = M_L+ M_S = M_j$
    
    \begin{table}[h!]
        \centering
        \begin{tabular}{|c|c|c|c|c|}
        \hline
            $m_l$ & $m_s$ & $M_L$ & $M_S$ & $M_J$  \\
            \hline
             $1$ & $+\frac{1}{2}$ & $1$ & $+\frac{1}{2}$ &$+\frac{3}{2}$\\
             $1$ & $-\frac{1}{2}$ &$1$ & $-\frac{1}{2}$ &$+\frac{1}{2}$\\
             $0$ & $+\frac{1}{2}$ &$0$ & $+\frac{1}{2}$ &$+\frac{1}{2}$\\
             $0$ & $-\frac{1}{2}$ & $0$ & $-\frac{1}{2}$ &$-\frac{1}{2}$\\
             $-1$ & $+\frac{1}{2}$ & $-1$ & $+\frac{1}{2}$ &$-\frac{1}{2}$\\
             $-1$ & $-\frac{1}{2}$ &  $-1$ & $-\frac{1}{2}$ &$-\frac{3}{2}$\\
             \hline
        \end{tabular}
    \end{table}
    
    entonces se tienen los estados $^2P_{1/2}$ y $^2P_{3/2}$, que son los símbolos de términos que representan a la configuración $np^1$.
    

    
    \item $np^5$
    
    \textbf{Sol:}
    
    Ya que la configuración $np^5$ es como la $np^1$ pero con mas electrones en la otra orbita de $p$ por lo que tiene los mismos símbolos de términos.
    
    
    
    \item $nsnp$
    
    \textbf{Sol:}
    
    \begin{table}[h!]
        \centering
        \begin{tabular}{|c|c|c|c|c|c|c|}
        \hline
            $m_{1l}$ & $m_{1s}$ &$m_{1l}$ & $m_{1s}$ & $M_L$ & $M_S$ & $M_J$  \\
            \hline
             $0$ & $+\frac{1}{2}$ & $1$ & $+\frac{1}{2}$& $1$ &$1$ & $2$\\
             $0$ & $-\frac{1}{2}$ &$1$ & $+\frac{1}{2}$& $1$ &$0$ & $1$\\
             $0$ & $+\frac{1}{2}$ &$1$ & $-\frac{1}{2}$&$1$ &$0$ & $1$\\
             $0$ & $-\frac{1}{2}$ & $1$ & $-\frac{1}{2}$&$1$ &$-1$ & $0$\\
             $0$ & $+\frac{1}{2}$ & $0$ & $+\frac{1}{2}$& $0$ &$1$ & $1$\\
             $0$ & $-\frac{1}{2}$ &  $0$ & $+\frac{1}{2}$& $0$ &$0$ & $0$\\
             $0$ & $+\frac{1}{2}$ & $0$ & $-\frac{1}{2}$& $0$ &$0$ & $0$\\
             $0$ & $-\frac{1}{2}$ &$0$ & $-\frac{1}{2}$& $0$ &$-1$ & $-1$\\
             $0$ & $+\frac{1}{2}$ &$-1$ & $+\frac{1}{2}$& $-1$ &$1$ &  $0$\\
             $0$ & $-\frac{1}{2}$ & $-1$ & $+\frac{1}{2}$& $-1$ &$0$ & $-1$\\
             $0$ & $+\frac{1}{2}$ & $-1$ & $-\frac{1}{2}$& $-1$ &$0$ & $-1$\\
             $0$ & $-\frac{1}{2}$ &  $-1$ & $-\frac{1}{2}$& $-1$ &$-1$ & $-2$\\
             \hline
        \end{tabular}
    \end{table}
    
    los simbolos de terminos de esta configuración son $^3P_2$, $^3=_1$, $^3P_0$ y $^1P_0$
    
    \item Determine en cada caso, el término que corresponde al estado base.
    
    \textbf{Sol:}
    
    Para $np^1$ y $np^5$,  por la tercera ley de Hund su estado base es $^2P_{1/2}$ y para nsnp es $^3P_0$
    
\end{enumerate}




%%%8%%%




\item Demostrar que los símbolos de términos de las configuraciones electrónicas $np^2$ y $np^4$ coinciden.

\textbf{Sol:}

Ambas configuraciones tienen la misma cantidad de posibilidades solo que  al tener $np^2$ 2 lectrones más, que llenan las orbitas mientras que en $np^2$ sobran 2 espacio.





%%%9%%%




\item Los símbolos de términos para una configuración electrónica $nd^2$ son : $^1S$,$^1D$, $^1G$, $^3P$, $^3F$. Calcule los valores del momento angular total $J$ asociado con cada uno de ésos términos y determine cuál de ellos representa el estado base,

\textbf{Sol:}

    \begin{table}[h!]
        \centering
        \begin{tabular}{|c|c|c|c|c|}
        \hline
            símbolo de término & $L$ & $S$ & $J$ & símbolo de término completo  \\
            \hline
             $1$ & $+\frac{1}{2}$ & $1$ & $+\frac{1}{2}$ &$+\frac{3}{2}$\\
             $1$ & $-\frac{1}{2}$ &$1$ & $-\frac{1}{2}$ &$+\frac{1}{2}$\\
             $0$ & $+\frac{1}{2}$ &$0$ & $+\frac{1}{2}$ &$+\frac{1}{2}$\\
             $0$ & $-\frac{1}{2}$ & $0$ & $-\frac{1}{2}$ &$-\frac{1}{2}$\\
             $-1$ & $+\frac{1}{2}$ & $-1$ & $+\frac{1}{2}$ &$-\frac{1}{2}$\\
             $-1$ & $-\frac{1}{2}$ &  $-1$ & $-\frac{1}{2}$ &$-\frac{3}{2}$\\
             \hline
        \end{tabular}
    \end{table}
    
    por las reglas de Hund el estado base es $^3F_2$.




%%%10%%%




\item Demuestre que para funciones de estado hidrogenoides, de un electrón, el operador directo de energía potencial (de Hartree-Fock) es un operador central, esto es, actúa sólo sobre la coordenada $r$ (distancia).

\textbf{Sol:}




\end{enumerate}

\end{document}