 \documentclass[12pt,a4paper]{article}

\usepackage{graphicx}% Include figure files
\usepackage{dcolumn}% Align table columns on decimal point
\usepackage{bm}% bold math
%\usepackage{hyperref}% add hypertext capabilities
%\usepackage[mathlines]{lineno}% Enable numbering of text and display math
%\linenumbers\relax % Commence numbering lines

%\usepackage[showframe,%Uncomment any one of the following lines to test 
%%scale=0.7, marginratio={1:1, 2:3}, ignoreall,% default settings
%%text={7in,10in},centering,
%%margin=1.5in,
%%total={6.5in,8.75in}, top=1.2in, left=0.9in, includefoot,
%%height=10in,a5paper,hmargin={3cm,0.8in},
%]{geometry}

\usepackage{multicol}%Para hacer varias columnas
\usepackage{multicol,caption}
\usepackage{multirow}
\usepackage{cancel}
\usepackage{hyperref}
\hypersetup{
    colorlinks=true,
    linkcolor=blue,
    filecolor=magenta,      
    urlcolor=cyan,
}

\setlength{\topmargin}{-1.0in}
\setlength{\oddsidemargin}{-0.3pc}
\setlength{\evensidemargin}{-0.3pc}
\setlength{\textwidth}{6.75in}
\setlength{\textheight}{9.5in}
\setlength{\parskip}{0.5pc}

\usepackage[utf8]{inputenc}
\usepackage{expl3,xparse,xcoffins,titling,kantlipsum}
\usepackage{graphicx}
\usepackage{xcolor} 
\usepackage{siunitx}
\usepackage{nopageno}
\usepackage{lettrine}
\usepackage{caption}
\renewcommand{\figurename}{Figura}
\usepackage{float}
\renewcommand\refname{Bibliograf\'ia}
\usepackage{amssymb}
\usepackage{amsmath}
\usepackage[rightcaption]{sidecap}
\usepackage[spanish]{babel}

\providecommand{\abs}[1]{\lvert#1\rvert}
\providecommand{\norm}[1]{\lVert#1\rVert}
\newcommand{\dbar}{\mathchar'26\mkern-12mu d}

\usepackage{mathtools}
\DeclarePairedDelimiter\bra{\langle}{\rvert}
\DeclarePairedDelimiter\ket{\lvert}{\rangle}
\DeclarePairedDelimiterX\braket[2]{\langle}{\rangle}{#1 \delimsize\vert #2}

% CABECERA Y PIE DE PÁGINA %%%%%
\usepackage{fancyhdr}
\pagestyle{fancy}
\fancyhf{}

\begin{document}

Macías Márquez Misael Iván

\begin{enumerate}



%%%1%%%



\item El estado de un sistema cuántico está descrito por la función $\Psi(\overline{r},t)$ y su Hamiltoniano es $\hat{H}$. Sea $\hat{G}_{1}$ un operador, asociado con una cantidad física del sistema y que actúa sobre la función de estado. Demuestre que el cambio en el tiempo, del valor esperado de $\hat{G}_{1}$ en el estado $\Psi(\overline{r},t)$ es

\begin{equation*}
    \frac{d<\hat{G}_{1}>}{dt} = \left(\frac{i}{\hbar
    }\right) <[\hat{H},\hat{G}_{1}]> + \left<\frac{\partial \hat{G}_{1}}{\partial t}\right>
\end{equation*}

con 

\begin{equation*}
    <\hat{G}_{1}> = \int \Psi^{*}(\overline{r},t) \hat{G}_{1} \Psi(\overline{r},t) d\tau
\end{equation*}

donde $d\tau$ representa el elemento de volumen del sistema en el espacio de configuración y $[\hat{H},\hat{G}_{1}]$ es el conmutador de los operadores.

\textbf{Sol:}

\begin{equation*}
    \frac{d}{dt} <\hat{G_1}> = \frac{d}{dt} \int \Psi^{*}(\overline{r},t) \hat{G}_{1} \Psi(\overline{r},t) d\tau
\end{equation*}

o bien

\begin{equation*}
    = \int \frac{\partial}{\partial t}(\Psi^{*}(\overline{r},t)) \hat{G}_{1} \Psi(\overline{r},t) d\tau + \int \Psi^{*}(\overline{r},t)\frac{\partial}{\partial t}( \hat{G}_{1}) \Psi(\overline{r},t) d\tau + \int \Psi^{*}(\overline{r},t) \hat{G}_{1} \frac{\partial}{\partial t}(\Psi(\overline{r},t)) d\tau
\end{equation*}

regresando a la notación para valor esperado

\begin{equation*}
    \frac{d}{dt} <\hat{G_1}> = <\frac{\partial}{\partial t}(\Psi(\overline{r},t))|\hat{G}_1 \Psi(\overline{r},t) > + <\Psi(\overline{r},t)|\hat{G}_1\frac{\partial}{\partial t}(\Psi(\overline{r},t))> +  \left<\frac{\partial \hat{G}_1}{\partial t}\right>
\end{equation*}

y como $\hat{H}$ es el hamiltoniano de $\Psi(\overline{r},t)$,entonces aplicando la ecuación de Schrodinger

\begin{equation*}
    \frac{d}{dt} <\hat{G_1}> = <-\frac{i}{\hbar} \hat{H} \Psi(\overline{r},t) |\hat{G}_1 \Psi(\overline{r},t) > + <\Psi(\overline{r},t)|-\frac{i}{\hbar}\hat{G}_1 \hat{H} \Psi(\overline{r},t)> +  \left<\frac{\partial \hat{G}_1}{\partial t}\right>
\end{equation*}

por propiedades del valor esperado

\begin{equation*}
    \frac{d}{dt} <\hat{G_1}> = \frac{i}{\hbar}< \hat{H} \Psi(\overline{r},t) |\hat{G}_1 \Psi(\overline{r},t) >  -\frac{i}{\hbar}<\Psi(\overline{r},t)|\hat{G}_1 \hat{H}\Psi(\overline{r},t)> +  \left<\frac{\partial \hat{G}_1}{\partial t}\right>
\end{equation*}

y como el hamiltoniano es hermitiano

\begin{equation*}
    \frac{d}{dt} <\hat{G_1}> =\frac{i}{\hbar}\left[<\Psi(\overline{r},t) |\hat{H}\hat{G}_1 \Psi(\overline{r},t) >  -<\Psi(\overline{r},t)|\hat{G}_1 \hat{H}\Psi(\overline{r},t)>\right] +  \left<\frac{\partial \hat{G}_1}{\partial t}\right>
\end{equation*}

\begin{equation*}
    =\frac{i}{\hbar} <\hat{H}\hat{G}_1 - \hat{G}_1 \hat{H}> +  \left<\frac{\partial \hat{G}_1}{\partial t}\right>
\end{equation*}

\begin{equation*}
    =\frac{i}{\hbar} <[\hat{H},\hat{G}_1]> +  \left<\frac{\partial \hat{G}_1}{\partial t}\right>
\end{equation*}

$\hspace{16cm} \blacksquare$



%%%2%%%



\item Considere un conjunto de $N$ partículas idénticas, descrito por un hamiltoniano que sólo actúa sobre coordenadas espaciales, de la forma

\begin{equation*}
    \hat{H}(\overline{r}_{1},\overline{r}_{2}, ...,\overline{r}_{j},...,\overline{r}_{k}, ..., \overline{r}_{N}) = \sum_{j =1}^{N} \left(\frac{\hat{p}_{j}^{2}}{2m} + V(\overline{r}_{j})\right) + \frac{1}{2}\sum_{j,k} V(|\overline{r}_{j}-\overline{r}_{k}|)
\end{equation*}

con $j \neq k$

Sea $\hat{P}_{jk}$ es el operador que intercambia las coordenadas, espaciales y de espín, de las partículas $j$ y $k$, esto es

\begin{equation*}
    \hat{P}_{jk} \phi(x_1,x_2,...,x_j,x_k,...,x_N) = \phi(x_1,x_2,...,x_k,x_j,...,x_N)
\end{equation*}

con $x_j = (\overline{r}_{j}, S_{zj})$

Demostrar que 

\begin{enumerate}
    \item $\hat{P}_{jk}$ es hermitiano.
    
    \textbf{Sol:} este operador es hermitiano si
    
    \begin{equation*}
        \braket{\psi}{\hat{P}_{jk} \phi} = \braket{\hat{P}_{jk} \psi}{\phi}
    \end{equation*}
    
    Por definición
    
    \begin{equation*}
        \braket{\psi}{\hat{P}_{jk} \psi} =\int \psi^*(x_1,x_2,...,x_j,x_k,...,x_N) \hat{P}_{jk}\phi(x_1,x_2,...,x_j,x_k,...,x_N) d\tau
    \end{equation*}
    
    ahora aplicando el operador $\hat{P}_{jk}$
    
    \begin{equation*}
        \braket{\psi}{\hat{P}_{jk} \psi} = \int \psi^*(x_1,x_2,...,x_j,x_k,...,x_N)\phi(x_1,x_2,...,x_k,x_j,...,x_N) d\tau
    \end{equation*}
    
    
    y sustituyendo la definición del operador $\hat{P}_{jk}$ pero aplicando de nuevo el operador ($\cancel{\hat{P}_{jk} \hat{P}_{jk}}\psi^*(x_1,x_2,...,x_j,x_k,...,x_N) = \hat{P}_{jk} \psi^*(x_1,x_2,...,x_k,x_j,...,x_N) $)
    
    \begin{equation*}
       \braket{\psi}{\hat{P}_{jk} \psi}  = \int \hat{P}_{jk} \psi^{*}(x_1,x_2,...,x_k,x_j,...,x_N) \phi(x_1,x_2,...,x_k,x_j,...,x_N)d\tau 
    \end{equation*}
    
    y como se trata de partículas idénticas,el intercambio de 2 partículas no cambia físicamente el sistema por lo tanto
    
    \begin{equation*}
       \braket{\psi}{\hat{P}_{jk} \psi} = \braket{\hat{P}_{jk}\psi}{\phi}
    \end{equation*}
    
    
    
    $\hspace{15cm} \blacksquare $
    
    
    
    \item $\hat{P}_{jk}$ conmuta con el hamiltoniano.
    
    \textbf{Sol:}
    
    Por definición del conmutador y distribuyendo
    
    \begin{equation*}
        [\hat{P}_{jk}, \hat{H}] \phi(x_1,x_2,...,x_j,x_k,...,x_N) =(\hat{P}_{jk} \hat{H} - \hat{H} \hat{P}_{jk}) \phi(x_1,x_2,...,x_j,x_k,...,x_N)
    \end{equation*}
    
    \begin{equation*}
        = \hat{P}_{jk} \hat{H}  \phi(x_1,x_2,...,x_j,x_k,...,x_N) - \hat{H} \hat{P}_{jk} \phi(x_1,x_2,...,x_j,x_k,...,x_N)
    \end{equation*}
    ahora dado que $i \hbar \frac{\partial}{\partial t} \phi = \hat{H} \phi$ 
    
    \begin{equation*}
        [\hat{P}_{jk}, \hat{H}] \phi(x_1,x_2,...,x_j,x_k,...,x_N) = \hat{P}_{jk} \frac{\partial}{\partial t}  \phi(x_1,x_2,...,x_j,x_k,...,x_N)
    \end{equation*}
    
    \begin{equation*}
        - \hat{H}\phi(x_1,x_2,...,x_k,x_j,...,x_N)
    \end{equation*}
    
    \begin{equation*}
        =i \hbar \frac{\partial}{\partial t}  \phi(x_1,x_2,...,x_k,x_j,...,x_N) - i \hbar \frac{\partial}{\partial t} \phi(x_1,x_2,...,x_j,x_k,...,x_N)
    \end{equation*}
    
    \begin{equation*}
        = i \hbar \frac{\partial}{ \partial t} (\phi(x_1,x_2,...,x_k,x_j,...,x_N) - \phi(x_1,x_2,...,x_k,x_j,...,x_N) ) = 0
    \end{equation*}
    
    por lo tanto $[\hat{P}_{jk}, \hat{H}] = 0$ o bien que $\hat{P}_{jk}$ conmuta con $\hat{H}$
    
    $\hspace{15cm} \blacksquare$
    
    \item $\frac{d <\hat{P}_{jk}>}{dt} = 0$
    
    \textbf{Sol:}
    
    Por lo demostrado en el primer ejercicio,
    
    \begin{equation*}
        \frac{d <\hat{P}_{jk}>}{dt} = \frac{i}{ \hbar} <[\hat{H}, \hat{P}_{jk}]> + \left<\frac{\partial \hat{P_{jk}}}{\partial t} \right>
    \end{equation*}
    
    entonces dado que el operador de permutación no depende explícitamente de $t$ y como por el inciso anterior este conmuta con el hamiltoniano, tenemos que
    
    \begin{equation*}
        \frac{d <\hat{P}_{jk}>}{dt} = \frac{i}{ \hbar} <\cancel{[\hat{H}, \hat{P}_{jk}]}> + \left <\cancel{  \frac{\partial \hat{P_{jk}}}{\partial t}  }\right> = 0
    \end{equation*}
    
    $\hspace{15cm} \blacksquare$
    
    
    
    \item La simetría de la función de estado $\phi(x_1,x_2,...,x_j,x_k,...,x_N)$ es una constante de movimiento. 
    
    \textbf{Sol:}
    
    por el inciso pasado al tener el operador $\hat{P}_{jk}$ una derivada total nula, podemos decir que es una constante de movimiento.
    
\end{enumerate}



%%%3%%%



\item Construya una función antisimétrica para un sistema de 3 electrones en un potencial de oscilador armónico.

\textbf{Sol:}

Dado que el sistema está conformado por 3 electrones que a su vez son fermiones, la función antisimétrica descrita por el determinante de Slater es 

\begin{equation*}
    \Psi(x_1,x_2,x_3) =\frac{1}{\sqrt{3!}} \left|\begin{matrix}
    \xi_1 (x_1) & \xi_1 (x_2) & \xi_1 (x_3) \\
    \xi_2 (x_1) & \xi_2 (x_2) & \xi_2 (x_3) \\
    \xi_{3} (x_1) & \xi_{3} (x_2) & \xi_{3} (x_3) \\
    \end{matrix}\right|
\end{equation*}

\begin{equation*}
    =\frac{1}{\sqrt{3!}}\left[ \xi_1 (x_1) [\xi_2 (x_2) \xi_1 (x_1) - \xi_3 (x_2) \xi_2 (x_3)] - \xi_1 (x_2) [\xi_2(x_1)\xi_3(x_3) - \xi_2(x_3)\xi_3(x_1)] \right.
\end{equation*}

\begin{equation*}
    \left. + \xi_1 (x_3) [\xi_2 (x_1) \xi_3 (x_2) - \xi_2(x_2) \xi_3 (x_1)]\right]
\end{equation*}

y como los electrones están en un potencial de oscilador armónico se tiene que

\begin{equation*}
    \xi_n (x_i) = \frac{1}{\sqrt{2^n n!}}\left(\frac{a^2}{\pi}\right)^{1/4} e^{-a^2 x_i/2} H_n (ax)
\end{equation*}

donde $a = \sqrt{\frac{m \omega}{\hbar}}$ y $H_n$ son los polinomios de Hermite



%%%4%%%



\item Para un átomo de $N$ electrones, la función de onda tipo Determinante de Slater puede escribirse como

\begin{equation*}
    \phi(x_1,x_2,...,x_N) = \hat{A}[\xi_1 (x_1)\xi_2(x_2) ... \xi_N (x_N)] = \hat{A} \Pi_{j=1}^{N} \xi_{j} (x_j)
\end{equation*}

con el operador $\hat{A}$ está definido por

\begin{equation*}
    \hat{A} = \frac{1}{\sqrt{N!}} \sum_{R =0}^{N!-1}(-1)^{R} \hat{P}_{R}
\end{equation*}

donde $\hat{P}_{R}$ representa la permutación R-ésima del producto de espín-orbitales de una partícula que intervienen en el desarrollo del Determinante de Slater. Note que $\hat{P}_{R}$ y $\hat{A}$ actúan sólo sobre el orden de las variables $x_j$.

Demostrar que el operador $\hat{A}$ tiene las siguientes propiedades:

\begin{enumerate}
    \item $\hat{A}^2 = \sqrt{N!}\hat{A}$
    
    \textbf{Sol:}

    
    
    
    \item $\hat{A} = \hat{A}^{\dagger}$, es autoadjunto
    
    \textbf{Sol:}(pd. $\braket{\phi}{\hat{A}^{\dagger} \psi} = \braket{\hat{A}\phi}{\psi}$)
    
   \begin{equation*}
       \braket{\phi}{\hat{A}^{\dagger} \psi}= \int \phi^* \hat{A}^{\dagger} \psi d\tau =\int \phi^* \frac{1}{\sqrt{N!}} \sum_{R=0}^{N!-1} (-1)^R \hat{P}_{R}^{\dagger} \psi d\tau
   \end{equation*}
   
   
   
   como ya se demostró que $\hat{P}_R$ es hermitiano y los operadores hermitianos son equivalentes a los autoadjuntos, entonces
   
   \begin{equation*}
       \braket{\phi}{\hat{A}^{\dagger} \psi}=\int \phi^* \frac{1}{\sqrt{N!}} \sum_{R=0}^{N!-1} (-1)^R \hat{P}_{R}^{*} \psi d\tau=\int \phi^* \hat{A}^* \psi d\tau= \int (\hat{A}\phi)^* \psi d\tau = \braket{\hat{A}\phi}{\psi}
   \end{equation*}
   
   $\hspace{15cm} \blacksquare$
    
    \item $\hat{A}\hat{\Omega} = \hat{\Omega} \hat{A}$, donde $\hat{\Omega}$ es un operador simétrico ante el intercambio de coordenadas espin-espaciales de 2 partículas
    
    \textbf{Sol:}
    
    Empecemos por el lado derecho de la igualdad, dado que
    
    \begin{equation*}
        \phi(x_1,...x_j,...,x_k,...,x_N) = \hat{A} \Pi_{i=1}^{N} \xi_i (x_i)
    \end{equation*}
    
    entonces
    
    \begin{equation*}
        \hat{\Omega }\hat{A}  \phi(x_1,...x_j,...,x_k,...,x_N) =\hat{\Omega} \hat{A}  \hat{A} \Pi_{i=1}^{N} \xi_i (x_i) 
    \end{equation*}
    
    aplicando lo demostrado en a)
    
    \begin{equation*}
        \hat{\Omega }\hat{A}  \phi(x_1,...x_j,...,x_k,...,x_N) = \sqrt{N!} \hat{\Omega} \hat{A} \Pi_{i=1}^{N} \xi_{i} (x_i) = \sqrt{N!} \hat{\Omega}  \phi(x_1,...x_j,...,x_k,...,x_N)
    \end{equation*}
    
    y como $\Omega$ es simétrico ante el intercambio de coordenadas espín-espaciales
    
    \begin{equation*}
        \hat{\Omega} \hat{A} \phi(x_1,...x_j,...,x_k,...,x_N) = \sqrt{N!} \phi(x_1,...x_j,...,x_k,...,x_N)
    \end{equation*}
    
    Ahora veamos el lado  izquierdo de la igualdad
    
    \begin{equation*}
        \hat{A} \hat{\Omega} \phi(x_1,...x_j,...,x_k,...,x_N) = \hat{A} \phi(x_1,...x_j,...,x_k,...,x_N)
    \end{equation*}
    
    \begin{equation*}
        =\hat{A} \hat{A} \Pi_{i=1}^{N} \xi_{i} (x_i) = \sqrt{N!} \hat{A} \Pi_{i=1}^{N} \xi_{i} (x_i) = \sqrt{N!} \phi(x_1,...x_j,...,x_k,...,x_N)
    \end{equation*}
    
    \begin{equation*}
        \therefore \hspace{1cm} \hat{A}\hat{\Omega}\phi(x_1,...x_j,...,x_k,...,x_N) = \hat{\Omega} \hat{A}\phi(x_1,...x_j,...,x_k,...,x_N)
    \end{equation*}
    
    $\hspace{15cm} \blacksquare$
    
    
    
\end{enumerate}






%%%5%%%



\item Revisar y reproducir la deducción, completando los detalles, de las Funciones de distribución Térmico-estadística de Fermi-Dirac y Bose-Einstein de partículas sin interacción entre ellas.

\textbf{Sol:}

\textbf{Distribución de Fermi-Dirac}

Al tratarse de fermiones, un estado del sistema estará definido por el numero de partículas que se encuentren en un determinado estado energéticos con números posibles 0 y 1.$\epsilon_r$ será el estado energético r-ésimo, $n_r$ el número de partículas en el estado r-esimo y $R$ cada una de las posibles combinaciones de números de ocupación. La función de partición es:

\begin{equation*}
    Z = \sum_{l} e ^{- \beta (E_l - \mu n_l)} = \sum_{R} e^{- \beta \sum_{r}(\epsilon_r n_r - \mu n_r)} = \sum_{R} \Pi_{r}e^{-\beta (\epsilon_r n_r - \mu n_r)} 
\end{equation*}

La anterior expresión contiene todas las combinaciones posibles de $n_r$ para los valores $0$ y $1$ de forma que puede ser reescrita de la siguiente forma:

\begin{equation*}
    Z = \Pi_{r} \sum_{n_r = 0}^{1} e^{- \beta(\epsilon_r n_r - \mu n_r)} = \Pi_r 1+ e^{- \beta(\epsilon_r - \mu)}
\end{equation*}

y como

\begin{equation*}
    \phi = -k_B T \ln{Z} \hspace{1cm} \text{y} \hspace{1cm} \frac{\partial \psi}{\partial \mu} = -N
\end{equation*}

entonces

\begin{equation*}
    \phi = - k_B T \ln{Z} = -k_B T \sum_{r} \ln{(1+ e^{- \beta(\epsilon_r - \mu)})} \hspace{0.3cm} \rightarrow \hspace{0.3cm} \frac{\partial\phi}{\partial \mu} = - N = - \sum_{r} n_r = - \sum_{r} \frac{e^{-\beta(\epsilon_r - \mu)}}{1+ e^{-\beta(\epsilon_r - \mu)}}
\end{equation*}

de modo que

\begin{equation*}
    n_r = \frac{1}{e^{\beta(\epsilon_r - \mu)} + 1}
\end{equation*}

Dado que pueden existir diferentes estados cuánticos con una misma energía el número de partículas con determinada energía vendrá dado por

\begin{equation*}
    n_\epsilon = \frac{g_\epsilon}{e^{\beta(\epsilon - \mu)} + 1}
\end{equation*}

con $g_\epsilon$ la degeneración de tal energía.

\textbf{Distribución de Bose-Einstein}

Como se tienen Bosones, el sistema estará definido por el número de partículas que se encuentren en un estado energético. $\epsilon_r$ será el estado energético r-ésimo,$n_r$ el número de partículas en el estado r-esimo  y R cada una de las posibles combinaciones de números de ocupación. La función de partición es:

\begin{equation*}
    Z = \sum_l e^{- \beta(E_l - \mu n_l)} = \sum_{R} e^{-\beta \sum_{r}(\epsilon_r n_r - \mu \n_r)} = \sum_{R} \pi_{r} e^{- \beta(\epsilon_r n_r - \mu n_r)}
\end{equation*}

Lo anterior contiene todas las combinaciones posibles de $n_r$ entre 0 e $\infty$ de forma que puede ser reescrita como:

\begin{equation*}
    Z = \Pi_{r} \sum_{n_r = 0}^{\infty} e^{-\beta(\epsilon_r n_r - \mu n_r)} = \Pi_{r} \frac{1}{1 - e ^{- \beta (\epsilon_r - \mu )}}
\end{equation*}

Aplicando que :

\begin{equation*}
    \phi = -k_B T \ln{Z} \hspace{1cm} \text{y} \hspace{1cm} \frac{\partial \psi}{\partial \mu} = -N
\end{equation*}


\begin{equation*}
    \phi = k_B T \ln{Z} = -k_B T \sum_{r} \ln{(1- e^{-\beta(\epsilon_r - \mu )})}\hspace{0.3cm} \rightarrow \hspace{0.3cm} \frac{\partial \phi}{\partial \mu}= -N  = - \sum_{r} n_r = -\sum_{r} \frac{e^{-\beta(\epsilon_r - \mu)}}{1 - e^{-\beta(\epsilon_r - \mu)}}
\end{equation*}

y así

\begin{equation*}
    n_r = \frac{1}{e^{\beta(\epsilon - \mu)} - 1}
\end{equation*}

Debido a que puede existir diferentes estados cuánticos con una misma energía el número de partículas con una determinada energía vendrá dada por:

\begin{equation*}
    n_\epsilon = \frac{g_\epsilon}{e^{\beta(\epsilon - \mu)}-1}
\end{equation*}

siendo $g_E$ la degeneración de tal energía.

    
    
\end{enumerate}

\end{document}