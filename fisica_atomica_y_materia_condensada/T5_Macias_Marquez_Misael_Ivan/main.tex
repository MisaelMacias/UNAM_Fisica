 \documentclass[12pt,a4paper]{article}

\usepackage{graphicx}% Include figure files
\usepackage{dcolumn}% Align table columns on decimal point
\usepackage{bm}% bold math
%\usepackage{hyperref}% add hypertext capabilities
%\usepackage[mathlines]{lineno}% Enable numbering of text and display math
%\linenumbers\relax % Commence numbering lines

%\usepackage[showframe,%Uncomment any one of the following lines to test 
%%scale=0.7, marginratio={1:1, 2:3}, ignoreall,% default settings
%%text={7in,10in},centering,
%%margin=1.5in,
%%total={6.5in,8.75in}, top=1.2in, left=0.9in, includefoot,
%%height=10in,a5paper,hmargin={3cm,0.8in},
%]{geometry}

\usepackage{multicol}%Para hacer varias columnas
\usepackage{multicol,caption}
\usepackage{multirow}
\usepackage{cancel}
\usepackage{hyperref}
\hypersetup{
    colorlinks=true,
    linkcolor=blue,
    filecolor=magenta,      
    urlcolor=cyan,
}

\setlength{\topmargin}{-1.0in}
\setlength{\oddsidemargin}{-0.3pc}
\setlength{\evensidemargin}{-0.3pc}
\setlength{\textwidth}{6.75in}
\setlength{\textheight}{9.5in}
\setlength{\parskip}{0.5pc}

\usepackage[utf8]{inputenc}
\usepackage{expl3,xparse,xcoffins,titling,kantlipsum}
\usepackage{graphicx}
\usepackage{xcolor} 
\usepackage{siunitx}

\usepackage{nopageno}
\usepackage{lettrine}
\usepackage{caption}
\renewcommand{\figurename}{Figura}
\usepackage{float}
\renewcommand\refname{Bibliograf\'ia}
\usepackage{amssymb}
\usepackage{amsmath}
\usepackage[rightcaption]{sidecap}
\usepackage[spanish]{babel}

\providecommand{\abs}[1]{\lvert#1\rvert}
\providecommand{\norm}[1]{\lVert#1\rVert}
\newcommand{\dbar}{\mathchar'26\mkern-12mu d}

\usepackage{mathtools}
\DeclarePairedDelimiter\bra{\langle}{\rvert}
\DeclarePairedDelimiter\ket{\lvert}{\rangle}
\DeclarePairedDelimiterX\braket[2]{\langle}{\rangle}{#1 \delimsize\vert #2}

% CABECERA Y PIE DE PÁGINA %%%%%
\usepackage{fancyhdr}
\pagestyle{fancy}
\fancyhf{}
\spanishdecimal{.}

\begin{document}

Macías Márquez Misael Iván

\begin{enumerate}



%%%1%%%



\item En la molécula ion hidrógeno; supongamos que los núcleos se encuentran en las posiciones $\overline{R}_A= -\hat{e}_{z}\left(\frac{R}{2}\right)$ y $\overline{R}_B = \hat{e}_{z} \left(\frac{R}{2}\right)$, el electrón en $\overline{r}= \hat{e}_x x+ \hat{e}_y y + \hat{e}_z z$, que $\overline{a}$ es el vector de proyección de $\overline{r}$ en el plano $XY$ y tiene magnitud $a$. Utilizando las definiciones de las coordenadas elípticas confocales $\mu$, $v$, $\phi$; Calcule:

\begin{enumerate}
    \item Relaciones para $a$, $x$, $y$ y $z$ en términos de las coordenadas elípticas confocales.
    
    \textbf{Sol}:
    
    Las coordenadas elípticas confocales son:
    
    \begin{equation*}
        \xi = \frac{\overline{r}_a+\overline{r}_b}{R} \hspace{1cm} \eta = \frac{\overline{r}_a- \overline{r}_b}{R}
    \end{equation*}
    
    con $R$ la distancia internuclear y:
    
    \begin{equation*}
        r_a = \norm{\overline{r}- \overline{R}_A} = \sqrt{(x-0)^2 + (y-0)^2 + (z- (-\frac{R}{2}))^2} 
    \end{equation*}
    
    \begin{equation*}
        = \sqrt{a^2 + (z+\frac{R}{2})^2}
    \end{equation*}
    
    \begin{equation*}
        r_b = \norm{\overline{r}-\overline{R}_A} = \sqrt{(x-0)^2+(y-0)^2+(z-\frac{R}{2})^2}
    \end{equation*}
    
    \begin{equation*}
        = \sqrt{a^2 + (z- \frac{R}{2})^2}
    \end{equation*}
    
    entonces multiplicando las coordenadas confocales:
    
    \begin{equation*}
        \xi \eta = \frac{\overline{r}_a + \overline{r}_b}{R}\frac{\overline{r}_a - \overline{r}_b}{R}= \frac{\overline{r}_{a}^{2} - \overline{r}_{b}^{2}}{R^2}
    \end{equation*}
    
    \begin{equation*}
        = \frac{\cancel{a^2}+(z+\frac{R}{2})^2-\cancel{a^2} - (z- \frac{R}{2})^2}{R^2}= \frac{\cancel{z^2} +Rz+ \cancel{\frac{R^2}{4}}-\cancel{z^2} + Rz - \cancel{\frac{R^2}{4}}}{R^2} = \frac{2\cancel{R}z}{R\cancel{^2}}
    \end{equation*}
    
    \begin{equation*}
        z = \frac{\xi \eta R}{2}
    \end{equation*}
    
    \item Los factores de escala de transformación: $h_{\mu}$, $h_{v}$ y $h_{\phi}$ entre los sistemas de coordenadas.
    
    \textbf{Sol}:
    
    las reglas de transformación son:
    
    \begin{equation*}
        x = a R \sin{\mu} \cos{v} \hspace{1cm} y = b R \sin{\mu} \sin{v} \hspace{1cm} z = c R \cos{\phi}
    \end{equation*}
    
    entonces los factores de escala son:
    
    \begin{equation*}
        h_\mu = \sqrt{\left(\frac{\partial x}{\partial \mu}\right)^2 + \left(\frac{\partial y}{\partial \mu}\right)^2 + \cancel{\left(\frac{\partial z}{\partial \mu}\right)^2}}
    \end{equation*}
    
    \begin{equation*}
        = \sqrt{\left(aR \cos{v}\cos{\mu}\right)^2 + \left(bR\cos{\mu}\sin{v}\right)^{2}} = R\sqrt{\cos^2{\mu}(a^2\cos^2{v}+b^2\sin^2{v})}
    \end{equation*}
    
    \begin{equation*}
        = R\cos{\mu}\sqrt{(a\cos{v})^2+(b\sin{v})^2}
    \end{equation*}
    
    \begin{equation*}
        h_{v} = \sqrt{\left(\frac{\partial x}{\partial v}\right)^2 + \left(\frac{\partial y}{\partial v}\right)^2 + \cancel{\left(\frac{\partial z}{\partial v}\right)^2}}
    \end{equation*}
    
    \begin{equation*}
        = \sqrt{\left(-aR\sin{\mu}\sin{v}\right)^{2}+ \left(bR\sin{\mu}\cos{v}\right)^2}= R\sin{\mu}\sqrt{(a\sin{v})^2+(b\cos{v})^2}
    \end{equation*}
    
    \begin{equation*}
        h_{\phi} = \sqrt{\cancel{\left(\frac{\partial x}{\partial \phi}\right)^2} + \cancel{\left(\frac{\partial y}{\partial \phi}\right)^2} + \left(\frac{\partial z}{\partial \phi}\right)^2}
    \end{equation*}
    
    \begin{equation*}
        = \left| -cR\sin{\theta}\right|=cR|\sin{\phi}|
    \end{equation*}
    
    \item Una expresión para el volumen en términos de las coordenadas elípticas confocales.
    
    \textbf{Sol}:
    
    
    
    \item El operador Laplaciano en coordenadas elípticas confocales.
    
    \textbf{Sol}:
    
    
    \item La ecuación de Schrodinger estacionaria en coordenadas elípticas confocales.
    
    \textbf{Sol}:
    
    
    \item Demuestre que dicha ecuación es separable en ése conjunto de coordenadas.
    
    \textbf{Sol}:
    
    
    \item Demuestre que la solución para la función que depende del ángulo $\phi$, es similar a la del electrón en el átomo de Hidrógeno.
    
    \textbf{Sol}:
    
\end{enumerate}



%%%2%%%




\item En la molécula de ion hidrógeno, sobre el electrón actúan fuerzas electrostáticas debido a los dos núcleos (protones). Como consecuencia, se produce una torca en la posición $\overline{r}$, del electrón. Así se produce un cambio en el tiempo sobre el momento angular orbital del electrón, esto es

\begin{equation*}
    \left(\frac{d\overline{L}}{dt}\right)= \overline{\tau}
\end{equation*}

\begin{enumerate}
    \item Calcule la torca total, que actúa sobre el electrón, debido a los protones.
    
    \textbf{Sol}:
    
    \begin{equation*}
        \overline{L} = \nabla \overline{V} \times \overline{r}
    \end{equation*}
    
    \begin{equation*}
        = \left(\frac{\overline{r}_a}{r_{a}^3}+ \frac{\overline{r}_b}{r_{b}^3}\right) \times \overline{r}
    \end{equation*}
    
    
    \item De acuerdo a las expresiones obtenidas ¿Qué puede decir sobre la evolución temporal de las componentes de $\overline{L}$?
    
    \textbf{Sol}:
    
\end{enumerate}




%%%3%%%




\item Para la molécula ion Hidrógeno, demuestre que el conmutador $[\hat{H},\hat{L}_z]=0$; y por lo tanto $\hat{L}_z$ es constante de movimiento.

\textbf{Sol}:





%%%4%%%




\item En la molécula ion hidrógeno, demuestre que la densidad de probabilidad de encontrar al electrón en $z=0$ es:

\begin{enumerate}
    \item Distinta de cero para el estado cuántico "de enlace"
    
    \textbf{Sol}:
    
    
    \item Cero para el estado cuántico "de antienlace"
    
    \textbf{Sol}:
    
\end{enumerate}




%%%5%%%



\item En la molécula de $H_{2}^{+}$ existe una operación de simetría llamada inversión respecto a un punto de inversión (en este caso, el punto medio de la distancia entre los dos átomos) a la operación la denotaremos por $\hat{I}_{N}$ y su efecto sobre el vector de posición del electrón $\overline{r}=(x,y,z)$ es el siguiente

\begin{equation*}
    \hat{I}_N \overline{r} = - \overline{r}
\end{equation*}

Demuestre que para esta molécula se satisfacen las siguientes relaciones:

Para $H_{2}^{+}$

\begin{equation*}
    I_N \phi_{+}(\overline{r})= \phi_{r}(\overline{r})
\end{equation*}

\begin{equation*}
    I_N \phi_{-}(\overline{r})=- \phi_{-}(\overline{r})
\end{equation*}

donde $\phi_{+}= \eta_{+}= \eta_{g}$ y $\phi_{-}= \eta_{-}= \eta_{u}$ son los estados moleculares de enlace y de antienlace respectivamente para la molécula.

\textbf{Sol}:





%%%6%%%%




\item Calcule las integrales, siguientes que aparecen en el análisis de la molécula $H_{2}^{+}$. Comente sus resultados.

\begin{enumerate}
    \item $S=\braket{\xi_{1s}(\overline{r}_A)}{\norm{\hat{I}}\xi_{1s}(\overline{r}_B)}$, donde $\hat{I}$ representa el operador idéntico.
    
    \textbf{Sol}:
    
    
    \item $J=\braket{\xi_{1s}(\overline{r}_A)}{\norm{\frac{e^2}{r_B}}\xi_{1s}(\overline{r}_A)}$
    
    \textbf{Sol}:
    
    
    \item $K = \braket{\xi_{1s}(\overline{r_A})}{\norm{\frac{e^2}{r_A}}\xi_{1s}(\overline{r}_B)}$
    
    \textbf{Sol}:
    
    
\end{enumerate}




%%%7%%%



\item Muestre graficamente que la energía de estado "de enlace" tiene un mínimo y encuentre el valor para el mismo.

\textbf{Sol}:





%%%8%%%




\item  

\textbf{Sol}:
    
    
\end{enumerate}

\end{document}