 \documentclass[12pt,a4paper]{article}

\usepackage{graphicx}% Include figure files
\usepackage{dcolumn}% Align table columns on decimal point
\usepackage{bm}% bold math
%\usepackage{hyperref}% add hypertext capabilities
%\usepackage[mathlines]{lineno}% Enable numbering of text and display math
%\linenumbers\relax % Commence numbering lines

%\usepackage[showframe,%Uncomment any one of the following lines to test 
%%scale=0.7, marginratio={1:1, 2:3}, ignoreall,% default settings
%%text={7in,10in},centering,
%%margin=1.5in,
%%total={6.5in,8.75in}, top=1.2in, left=0.9in, includefoot,
%%height=10in,a5paper,hmargin={3cm,0.8in},
%]{geometry}

\usepackage{multicol}%Para hacer varias columnas
\usepackage{multicol,caption}
\usepackage{multirow}
\usepackage{cancel}
\usepackage{hyperref}
\hypersetup{
    colorlinks=true,
    linkcolor=blue,
    filecolor=magenta,      
    urlcolor=cyan,
}

\setlength{\topmargin}{-1.0in}
\setlength{\oddsidemargin}{-0.3pc}
\setlength{\evensidemargin}{-0.3pc}
\setlength{\textwidth}{6.75in}
\setlength{\textheight}{9.5in}
\setlength{\parskip}{0.5pc}

\usepackage[utf8]{inputenc}
\usepackage{expl3,xparse,xcoffins,titling,kantlipsum}
\usepackage{graphicx}
\usepackage{xcolor} 
\usepackage{siunitx}

\usepackage{nopageno}
\usepackage{lettrine}
\usepackage{caption}
\renewcommand{\figurename}{Figura}
\usepackage{float}
\renewcommand\refname{Bibliograf\'ia}
\usepackage{amssymb}
\usepackage{amsmath}
\usepackage[rightcaption]{sidecap}
\usepackage[spanish]{babel}

\providecommand{\abs}[1]{\lvert#1\rvert}
\providecommand{\norm}[1]{\lVert#1\rVert}
\newcommand{\dbar}{\mathchar'26\mkern-12mu d}

\usepackage{mathtools}
\DeclarePairedDelimiter\bra{\langle}{\rvert}
\DeclarePairedDelimiter\ket{\lvert}{\rangle}
\DeclarePairedDelimiterX\braket[2]{\langle}{\rangle}{#1 \delimsize\vert #2}

% CABECERA Y PIE DE PÁGINA %%%%%
\usepackage{fancyhdr}
\pagestyle{fancy}
\fancyhf{}
\spanishdecimal{.}

\begin{document}

Macías Márquez Misael Iván

\begin{enumerate}



%%%1%%%



\item Se utiliza luz de $550.5 \text{ nm}$ en un interferómetro de Michelson, el espejo móvil se desplaza $0.18\text{ mm}$. ¿Cuántas franjas obscuras se cuentan?

\textbf{Sol:}

Usando la ecuación 11.7, las franjas negras son:

\begin{equation*}
    N = \frac{2\Delta }{\lambda}
\end{equation*}

sustituyendo $\Delta = 0.18 \times 10^{-3} m$ y $\lambda = 550.5 \times 10^{-9} m$:

\begin{equation*}
    N = \frac{2 \cdot 0.18 \times 10^{-3} m}{550.5 \times 10^{-9} m} \approx 653
\end{equation*}



%%%2%%%




\item El espejo móvil de un interferómetro se desplaza una distancia $L$. Durante dicho desplazamiento se cuentan $250$ franjas brillantes. Si se emplea luz de $632.8 \text{ nm}$, ¿Cuál es el valor del desplazamiento $L$?

\textbf{Sol:}

Despejando $\Delta$  de la ecuación 11.7:

\begin{equation*}
    \Delta = \frac{\lambda N}{2}
\end{equation*}

sustituyendo $\Delta =L$ , $\lambda = 632.8 \times 10^{-9}m$ y $N =250$:

\begin{equation*}
    L = \frac{632.8 \times 10^{-9} \cdot 250}{2} = 7.91 \times 10^{-5} m
\end{equation*}



%%%3%%%



\item ¿Cuál es la incertidumbre para la longitud de onda en la ecuación (11.7) si la medición del desplazamiento tiene una incertidumbre $\sigma_{d}$?

\textbf{Sol:}

\begin{equation*}
    \delta \lambda = \sqrt{\left(\frac{\partial \lambda}{\partial \Delta} \sigma_d \right)^{2}}
\end{equation*}

\begin{equation*}
    = \frac{2\sigma_d }{N}
\end{equation*}

    
    
\end{enumerate}

\end{document}