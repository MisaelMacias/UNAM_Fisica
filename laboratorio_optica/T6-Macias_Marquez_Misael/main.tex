 \documentclass[12pt,a4paper]{article}

\usepackage{graphicx}% Include figure files
\usepackage{dcolumn}% Align table columns on decimal point
\usepackage{bm}% bold math
%\usepackage{hyperref}% add hypertext capabilities
%\usepackage[mathlines]{lineno}% Enable numbering of text and display math
%\linenumbers\relax % Commence numbering lines

%\usepackage[showframe,%Uncomment any one of the following lines to test 
%%scale=0.7, marginratio={1:1, 2:3}, ignoreall,% default settings
%%text={7in,10in},centering,
%%margin=1.5in,
%%total={6.5in,8.75in}, top=1.2in, left=0.9in, includefoot,
%%height=10in,a5paper,hmargin={3cm,0.8in},
%]{geometry}

\usepackage{multicol}%Para hacer varias columnas
\usepackage{multicol,caption}
\usepackage{multirow}
\usepackage{cancel}
\usepackage{hyperref}
\hypersetup{
    colorlinks=true,
    linkcolor=blue,
    filecolor=magenta,      
    urlcolor=cyan,
}

\setlength{\topmargin}{-1.0in}
\setlength{\oddsidemargin}{-0.3pc}
\setlength{\evensidemargin}{-0.3pc}
\setlength{\textwidth}{6.75in}
\setlength{\textheight}{9.5in}
\setlength{\parskip}{0.5pc}

\usepackage[utf8]{inputenc}
\usepackage{expl3,xparse,xcoffins,titling,kantlipsum}
\usepackage{graphicx}
\usepackage{xcolor} 
\usepackage{siunitx}

\usepackage{nopageno}
\usepackage{lettrine}
\usepackage{caption}
\renewcommand{\figurename}{Figura}
\usepackage{float}
\renewcommand\refname{Bibliograf\'ia}
\usepackage{amssymb}
\usepackage{amsmath}
\usepackage[rightcaption]{sidecap}
\usepackage[spanish]{babel}

\providecommand{\abs}[1]{\lvert#1\rvert}
\providecommand{\norm}[1]{\lVert#1\rVert}
\newcommand{\dbar}{\mathchar'26\mkern-12mu d}

\usepackage{mathtools}
\DeclarePairedDelimiter\bra{\langle}{\rvert}
\DeclarePairedDelimiter\ket{\lvert}{\rangle}
\DeclarePairedDelimiterX\braket[2]{\langle}{\rangle}{#1 \delimsize\vert #2}

% CABECERA Y PIE DE PÁGINA %%%%%
\usepackage{fancyhdr}
\pagestyle{fancy}
\fancyhf{}
\spanishdecimal{.}

\begin{document}

Macías Márquez Misael Iván

\begin{enumerate}



%%%1%%%



\item Los lentes en una cámara fotográfica usan refracción para formar imágenes en una película. Una lente de calidad es considerada aquella que hace todos los colores se refracten a un ángulo de desviación muy similar, es decir, que la dispersión sea lo menor posible. De los materiales en la Figura $7.1$, ¿Cuál es el mejor material para construir una lente?, ¿Por qué?

\textbf{Sol:}

Cuarzo fundido porque tiene un menor rango del índice de refracción.




%%%2%%%



\item Un prisma de vidrio con un ángulo de ápice de $60.0^{\circ}$ tiene un índice de refracción de $n=1.5$ para una  cierta longitud de onda.

\begin{enumerate}
    \item Si el ángulo de incidencia en la primer superficie es $\theta_{I1} = 48.59^{\circ}$, mostrar que la luz pasará simétricamente a través del prisma.
    
    El ángulo de desviación es (ecuación 7.3):
    
    \begin{equation*}
        \delta = \theta_{I1}+ \sin^{-1}{\left(\sin{\alpha}[n^2-\sin^{2}{\theta_{I1}}]^{1/2}-\sin{\theta_{I1}\cos{\alpha}}\right)}-\alpha 
    \end{equation*}
    
    y sustituyendo en 7.2:
    
    \begin{equation*}
         \cancel{\theta_{I1}}+ \sin^{-1}{\left(\sin{\alpha}[n^2-\sin^{2}{\theta_{I1}}]^{1/2}-\sin{\theta_{I1}\cos{\alpha}}\right)}-\cancel{\alpha}= \cancel{\theta_{I1}}+\theta_{T2}-\cancel{\alpha}
    \end{equation*}
    
    \begin{equation*}
        \theta_{T2} = \sin^{-1}{\left(\sin{\alpha}[n^2-\sin^{2}{\theta_{I1}}]^{1/2}-\sin{\theta_{I1}\cos{\alpha}}\right)}
    \end{equation*}
    
    evaluando $\theta_{I1}$, $\alpha$ y $n$: 
    
    \begin{equation*}
        \theta_{T2} = \sin^{-1}{\left(\sin{60^{\circ}}[(1.5)^2-\sin^{2}{48.59^{\circ}}]^{1/2}-\sin{48.59^{\circ}}\cos{60^{\circ}}\right)} \approx 48.59^{\circ}
    \end{equation*}
    
    \item Encontrar el ángulo de desviación ($\delta$) para $\theta_{I1}=35.6^{\circ}$ y $\theta_{I1}=51.7^{\circ}$
    
    Por la ecuación 7.3,
    
    \begin{equation*}
        \delta = \theta_{I1}+ \sin^{-1}{\left(\sin{\alpha}[n^2-\sin^{2}{\theta_{I1}}]^{1/2}-\sin{\theta_{I1}\cos{\alpha}}\right)}-\alpha 
    \end{equation*}
    
    para $\theta_{I1}=35.6^{\circ}$ se tiene:
    
    \begin{equation*}
        \delta = 35.6^{\circ}+ \sin^{-1}{\left(\sin{60^{\circ}}[(1.5)^2-\sin^{2}{35.6^{\circ}}]^{1/2}-\sin{35.6^{\circ}}\cos{60^{\circ}}\right)}-60^{\circ} \approx 40.58^{\circ} 
    \end{equation*}
    
    para $\theta_{I1}=51.7^{\circ}$ se tiene:
    
    \begin{equation*}
        \delta = 51.7^{\circ}+ \sin^{-1}{\left(\sin{60^{\circ}}[(1.5)^2-\sin^{2}{51.7^{\circ}}]^{1/2}-\sin{51.7^{\circ}}\cos{60^{\circ}}\right)}-60^{\circ} \approx 37.32^{\circ} 
    \end{equation*}
    
    
    
\end{enumerate}

    
    
\end{enumerate}

\end{document}