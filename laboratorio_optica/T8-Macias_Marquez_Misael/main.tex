 \documentclass[12pt,a4paper]{article}

\usepackage{graphicx}% Include figure files
\usepackage{dcolumn}% Align table columns on decimal point
\usepackage{bm}% bold math
%\usepackage{hyperref}% add hypertext capabilities
%\usepackage[mathlines]{lineno}% Enable numbering of text and display math
%\linenumbers\relax % Commence numbering lines

%\usepackage[showframe,%Uncomment any one of the following lines to test 
%%scale=0.7, marginratio={1:1, 2:3}, ignoreall,% default settings
%%text={7in,10in},centering,
%%margin=1.5in,
%%total={6.5in,8.75in}, top=1.2in, left=0.9in, includefoot,
%%height=10in,a5paper,hmargin={3cm,0.8in},
%]{geometry}

\usepackage{multicol}%Para hacer varias columnas
\usepackage{multicol,caption}
\usepackage{multirow}
\usepackage{cancel}
\usepackage{hyperref}
\hypersetup{
    colorlinks=true,
    linkcolor=blue,
    filecolor=magenta,      
    urlcolor=cyan,
}

\setlength{\topmargin}{-1.0in}
\setlength{\oddsidemargin}{-0.3pc}
\setlength{\evensidemargin}{-0.3pc}
\setlength{\textwidth}{6.75in}
\setlength{\textheight}{9.5in}
\setlength{\parskip}{0.5pc}

\usepackage[utf8]{inputenc}
\usepackage{expl3,xparse,xcoffins,titling,kantlipsum}
\usepackage{graphicx}
\usepackage{xcolor} 
\usepackage{siunitx}

\usepackage{nopageno}
\usepackage{lettrine}
\usepackage{caption}
\renewcommand{\figurename}{Figura}
\usepackage{float}
\renewcommand\refname{Bibliograf\'ia}
\usepackage{amssymb}
\usepackage{amsmath}
\usepackage[rightcaption]{sidecap}
\usepackage[spanish]{babel}

\providecommand{\abs}[1]{\lvert#1\rvert}
\providecommand{\norm}[1]{\lVert#1\rVert}
\newcommand{\dbar}{\mathchar'26\mkern-12mu d}

\usepackage{mathtools}
\DeclarePairedDelimiter\bra{\langle}{\rvert}
\DeclarePairedDelimiter\ket{\lvert}{\rangle}
\DeclarePairedDelimiterX\braket[2]{\langle}{\rangle}{#1 \delimsize\vert #2}

% CABECERA Y PIE DE PÁGINA %%%%%
\usepackage{fancyhdr}
\pagestyle{fancy}
\fancyhf{}
\spanishdecimal{.}

\begin{document}

Macías Márquez Misael Iván

\begin{enumerate}



%%%1%%%



\item Una lente doble cóncava tiene un índice de refracción de 1.5 y radios de curvatura de $15cm$ y $10cm$, ¿Cuáles son su longitud focal ($f$) y su potencia ($D$)?

\textbf{Sol}:

El foco de esta lente está determinada por:

\begin{equation*}
    f = \frac{n_1}{n_2 - n_1} \frac{1}{\frac{1}{R_1}- \frac{1}{R_2}}
\end{equation*}

sustituyendo $n_1 = 1$, $n_2 =1.5$, $R_1 = 15 cm$ y $R_2 = 10cm$

\begin{equation*}
    f = \frac{1}{1.5- 1} \frac{1}{\frac{1}{15 cm} - \frac{1}{10cm}}= -60cm
\end{equation*}

y la potencia por:

\begin{equation*}
    D = (n-1)\left(\frac{1}{R_1} - \frac{1}{R_2}\right)
\end{equation*}

\begin{equation*}
    = (1.5 - 1) \left(\frac{1}{0.15m}- \frac{1}{0.1m}\right)= -\frac{5}{3} D
\end{equation*}









%%%2%%%




\item Un objeto $1.2cm$ de alto está a $6cm$ de una lente doble convexa con $f=4cm$. Localizar la posición de la imagen y su tamaño. ¿Qué tipo de imagen es?

\textbf{Sol}:

Usando la ecuación del constructor de lentes:

\begin{equation*}
    \frac{1}{o} + \frac{1}{i} = \frac{1}{f}
\end{equation*}

despejando $i$ y sustituyendo:

\begin{equation*}
    i = \frac{1}{\frac{1}{f} - \frac{1}{o}} = \frac{1}{\frac{1}{4cm} - \frac{1}{6cm}} = 12cm > 0
\end{equation*}


El tamaño de la imagen es:

\begin{equation*}
    M = -\frac{i}{o} = -\frac{12}{6} = -2 < 0
\end{equation*}

por lo tanto es una imagen real e invertida.




%%%3%%%



\item Dos lentes convergentes de longitudes focales $f_1 = 10.0cm$ y $f_2 = 20.0cm$ están separadas $20.0cm$ Un objeto de $1cm$ de alto se sitúa a $15cm$ al frente (izquierda) de la primer lente. Determinar posición y tamaño de la imagen intermedia y de la final.

\textbf{Sol}:

usando la ecuación 8.8 para la primera lente:
\begin{equation*}
    \frac{1}{i_1} +  \frac{1}{o_1} = \frac{1}{f_1} \hspace{1cm} \rightarrow \hspace{1cm} i_1 = \frac{1}{\frac{1}{f_1}- \frac{1}{o_1}} = \frac{1}{\frac{1}{10cm} - \frac{1}{15cm}}= 30cm
\end{equation*}

con una magnificación de:

\begin{equation*}
    M_1= -\frac{i_1}{o_1} = -\frac{30cm}{15cm}=-2
\end{equation*}

ahora para la lente 2:

\begin{equation*}
    o_2 = 20cm-30cm=-10cm
\end{equation*}

\begin{equation*}
    i_2 = \frac{1}{\frac{1}{f_2} - \frac{1}{o_2}} = \frac{1}{\frac{1}{20cm} - \frac{1}{-10cm}} = 6.67 cm
\end{equation*}

con una magnificación de:

\begin{equation*}
    M_2 = -\frac{i_2}{o_2} = -\frac{6.67cm}{-10cm}= 0.67
\end{equation*}



%%%4%%%



\item Asumiendo que las distancias objeto ($o$) e imagen ($i$) en la ecuación 8.8 son mediciones experimentales con incertidumbres $\Delta o$ y $\Delta i$ respectivamente, calcular la propagación de incertidumbre de la distancia focal. ¿Cómo se podría linealizar esta ecuación para encontrar $f$ en una gráfica?

\textbf{Sol}:

Despejando f de la ecuación 8.8:

\begin{equation*}
    f = \frac{1}{\frac{1}{o}+\frac{1}{i}}
\end{equation*}

y propagando su incertidumbre:

\begin{equation*}
    \Delta f =\sqrt{\left(\frac{\partial f}{\partial o} \Delta o \right)^{2} + \left(\frac{\partial f}{\partial i} \Delta i \right)^{2}}
\end{equation*}

\begin{equation*}
    = \sqrt{\left(-\frac{\Delta o}{o^2(\frac{1}{o}+\frac{1}{i})^2}\right)^{2}+\left(-\frac{\Delta i}{i^2 (\frac{1}{o}+\frac{1}{i})^2}\right)^{2}}
\end{equation*}

\begin{equation*}
    = \frac{1}{i^2 o^2(\frac{1}{o}+\frac{1}{i})^2} \sqrt{(i\Delta o)^2 + (o \Delta i)^2}
\end{equation*}

Una posible linealización de la ecuación 8.8 es:

\begin{equation*}
    \frac{1}{o} = -\frac{1}{i} + \frac{1}{f}
\end{equation*}

que es de la forma $y= ax+b$ con $y= \frac{1}{o}$, $a = -1$, $x= \frac{1}{i}$ y $b = \frac{1}{b}$.




%%%5%%%



\item Cuál sería la incertidumbre de la magnificación dada por la ecuación 8.10, si se tienen incertidumbres $\Delta o$ y $\Delta i$ en las mediciones de $o$ y $i$ respectivamente.

\textbf{Sol}:

La ecuación 8.10 es:

\begin{equation*}
    M = -\frac{i}{o}
\end{equation*}

y propagando su incertidumbre:

\begin{equation*}
    \Delta M = \sqrt{\left(\frac{\partial M}{\partial i} \Delta i\right)^{2} + \left(\frac{\partial M}{\partial o} \Delta o\right)^{2}}
\end{equation*}

\begin{equation*}
    = \sqrt{\left(-\frac{\Delta i}{o}\right)^{2}+ \left(\frac{i\Delta o}{o^2}\right)^{2}}
\end{equation*}

\begin{equation*}
    \frac{1}{o}\sqrt{\Delta i^2 + \left(\frac{i\Delta o}{o}\right)^2}
\end{equation*}




%%%6%%%



\item En el método de Bessel, si conocemos las distancias $a$ y $b$ con sus respectivas incertidumbres, ¿Cuál será la incertidumbre de la distancia focal dada por la ecuación 8.11?

\textbf{Sol}:

La ecuación 8.11 es:

\begin{equation*}
    f = \frac{b^2-a^2}{4b}
\end{equation*}

y propagando su incertidumbre:

\begin{equation*}
    \Delta f = \sqrt{\left(\frac{\partial f}{\partial a} \Delta a\right)^2 + \left(\frac{\partial f}{\partial b}\Delta b\right)^2}
\end{equation*}

\begin{equation*}
    = \sqrt{\left(-\frac{2a\Delta a}{4b}\right)^{2}+\left(\frac{2b(4b)-4(b^2-a^2)}{16b^2}\Delta b\right)^{2}}
\end{equation*}

\begin{equation*}
    =\frac{1}{2b}\sqrt{\left(-a\Delta a\right)^{2}+\left(\frac{b^2+a^2}{2b}\Delta b\right)^{2}}
\end{equation*}

    
    
\end{enumerate}

\end{document}