 \documentclass[12pt,a4paper]{article}

\usepackage{graphicx}% Include figure files
\usepackage{dcolumn}% Align table columns on decimal point
\usepackage{bm}% bold math
%\usepackage{hyperref}% add hypertext capabilities
%\usepackage[mathlines]{lineno}% Enable numbering of text and display math
%\linenumbers\relax % Commence numbering lines

%\usepackage[showframe,%Uncomment any one of the following lines to test 
%%scale=0.7, marginratio={1:1, 2:3}, ignoreall,% default settings
%%text={7in,10in},centering,
%%margin=1.5in,
%%total={6.5in,8.75in}, top=1.2in, left=0.9in, includefoot,
%%height=10in,a5paper,hmargin={3cm,0.8in},
%]{geometry}

\usepackage{multicol}%Para hacer varias columnas
\usepackage{multicol,caption}
\usepackage{multirow}
\usepackage{cancel}
\usepackage{hyperref}
\hypersetup{
    colorlinks=true,
    linkcolor=blue,
    filecolor=magenta,      
    urlcolor=cyan,
}

\setlength{\topmargin}{-1.0in}
\setlength{\oddsidemargin}{-0.3pc}
\setlength{\evensidemargin}{-0.3pc}
\setlength{\textwidth}{6.75in}
\setlength{\textheight}{9.5in}
\setlength{\parskip}{0.5pc}

\usepackage[utf8]{inputenc}
\usepackage{expl3,xparse,xcoffins,titling,kantlipsum}
\usepackage{graphicx}
\usepackage{xcolor} 
\usepackage{siunitx}
\usepackage{nopageno}
\usepackage{lettrine}
\usepackage{caption}
\renewcommand{\figurename}{Figura}
\usepackage{float}
\renewcommand\refname{Bibliograf\'ia}
\usepackage{amssymb}
\usepackage{amsmath}
\usepackage[rightcaption]{sidecap}
\usepackage[spanish]{babel}

\providecommand{\abs}[1]{\lvert#1\rvert}
\providecommand{\norm}[1]{\lVert#1\rVert}
\newcommand{\dbar}{\mathchar'26\mkern-12mu d}

\usepackage{mathtools}
\DeclarePairedDelimiter\bra{\langle}{\rvert}
\DeclarePairedDelimiter\ket{\lvert}{\rangle}
\DeclarePairedDelimiterX\braket[2]{\langle}{\rangle}{#1 \delimsize\vert #2}

% CABECERA Y PIE DE PÁGINA %%%%%
\usepackage{fancyhdr}
\pagestyle{fancy}
\fancyhf{}

\begin{document}

Macías Márquez Misael Iván

\begin{enumerate}



%%%1%%%



\item ¿Cuál es la incertidumbre absoluta de la función $f=x/y$ en el caso de una sola medición, si tanto $x$ como $y$ tiene incertidumbres absolutas $\Delta x$ y $\Delta y$?

\textbf{Sol:}

Por la regla de derivación para la propagación de incertidumbres

\begin{equation*}
    \Delta f = \left(\frac{\partial f}{\partial x}\right) \Delta x + \left(\frac{\partial f}{\partial y}\right) \Delta y
\end{equation*}

\begin{equation*}
    = \frac{\Delta x}{ y} - \frac{x \Delta y}{y^2}
\end{equation*}

\begin{equation*}
    =\frac{y^2 \Delta x - yx \Delta y}{y^3} = \frac{y \Delta x - x \Delta y}{y^2}
\end{equation*}

o una estimación más realista con la adición de cuadraturas es

\begin{equation*}
    \Delta f = \sqrt{\left[\left(\frac{\partial f}{\partial x}\right) \Delta x\right]^2 + \left[\left(\frac{\partial f}{\partial y}\right) \Delta y\right]^2}
\end{equation*}

\begin{equation*}
    = \sqrt{\left[ \frac{\Delta x}{y}\right]^2 + \left[-\frac{x\Delta y}{y^2}\right]^2} = \frac{\sqrt{y^2 \Delta x^2 + x^2 \Delta y^2}}{y^2}
\end{equation*}




%%%2%%%


    
\item Se miden la altura $H$ y diámetro $D$ de un cilindro una sola vez, con incertidumbres absolutas $\Delta H$ y $\Delta D$ respectivamente. ¿Cuál es la incertidumbre absoluta del área $S$ de toda la superficie del cilindro?

\textbf{Sol:}

Recordando que el área superficial de un cilindro de diámetro $D$ y altura $H$ es

\begin{equation*}
    A = \pi D H + \frac{\pi D^2}{2}
\end{equation*}

entonces por la regla de derivación para la propagación de incertidumbres

\begin{equation*}
    \Delta A = \left(\frac{\partial A}{\partial D}\right) \Delta D + \left(\frac{\partial A}{\partial H}\right) \Delta H
\end{equation*}

\begin{equation*}
    = (\pi H + \pi D) \Delta D + (\pi H) \Delta H
\end{equation*}

o también

\begin{equation*}
    \Delta A = \sqrt{\left[\left(\frac{\partial A}{\partial D}\right) \Delta D\right]^2 + \left[\left(\frac{\partial A}{\partial H}\right) \Delta H\right]^2}
\end{equation*}

\begin{equation*}
    = \sqrt{\left[ (\pi H + \pi D) \Delta D\right]^2 + \left[\pi H \Delta H\right]^2}
\end{equation*}



%%%3%%%



\item La relación entre dos resistencias en paralelo es:

\begin{equation*}
    \frac{1}{R_{eq}} = \frac{1}{R_1} + \frac{1}{R_2}
\end{equation*}

Donde $R_{eq}$ es la resistencia equivalente del par. Si queremos encontrar una cierta resistencia equivalente de valor fijo variando las resistencias $R_1$ y $R_2$ (son variables).¿Cómo se linealiza esta ecuación?

\textbf{Sol:}

Reordenando la ecuación

\begin{equation*}
    \frac{1}{R_1} = (-1)\frac{1}{R_2} + \frac{1}{R_{eq}}
\end{equation*}

entonces así tendríamos esta ecuación linealizada de la forma $y = mx + b$ donde $y = \frac{1}{R_1}$, $m = -1$ , $x= \frac{1}{R_2}$ y $b = \frac{1}{R_{eq}}$.




%%%4%%%



\item Utilizando un péndulo se puede calcular el valor de la constante gravitacional $g$, simplemente midiendo la longitud del péndulo ($L$) y su periodo de oscilación ($T$).

\begin{equation*}
    g = 4 \pi^2 \frac{L}{T^2}
\end{equation*}

Asumiendo que $L$ y $T$ son mediciones experimentales con incertidumbres absolutas $\Delta L$ y $\Delta T$, calcular la incertidumbre nominal de la gravedad ($\sigma_{nom,g}$)

\textbf{Sol:}

De nuevo por la regla de derivación para la propagación de incertidumbres

\begin{equation*}
    \Delta g = \left(\frac{\partial g}{\partial L}\right) \Delta L + \left( \frac{\partial g}{\partial T} \right) \Delta T
\end{equation*}

\begin{equation*}
    = \frac{4 \pi^2 \Delta L}{T^2}  - \frac{8 \pi^2 L \Delta T}{T^3}
\end{equation*}

\begin{equation*}
    =\frac{4 \pi^2 T^3 \Delta L - 8 \pi^2 L T^2 \Delta T}{T^5} = \frac{4 \pi^2 T \Delta L - 8 \pi^2 L \Delta T}{T^3}
\end{equation*}

o también

\begin{equation*}
    \Delta g = \sqrt{\left[\left(\frac{\partial g}{\partial L}\right) \Delta L\right]^2 + \left[\left(\frac{\partial g}{\partial T}\right) \Delta T\right]^2}
\end{equation*}

\begin{equation*}
    = \sqrt{\left[\frac{4 \pi^2 \Delta L}{T^2}\right]^2 + \left[-\frac{8 \pi^2 L \Delta T}{T^3}\right]^2}
\end{equation*}




%%%5%%%



\item Del problema anterior, linealizar la ecuación y encontrar, a partir del valor de la pendiente (m) de la línea, el valor de la constante de gravedad $g$.

\textbf{Sol:}

\begin{equation*}
    L = \frac{g}{4 \pi^2} T^2
\end{equation*}

que queda de la forma $y = mx + b$ con $y = L$, $m = \frac{g}{4\pi^2}$, $x = T^2$ y $b = 0$, por lo tanto $g = 4 \pi^2 m$ y con incertidumbre $\Delta g = 4 \pi^2 \Delta m$
    
\end{enumerate}

\end{document}
