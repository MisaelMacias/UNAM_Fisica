 \documentclass[12pt,a4paper]{article}

\usepackage{graphicx}% Include figure files
\usepackage{dcolumn}% Align table columns on decimal point
\usepackage{bm}% bold math
%\usepackage{hyperref}% add hypertext capabilities
%\usepackage[mathlines]{lineno}% Enable numbering of text and display math
%\linenumbers\relax % Commence numbering lines

%\usepackage[showframe,%Uncomment any one of the following lines to test 
%%scale=0.7, marginratio={1:1, 2:3}, ignoreall,% default settings
%%text={7in,10in},centering,
%%margin=1.5in,
%%total={6.5in,8.75in}, top=1.2in, left=0.9in, includefoot,
%%height=10in,a5paper,hmargin={3cm,0.8in},
%]{geometry}

\usepackage{multicol}%Para hacer varias columnas
\usepackage{multicol,caption}
\usepackage{multirow}
\usepackage{cancel}
\usepackage{hyperref}
\hypersetup{
    colorlinks=true,
    linkcolor=blue,
    filecolor=magenta,      
    urlcolor=cyan,
}

\setlength{\topmargin}{-1.0in}
\setlength{\oddsidemargin}{-0.3pc}
\setlength{\evensidemargin}{-0.3pc}
\setlength{\textwidth}{6.75in}
\setlength{\textheight}{9.5in}
\setlength{\parskip}{0.5pc}

\usepackage[utf8]{inputenc}
\usepackage{expl3,xparse,xcoffins,titling,kantlipsum}
\usepackage{graphicx}
\usepackage{xcolor} 
\usepackage{siunitx}

\usepackage{nopageno}
\usepackage{lettrine}
\usepackage{caption}
\renewcommand{\figurename}{Figura}
\usepackage{float}
\renewcommand\refname{Bibliograf\'ia}
\usepackage{amssymb}
\usepackage{amsmath}
\usepackage[rightcaption]{sidecap}
\usepackage[spanish]{babel}

\providecommand{\abs}[1]{\lvert#1\rvert}
\providecommand{\norm}[1]{\lVert#1\rVert}
\newcommand{\dbar}{\mathchar'26\mkern-12mu d}

\usepackage{mathtools}
\DeclarePairedDelimiter\bra{\langle}{\rvert}
\DeclarePairedDelimiter\ket{\lvert}{\rangle}
\DeclarePairedDelimiterX\braket[2]{\langle}{\rangle}{#1 \delimsize\vert #2}

% CABECERA Y PIE DE PÁGINA %%%%%
\usepackage{fancyhdr}
\pagestyle{fancy}
\fancyhf{}
\spanishdecimal{.}

\begin{document}

Macías Márquez Misael Iván

\begin{enumerate}



%%%1%%%



\item Sea $\overline{E}$ el campo eléctrico y $\overline{B}$ el campo magnético medidos en el sistema $S$. Muestre que la cantidad $|\overline{E}|^2-|\overline{B}|^2$ es una invariante de Lorentz.

\textbf{Sol:} 



%%%2%%%



\item Sea $\overline{E}$ y $\overline{B}$ los campos eléctrico y magnético medidos en el sistema $S$. Encuentre la velocidad relativa de un sistema $S'$ con respecto a $S$ en el cual $\overline{E}$ es paralelo a $\overline{B}$.

\textbf{Sol:}



%%%3%%%




\item Dadas las leyes de transformación entre tensores $F^{\alpha' \beta'} = \Lambda_{\gamma}^{\alpha'}\Lambda_{\delta}^{\beta'}F^{\gamma \delta}$

Utilice las transformaciones de Lorentz para un sistema que se mueve con velocidad relativa $v$ en dirección $x$ para deducir las leyes de transformación del campo eléctrico y el campo magnético.

\textbf{Sol:}



%%%4%%%





\item Una partícula relativista de carga $q$ y masa $m$, se mueve bajo la acción de un campo eléctrico constante de la forma $\overline{E}=E_0 \hat{e}_x$. Considerando que en este sistema cartesiano $\overline{B}=0$, determine la trayectoria de la partícula en el espacio tiempo de Minkowski.

\textbf{Sol:}

    
    
\end{enumerate}

\end{document}