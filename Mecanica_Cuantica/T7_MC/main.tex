 \documentclass[12pt,a4paper]{article}

\usepackage{graphicx}% Include figure files
\usepackage{dcolumn}% Align table columns on decimal point
\usepackage{bm}% bold math
%\usepackage{hyperref}% add hypertext capabilities
%\usepackage[mathlines]{lineno}% Enable numbering of text and display math
%\linenumbers\relax % Commence numbering lines

%\usepackage[showframe,%Uncomment any one of the following lines to test 
%%scale=0.7, marginratio={1:1, 2:3}, ignoreall,% default settings
%%text={7in,10in},centering,
%%margin=1.5in,
%%total={6.5in,8.75in}, top=1.2in, left=0.9in, includefoot,
%%height=10in,a5paper,hmargin={3cm,0.8in},
%]{geometry}

\usepackage{multicol}%Para hacer varias columnas
\usepackage{multicol,caption}
\usepackage{multirow}
\usepackage{cancel}
\usepackage{hyperref}
\hypersetup{
    colorlinks=true,
    linkcolor=blue,
    filecolor=magenta,      
    urlcolor=cyan,
}

\setlength{\topmargin}{-1.0in}
\setlength{\oddsidemargin}{-0.3pc}
\setlength{\evensidemargin}{-0.3pc}
\setlength{\textwidth}{6.75in}
\setlength{\textheight}{9.5in}
\setlength{\parskip}{0.5pc}

\usepackage[utf8]{inputenc}
\usepackage{expl3,xparse,xcoffins,titling,kantlipsum}
\usepackage{graphicx}
\usepackage{xcolor} 
\usepackage{siunitx}
\usepackage{nopageno}
\usepackage{lettrine}
\usepackage{caption}
\renewcommand{\figurename}{Figura}
\usepackage{float}
\renewcommand\refname{Bibliograf\'ia}
\usepackage{amssymb}
\usepackage{amsmath}
\usepackage[rightcaption]{sidecap}
\usepackage[spanish]{babel}

\providecommand{\abs}[1]{\lvert#1\rvert}
\providecommand{\norm}[1]{\lVert#1\rVert}
\newcommand{\dbar}{\mathchar'26\mkern-12mu d}

% CABECERA Y PIE DE PÁGINA %%%%%
\usepackage{fancyhdr}
\pagestyle{fancy}
\fancyhf{}

\begin{document}

Macías Márquez Misael Iván

\begin{enumerate}




\item El hamiltoniano de un oscilador armónico isotrópico en tres dimensiones, de masa $m$ y frecuencia angular $\omega$, es

\begin{equation*}
    \hat{H} = - \frac{\hbar^2}{2m} \left(\frac{\partial^2}{\partial x^2} + \frac{\partial^2}{\partial^2 y} + \frac{\partial^2}{\partial z^2}\right) + \frac{m \omega^2}{2} (x^2 + y^2 + z^2)
\end{equation*}

\begin{enumerate}
    \item Utilizando separación de variables en coordenadas cartesianas, muestre que el espectro de energías vale $E = \hbar \omega (n + \frac{3}{2})$, con $n$ un entero  no negativo.
    
    \textbf{Sol:}
    
    Propongamos una solución del tipo $\psi (x,y,z) =  X(x) Y(y) Z(z)$, entonces
    
    \begin{equation*}
        E\psi (x,y,z)= E X(x) Y(y) Z(z) = \hat{H} \psi (x,y,z) 
    \end{equation*}
    
    \begin{equation*}
        = - \frac{\hbar^2}{2m} \left(Y(y) Z(z)\frac{\partial^2 X(x)}{\partial x^2} +X(x) Z(z) \frac{\partial^2 Y(y)}{\partial^2 y} +X(x)Y(y) \frac{\partial^2 Z(z)}{\partial z^2}\right) 
    \end{equation*}
    
    \begin{equation*}
        + \frac{m \omega^2}{2} (x^2 + y^2 + z^2) X(x) Y(y) Z(z)
    \end{equation*}
    
    o bien
    
    \begin{equation*}
        E = -  \left(\frac{\hbar^2}{2m X(x)}\frac{\partial^2 X(x)}{\partial x^2} +\frac{\hbar^2}{2m Y(y)} \frac{\partial^2 Y(y)}{\partial^2 y} +\frac{\hbar^2}{2m Z(z)} \frac{\partial^2 Z(z)}{\partial z^2}\right) + \frac{m \omega^2}{2} (x^2 + y^2 + z^2)
    \end{equation*}
    
    \begin{equation*}
        E =   \left(-\frac{\hbar^2}{2m X(x)}\frac{\partial^2 X(x)}{\partial x^2}+ \frac{m \omega^2}{2}  x^2\right) +\left(-\frac{\hbar^2}{2m Y(y)} \frac{\partial^2 Y(y)}{\partial^2 y}+ \frac{m \omega^2}{2} y^2\right)
    \end{equation*}
    
    \begin{equation*}
        +\left(-\frac{\hbar^2}{2m Z(z)} \frac{\partial^2 Z(z)}{\partial z^2} + \frac{m \omega^2}{2} z^2\right) 
    \end{equation*}
    
    \begin{equation*}
        E = E_x + E_y + E_z
    \end{equation*}
    
    con $(-\frac{\hbar^2}{2m} \frac{\partial^2}{\partial x_i^2} + \frac{m \omega^2}{2} x_i^2) X_i(x_i) = E_i X_i(x_i)$, y entonces tenemos que $E_{x_i}= \hbar \omega (n_{x_i} + \frac{1}{2}) $ y así
    
    \begin{equation*}
        E = \hbar\omega (n_{x} + \frac{1}{2}) + \hbar\omega (n_{y} + \frac{1}{2}) + \hbar\omega (n_{z} + \frac{1}{2})
    \end{equation*}
    
    \begin{equation*}
        = \hbar \omega (n_x + n_y + n_z + \frac{3}{2}) = \hbar \omega (n + \frac{3}{2})
    \end{equation*}
    
    con $n = n_x + n_y + n_z$
    
    
    \item Muestre que la degeneración de los primeros tres niveles de energía es 1, 3, 6 y en general la degeneración del nivel $n$ vale $\frac{1}{2}(n+1)(n+2)$.
    
    \textbf{Sol:}
    
    Sea $m = n_x + n_y$, como se tienen $n+1$ formas de obtener a $n_x + n_y = n - n_z$, entonces para cada $m$ se tiene $m + 1$ valores distintos y su degeneración es
    
    \begin{equation*}
        \sum_{m=0}^{n} m+1 = \sum_{m=1}^{n+1} m = \frac{(n+1)(n+2)}{2}
    \end{equation*}
    
    
    
\end{enumerate}



%%%3%%%



\item Considere el hamiltoniano de una partícula de masa $m$ en un potencial unidimensional de la forma

\begin{equation*}
    V(\hat{x}) = \left\{\begin{matrix}
    \infty & x< 0 \\
    \frac{m \omega^2 \hat{x}^2}{2} & x>0
    \end{matrix}\right.
\end{equation*}

con $\omega$ una constante positiva.

\begin{enumerate}
    \item ¿Qué condiciones de frontera deben cumplir las funciones propias del hamiltoniano asociado?
    
    \textbf{Sol:}
    
    Las condiciones de frontera son
    
    \begin{equation*}
        \psi(x)_{x \rightarrow \infty }  \hspace{0.5cm} \rightarrow \hspace{0.5cm} 0 \hspace{4cm} \psi(0) = 0 
    \end{equation*}
    
    
    
    
    \item ¿Qué relación hay entre las funciones propias del hamiltoniano del oscilador armónico y las funciones propias de este hamiltoniano?
    
    \textbf{Sol:}
    
    para que se cumpla $\psi(0) =0$ para toda $x$,la solución $\Psi_N (x) = 0$ para $x \leq 0$, ahora para $x \geq 0$ tenemos
    
    
    \begin{equation*}
        \left(-\frac{\hbar^2}{2m}\frac{d^2}{dx^2}+ \frac{m \omega^2 x^2}{2}\right) \psi_{n} (x) = E_n \psi_{n} (x)
    \end{equation*}
    
    pero para esta ecuación de valores propios, la solución solo es cero en $x=0$ para $n= 2N +1$ con $N$ natural, así que
    
    \begin{equation*}
        \Psi(x) = \left\{\begin{matrix}
    0 & x< 0 \\
    \psi_{2N+1} (x) & x \geq 0
    \end{matrix}\right.
    \end{equation*}
    
\end{enumerate}



%%%4%%%



\item Una partícula de masa $m$ y carga $q$ está constreñida a moverse en una dimensión. Se encuentra sujeta a una fuerza armónica , además de estar inmersa en un campo electrostático homogéneo de magnitud $\epsilon$. El hamiltoniano para este sistema es

\begin{equation*}
    \hat{H} = \frac{\hat{p}^2}{2m} + \frac{m \omega^2 \hat{x}^2}{2} - q \epsilon \hat{x}
\end{equation*}

con $\omega$ una constante positiva.

\begin{enumerate}
    \item Encuentre los valores propios y funciones propias de este hamiltoniano. Sugerencia: trate de proponer un cambio de variable que convierta este problema en el hamiltoniano  del oscilador armónico usual $\hat{H}_0$
    
    \textbf{Sol:}
    
    Sea $\hat{x} = \hat{x}' + c$, entonces $[\hat{p},\hat{x}'] = \hat{p}\hat{x}' - \hat{x}'\hat{p} = \hat{p} \hat{x} - \cancel{\hat{p}c} - \hat{x} \hat{p} + \cancel{c\hat{p}} = [\hat{p},\hat{x}] = - i\hbar$ por lo que $\hat{p} = \hat{p}'$, ahora sustituyendo
    
    \begin{equation*}
        \hat{H} = \frac{\hat{p}^2}{2m} + \frac{m \omega^2}{2}(\hat{x}' + c)^2 - q\epsilon (\hat{x}' + c)
    \end{equation*}
    
    si $c= \frac{q \epsilon}{m \omega^2}$
    
    \begin{equation*}
        \hat{H} = \frac{\hat{p}^2}{2m}+\frac{m \omega^2 \hat{x}'^2}{2} - \frac{q^2 \epsilon^2}{2m \omega^2} = \hat{H}_0 -\frac{q^2 \epsilon^2}{2m \omega^2}
    \end{equation*}
    
    entonces el problema de valores propios queda como
    
    \begin{equation*}
        \hat{H} \psi_{n} (x') =\hat{H}_0 \psi_{n}(x') = (E_n + \frac{q^2 \epsilon^2}{2m \omega^2}) \psi_{n} (x') 
    \end{equation*}
    
    donde $E_n = \hbar \omega (n + \frac{1}{2}) - \frac{q^2 \epsilon^2}{2m \omega^2}$ y $\psi_{n}(x') = \frac{(\hat{a}'^{\dagger})^n}{\sqrt{n!}} \psi_0 (x')$
    
    
    \item Si la partícula se encuentra al tiempo $t=0$ en el estado base del hamiltoniano $\hat{H}_0$, calcule la probabilidad de encontrar a la partícula en el estado base del hamiltoniano $\hat{H}$ al tiempo $t>0$. Sugerencia : utilice el postulado de desarrollo para expresar el estado base de $\hat{H}_0$ en términos de las funciones propias de $\hat{H}$.
    
    \textbf{Sol:}
    
    Dadas las condiciones la probabilidad es
    
    \begin{equation*}
        |<\psi_{0}(x-c),\psi(x,t)>|^2
    \end{equation*}
    
    y por el teorema de desarrollo
    
    \begin{equation*}
        \psi(x,t) = \sum_{n=0}^{\infty} e^{-\frac{i E_n t}{\hbar}} C_n \psi_{n} (x-c9
    \end{equation*}
    
    con $C_n = <\psi_{n}(x-c), \psi_{0} (x)>$, entonces
    
    \begin{equation*}
        |<\psi_{0}(x-c),\psi(x,t)>|^2 = \left|\sum_{n=0}^{\infty} e^{- \frac{iE_nt}{\hbar}}C_n<\psi_{0}(x-c), \psi_{n}(x-c)>\right|^2
    \end{equation*}
    
    \begin{equation*}
        = \left| \sum_{n=0}^{\infty} e^{- \frac{iE_nt}{\hbar}}C_n \delta_{0n} \right|^2 = \left|  e^{- \frac{iE_0t}{\hbar}}C_0 \right|^2= |C_0|^2
    \end{equation*}
    
    \begin{equation*}
        = |<\psi_{0}(x-c),\psi_{0}(x)>|^2 = \left|\sqrt{\frac{m \omega}{\pi \hbar}} \int_{-\infty}^{\infty} dx e^{- \frac{m\omega}{\hbar}(x^2 -cx+c^2/4 )} e^{-\frac{m\omega c^2}{4 \hbar} } \right|^2
    \end{equation*}
    
    \begin{equation*}
        = \frac{m \omega}{\pi \hbar} e^{- \frac{m \omega c^2}{2 \hbar}} \left|\int_{-\infty}^{\infty} dx e^{-\frac{m \omega}{\hbar} (x-c/2)^2}\right|^2 = e^{- \frac{m \omega c^2}{2 \hbar}} 
    \end{equation*}
    
    
\end{enumerate}



%%%5%%%



\item Considere los operadores de ascenso $\hat{a}_{1}^{\dagger}$,$\hat{a}_{2}^{\dagger}$ y descenso $\hat{a}_1$, $\hat{a}_2$ de dos osciladores armónicos independientes, es decir

\begin{equation*}
    [\hat{a}_i,\hat{a}_j] = [\hat{a}_{i}^{\dagger},\hat{a}_{j}^{\dagger}] = 0, \hspace{2cm} [\hat{a}_i , \hat{a}_{j}^{\dagger}] = \delta_{ij}
\end{equation*}

Si a partir de ellos se definen los operadores 

\begin{equation*}
    \hat{J}_{+}= \hbar \hat{a}_{1}^{\dagger} \hat{a}_{2},\hspace{1cm} \hat{J}_{-} = \hbar \hat{a}_{2}^{\dagger}\hat{a}_{1}, \hspace{1cm} \hat{J}_{z} = \frac{\hbar}{2} (\hat{a}_{1}^{\dagger}\hat{a}_1 - \hat{a}_{2}^{\dagger}\hat{a}_2) \hspace{1cm} \hat{N} = \hat{a}_{1}^{\dagger}\hat{a}_1 + \hat{a}_{2}^{\dagger}\hat{a}_2
\end{equation*}

pruebe que 

\begin{enumerate}
    \item $[\hat{J}_{z},\hat{J}_{\pm}] = \pm \hbar \hat{J}_{\pm}$
    
    \textbf{Sol:}
    
    \begin{equation*}
        [\hat{J}_{z},\hat{J}_{+}] = [\frac{\hbar}{2} (\hat{a}_{1}^{\dagger} \hat{a}_{1} - \hat{a}_{2}^{\dagger} \hat{a}_{2}, \hbar \hat{a}_{1}^{\dagger} \hat{a}_2 ] = \frac{\hbar^2}{2}[\hat{a}_{1}^{\dagger}\hat{a}_{1}, \hat{a}_{1}^{\dagger}\hat{a}_{2}] - \frac{\hbar^2}{2}[\hat{a}_{2}^{\dagger}\hat{a}_{2},\hat{a}_{1}^{\dagger}\hat{a}_{2}]
    \end{equation*}
    
    \begin{equation*}
        = \frac{\hbar^2 \hat{a}_2}{2} [\hat{a}_{1}^{\dagger} \hat{a}_{1}, \hat{a}_{1}^{\dagger}] - \frac{\hbar^2 \hat{a}_{1}^{\dagger}}{2} [\hat{a}_{2}^{\dagger}\hat{a}_{2},\hat{a}_{2}] = \frac{\hbar^2 \hat{a}_2}{2}(\hat{a}_{1}^{\dagger}[\hat{a}_1,\hat{a}_{1}^{\dagger}] + \cancel{[\hat{a}_{1}^{\dagger}, \hat{a}_{1}^{\dagger}]}\hat{a}_1) - \frac{\hbar^2 \hat{a}_{1}^{\dagger}}{2}(\hat{a}_{2}^{\dagger}\cancel{[\hat{a}_2,\hat{a}_2]}+[\hat{a}_{2}^{\dagger}, \hat{a}_2] \hat{a}_2)
    \end{equation*}
    
    \begin{equation*}
        = \frac{\hbar^2}{2} \hat{a}_2 \hat{a}_{1}^{\dagger} + \frac{\hbar^2}{2} \hat{a}_{1}^{\dagger} \hat{a}_2 = \hbar^2 \hat{a}_{1}^{\dagger} \hat{a}_2 = \hbar \hat{J}_{+}
    \end{equation*}
    
    \begin{equation*}
        [\hat{J}_z , \hat{J}_{-}] = [\frac{\hbar}{2}(\hat{a}_{1}^{\dagger}\hat{a}_1 - \hat{a}_{2}^{\dagger}\hat{a}_{2}), \hbar \hat{a}_{2}^{\dagger} \hat{a}_{1} ] = \frac{\hbar^2}{2}[\hat{a}_{1}^{\dagger}\hat{a}_{1}, \hat{a}_{2}^{\dagger}\hat{a}_{1}] - \frac{\hbar^2}{2}[\hat{a}_{2}^{\dagger}\hat{a}_{2},\hat{a}_{2}^{\dagger}\hat{a}_{1}]
    \end{equation*}
    
    \begin{equation*}
        = -\left(\frac{\hbar^2}{2} \hat{a}_{2}^{\dagger} \hat{a}_{1} + \frac{\hbar^2}{2}\hat{a}_{1}\hat{a}_{2}^{\dagger}\right) = -\hbar^2 \hat{a}_{2}^{2} \hat{a}_{1} = - \hbar \hat{J}_{-}
    \end{equation*}
    
    
    \item Si $\mathbf{\hat{J}}^2 = \hat{J}_{+}\hat{J}_{-} - \hbar \hat{J}_{z} + \hat{J}_{z}^2$, entonces $\mathbf{\hat{J}}^2 = \frac{\hbar^2}{2}\hat{N}(\frac{\hat{N}}{2} + 1)$
    
    
    \textbf{Sol:}
    
    \begin{equation*}
        \mathbf{\hat{J}}^2 = \hat{J}_{+}\hat{J}_{-} - \hbar \hat{J}_{z} + \hat{J}_{z}^2 = (\hbar \hat{a}_{1}^{\dagger} \hat{a}_{2})(\hbar \hat{a}_{2}^{\dagger} \hat{a}_{1}) - \hbar \left(\frac{\hbar}{2}(\hat{a}_{1}^{\dagger} \hat{a}_{1} - \hat{a}_{2}^{\dagger} \hat{a}_{2})\right) + \frac{\hbar^2}{4}(\hat{a}_{1}^{\dagger}\hat{a}_{1} - \hat{a}_{2}^{\dagger} \hat{a}_{2})^2
    \end{equation*}
    
    \begin{equation*}
        = \hbar^2 \hat{a}_{1}^{\dagger}\hat{a}_{1} \hat{a}_{2} \hat{a}_{2}^{\dagger} - \frac{\hbar^2}{2} (\hat{a}_{1}^{\dagger} a_{1} - \hat{a}_{1}^{\dagger} - \hat{a}_{2}^{\dagger} \hat{a}_{2}) + \frac{\hbar^2}{4} (\hat{a}_{1}^{\dagger}\hat{a}_{1}\hat{a}_{1}^{\dagger} \hat{a}_{1} + \hat{a}_{2}^{\dagger}\hat{a}_{2}\hat{a}_{2}^{\dagger} \hat{a}_{2} - \hat{a}_{2}^{\dagger}\hat{a}_{2}\hat{a}_{1}^{\dagger} \hat{a}_{1} - \hat{a}_{2}^{\dagger}\hat{a}_{2}\hat{a}_{1}^{\dagger} \hat{a}_{1} )
    \end{equation*}
    
    \begin{equation*}
        = \hbar^2 \hat{a}_{1}^{\dagger} \hat{a}_{1} (1 + \hat{a}_{2}^{\dagger} \hat{a}_{2}) - \frac{\hbar^2}{2}(\hat{a}_{1}^{\dagger}\hat{a}_{1} - \hat{a}_{2}^{\dagger}\hat{a}_{2}) + \frac{\hbar^2}{4} (\hat{a}_{1}^{\dagger}\hat{a}_{1}\hat{a}_{1}^{\dagger} \hat{a}_{1} + \hat{a}_{2}^{\dagger}\hat{a}_{2}\hat{a}_{2}^{\dagger} \hat{a}_{2} - 2 \hat{a}_{2}^{\dagger}\hat{a}_{2}\hat{a}_{1}^{\dagger} \hat{a}_{1} )
    \end{equation*}
    
    \begin{equation*}
        = \frac{\hbar^2}{2}\hat{N} + \frac{\hbar^2}{4}\hat{N}^2 = \frac{\hbar^2}{2}\hat{N}\left(\frac{\hat{N}}{2} + 1\right)
    \end{equation*}
    
    
    
    
    \item $[\mathbf{\hat{J}}^2 , \hat{J}_{z}] = 0$
    
    \textbf{Sol:}
    
    \begin{equation*}
        [\mathbf{\hat{J}}^2 , \hat{J}_{z}] = [\hat{J}_{+}\hat{J}_{-}, \hat{J}_{z}] - \hbar \cancel{[\hat{J}_{z}, \hat{J}_{z}]} + \hat{J}_{z}\cancel{[\hat{J}_{z}, \hat{J_{z}}]} + \cancel{[\hat{J}_{z}, \hat{J}_{z}} \hat{J}_{z} 
    \end{equation*}
    
    \begin{equation*}
        = \hat{J}_{+}[\hat{J}_{-}, \hat{J}_{z}] + [\hat{J}_{+}, \hat{J}_{z}] \hat{J}_{-} = \pm \hbar \hat{J}_{+} \hat{J}_{-} = 0
    \end{equation*}
    
    
    
\end{enumerate}



%%%6%%%



\item Llamaremos estados coherentes $\psi_{a}$ a los estados propios, normalizados, del operador de descenso $\hat{a}$, es decir

\begin{equation*}
    \hat{a} \psi_{\alpha} = \alpha \psi_{\alpha}
\end{equation*}

Utilizando la completés de los estados propios del operador $\hat{N} = \hat{a}^{\dagger}\hat{a}$, demuestre que los estados $\psi_{\alpha}$ se pueden escribir como

\begin{equation*}
    \psi_a = e^{-|\alpha|^{2}/2} \sum_{n=0}^{\infty} \frac{\alpha^n}{\sqrt{n!}}\phi_{n} = e^{-|\alpha|^{2}/2} e^{\alpha \hat{a}^{\dagger}} \phi_0
\end{equation*}

donde $\phi_n$ es el estado propio del operador $\hat{N}$ con valor propio n

\textbf{Sol:}

Como las funciones propias de $\hat{N}$ forman una base del espacio de estados, y así

\begin{equation*}
    \psi_\alpha = \sum_{n = 0}^{\infty} C_n  \psi_n
\end{equation*}

con $C_n = <\psi_n , \psi_\alpha>$, $\psi_{n} = \frac{(\hat{a}^{\dagger})^{n}}{\sqrt{n!}} \psi_0$, que sustituyendo

\begin{equation*}
    C_n = <\frac{(\hat{a}^{\dagger})^{n}}{\sqrt{n!}} \psi_0,\psi_\alpha> = \frac{1}{\sqrt{n!}}<\psi_0, \hat{a}^{n} \psi_{\alpha}> = \frac{1}{\sqrt{n!}} <\psi_0, \alpha^n \psi_{\alpha}>
\end{equation*}

\begin{equation*}
    = \frac{\alpha^n}{\sqrt{n!}} <\psi_0, \psi_\alpha>= \frac{\alpha^n}{\sqrt{n!}} C
\end{equation*}

entonces

\begin{equation*}
    \psi_{\alpha} = \sum_{n=0}^{\infty} C_n \psi_n = C \sum_{n=0}^{\infty} \frac{\alpha^n}{\sqrt{n!}} \psi_n
\end{equation*}

ahora veamos a la constante $C$

\begin{equation*}
    <\psi_{\alpha}, \psi_{\alpha}> = <C\sum_{n=0}^{\infty} \frac{\alpha^n}{\sqrt{n!} }\psi_{n},C\sum_{m=0}^{\infty} \frac{\alpha^m}{\sqrt{m!} }\psi_{m}>
\end{equation*}

\begin{equation*}
    = |C|^2 \sum_{n=0}^{\infty} \sum_{m=0}^{\infty} \frac{(\alpha^{n})^{*}}{\sqrt{n!}} \frac{\alpha^{m}}{\sqrt{m!}} <\psi_{n}, \psi_{m}> = |C|^2 \sum_{n=0}^{\infty} \sum_{m=0}^{\infty}\frac{(\alpha^{n})^{*}}{\sqrt{n!}} \frac{\alpha^{m}}{\sqrt{m!}} \delta_{nm}
\end{equation*}

\begin{equation*}
    = |C|^2 \sum_{n=0}^{\infty} \frac{(|a|^2)^{n}}{n!} = |c|^2 e^{|\alpha|^2} = 1
\end{equation*}

o bien

\begin{equation*}
    |C|^2 = e^{- |\alpha|^2} \hspace{1cm} \rightarrow \hspace{1cm}  C= e^{- \frac{|\alpha|^2}{2}}
\end{equation*}

y así

\begin{equation*}
    \psi_{\alpha} = e^{-\frac{|\alpha|^2}{2}} \sum_{n=0}^{\infty} \frac{\alpha^n}{\sqrt{n!}} \psi_n = e^{-\frac{|\alpha|^2}{2}} \sum_{n=0}^{\infty} \frac{(\alpha\hat{a}^{\dagger})^n}{n!} \psi_0 = e^{-\frac{|\alpha|^2}{2}} e^{\alpha \hat{a}^{\dagger}} \psi_0
\end{equation*}
    
    
\end{enumerate}

\end{document}