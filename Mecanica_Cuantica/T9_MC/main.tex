 \documentclass[12pt,a4paper]{article}

\usepackage{graphicx}% Include figure files
\usepackage{dcolumn}% Align table columns on decimal point
\usepackage{bm}% bold math
%\usepackage{hyperref}% add hypertext capabilities
%\usepackage[mathlines]{lineno}% Enable numbering of text and display math
%\linenumbers\relax % Commence numbering lines

%\usepackage[showframe,%Uncomment any one of the following lines to test 
%%scale=0.7, marginratio={1:1, 2:3}, ignoreall,% default settings
%%text={7in,10in},centering,
%%margin=1.5in,
%%total={6.5in,8.75in}, top=1.2in, left=0.9in, includefoot,
%%height=10in,a5paper,hmargin={3cm,0.8in},
%]{geometry}

\usepackage{multicol}%Para hacer varias columnas
\usepackage{multicol,caption}
\usepackage{multirow}
\usepackage{cancel}
\usepackage{hyperref}
\hypersetup{
    colorlinks=true,
    linkcolor=blue,
    filecolor=magenta,      
    urlcolor=cyan,
}

\setlength{\topmargin}{-1.0in}
\setlength{\oddsidemargin}{-0.3pc}
\setlength{\evensidemargin}{-0.3pc}
\setlength{\textwidth}{6.75in}
\setlength{\textheight}{9.5in}
\setlength{\parskip}{0.5pc}

\usepackage[utf8]{inputenc}
\usepackage{expl3,xparse,xcoffins,titling,kantlipsum}
\usepackage{graphicx}
\usepackage{xcolor} 
\usepackage{siunitx}
\usepackage{nopageno}
\usepackage{lettrine}
\usepackage{caption}
\renewcommand{\figurename}{Figura}
\usepackage{float}
\renewcommand\refname{Bibliograf\'ia}
\usepackage{amssymb}
\usepackage{amsmath}
\usepackage[rightcaption]{sidecap}
\usepackage[spanish]{babel}

\providecommand{\abs}[1]{\lvert#1\rvert}
\providecommand{\norm}[1]{\lVert#1\rVert}
\newcommand{\dbar}{\mathchar'26\mkern-12mu d}

\usepackage{mathtools}
\DeclarePairedDelimiter\bra{\langle}{\rvert}
\DeclarePairedDelimiter\ket{\lvert}{\rangle}
\DeclarePairedDelimiterX\braket[2]{\langle}{\rangle}{#1 \delimsize\vert #2}

% CABECERA Y PIE DE PÁGINA %%%%%
\usepackage{fancyhdr}
\pagestyle{fancy}
\fancyhf{}

\begin{document}

Macías Márquez Misael Iván

\begin{enumerate}



%%%1%%%



\item Suponga que un sistema compuesto por dos subsistemas no interactuantes está descrito por un hamiltoniano de la forma $\hat{H} = \hat{H}_{1} \times I + I \times \hat{H}_2$, donde $I$ es el operador identidad en cada uno de los subsistemas, los hamiltonianos $\hat{H}_{1}$ y $\hat{H_{2}}$ no dependen explícitamente del tiempo y describen por separado a cada uno de estos subsistemas. Muestre que el operador de evolución para el sistema compuesto tiene la forma

\begin{equation*}
    \hat{U}(t,t_0) = \hat{U}_{1} (t,t_0) \times \hat{U}_{2} (t,t_0)
\end{equation*}

donde $\hat{U}_{1}(t,t_0)= e^{-i  \hat{H}_1 (t-t_0) \hbar}$ y $\hat{U}_{2} (t,t_0) = e^{-i \hat{H}_{2}(t-t_0)/ \hbar}$. Muestre explícitamente que este operador satisface todas las propiedades de un operador de evolución para la ecuación de Schrodinger respectiva.

\textbf{Sol:}


Tenemos que $\ket{\psi_{j}(t)} = e^{-\frac{i \hat{H}_{j}(t-t_0)}{\hbar}} \ket{\psi_{i}(t_0)}$, entonces

\begin{equation*}
    \ket{\Psi (t)} = e^{-\frac{i \hat{H}_1 (t-t_0)}{\hbar}} \ket{\psi_1 (t_0)}  \times e^{-\frac{i \hat{H}_{2} (t-t_0)}{\hbar}} \ket{\psi_2 (t_0)}
\end{equation*}

\begin{equation*}
    = e^{- \frac{i \hat{H}_1 (t-t_0)}{\hbar}} \times e^{- \frac{i \hat{H}_2 (t-t_0)}{\hbar}} \ket{\psi_{1}(t_0)} \times \ket{\psi_2 (t_0)} = \hat{U}(t,t_0) \ket{\psi_1 \psi_2}_{t_0}
\end{equation*}

y así

\begin{equation*}
    i \hbar \frac{\partial}{\partial t} \ket{\Psi (t)} = i \hbar \frac{\partial}{\partial t} \hat{U}(t,t_0) \ket{\psi_1 \psi_2}_{t_0}
\end{equation*}

\begin{equation*}
    = i \hbar \left[\frac{\partial}{\partial t} (\hat{U}_1 (t,t_0)) \hat{U}_2 (t,t_0) + \hat{U}_{1}(t,t_0) \frac{\partial}{\partial t} \hat{U}_2(t-t_0)\right]\ket{\psi_1 \psi_2}_{t_0}
\end{equation*}

\begin{equation*}
    = i \hbar\left[-\frac{i \hat{H}_1}{\hbar} \times 1 \hat{U}_1 (t,t_0) \hat{U}_2 (t,t_0) + 1 \times \frac{\hat{H}_2}{i \hbar} \hat{U}_1(t-t_0) \hat{U}_2 (t-t_0) \right]  \ket{\psi_1 \psi_2}_{t_0}
\end{equation*}

\begin{equation*}
    = (\hat{U}_1 \times 1 + 1 \times \hat{U}_2) (\hat{U}_1 (t,t_0) \hat{U}_0 (t,t_0)) \ket{\psi_1 \psi_2}_{t_0}
\end{equation*}

\begin{equation*}
    = \hat{H} \hat{U}(t,t_0) \ket{\psi_1 \psi_2}_{t_0}
\end{equation*}

por lo tanto 

\begin{equation*}
    i \hbar \frac{\partial}{\partial t} \hat{U} (t,t_0) = \hat{H} \hat{U} (t,t_0)
\end{equation*}

ahora

\begin{equation*}
    \hat{U}^{\dagger} (t,t_0) = \hat{U}_{1}^{\dagger} (t,t_0) \hat{U}_{2}^{\dagger} (t,t_0)
\end{equation*}

entonces

\begin{equation*}
    \hat{U}^{\dagger}(t,t_0) \hat{U}(t,t_0) = \hat{U}_{2}^{\dagger} (t,t_0) \cancel{\hat{U}_{1}^{\dagger} (t,t_0) \hat{U}_{1}(t,t_0)} \hat{U}_{2} (t,t_0) = \hat{U}_{2}^{\dagger} (t,t_0) \hat{U}_{2} (t,t_0) = 1
\end{equation*}

\begin{equation*}
    \therefore \hat{U}^{\dagger} = \hat{U}^{-1}
\end{equation*}

y ya por último

\begin{equation*}
    \hat{U}(t_0,t_0) = \hat{U}_{1} (t_0,t_0) \times \hat{U}_{2} (t_0,t_0) = 1 \times 1 = 1
\end{equation*}

\begin{equation*}
    \hat{U}(t,t') \hat{U}(t',t'') = \hat{U}_1 (t,t') \times \hat{U}_{2}(t,t') \hat{U}_{1}(t',t'') \times \hat{U}_{2}(t',t'')
\end{equation*}

\begin{equation*}
    \hat{U}_{1}(t,t') \hat{U}_{1}(t',t'') \times \hat{U}_{2}(t,t') \hat{U}_{2}(t',t'') = \hat{U}_{1}(t,t'') \times \hat{U}_2 (t,t'')= \hat{U}(t,t'')
\end{equation*}



%%%2%%%



\item Una partícula de masa $m$, sujeta a un potencial de oscilador armónico unidimensional en presencia de un campo eléctrico externo constante se encuentra descrita por el hamiltoniano

\begin{equation*}
    \hat{H} = \frac{\hat{p}^2}{2m} + \frac{1}{2} m \omega ^2 \hat{x}^2 - e E\hat{x}
\end{equation*}

con la frecuencia natural $\omega$, la carga $e$ y la magnitud del campo eléctrico $E$. Encuentre las ecuaciones de movimiento para los operadores $\hat{p}_{H}(t)$ y $\hat{x}_{H}(t)$. Resuelva estas ecuaciones en términos de condiciones iniciales generales $\hat{p}_{H} (0)$ y $\hat{x}_{H}(0)$.

\textbf{Sol:} 

 sabemos que y por hipótesis

\begin{equation*}
    i \hbar \frac{d}{dt} \hat{x}_{H} = [\hat{x}_{H}, \frac{\hat{p}_{H}^{2}}{2m}] = i \hbar \frac{\hat{p}_{H}}{m}
\end{equation*}

\begin{equation*}
    i \hbar \frac{d}{dt} \hat{p}_{H} =  [\hat{p}_{H},\frac{1}{2}m \omega^2 \hat{x}_{H}^{2} - q \epsilon \hat{x}_{H}] = - i \hbar m \omega ^2 \hat{x}_{H}+ i \hbar q \epsilon
\end{equation*}

entonces

\begin{equation*}
    \frac{d^2}{dt^2} \hat{p}_{H} = - m \omega^2 \frac{d}{dt} \hat{x}_{H} = - \omega^2 \hat{p}_{H}
\end{equation*}

\begin{equation*}
    \hat{p}_{H} (t) = \hat{A} \cos{\omega t} + \hat{B} \sin{\omega t} \hspace{1cm} \text{con } \frac{\partial \hat{A}}{\partial t} = \frac{\partial \hat{B}}{\partial t} = 0 
    \end{equation*}
    
    y así
    
    \begin{equation*}
        \hat{x}_{H}= - \frac{1}{m \omega^2} \frac{d}{dt} \hat{p}_{H} + \frac{q \epsilon}{m \omega^2} = -\frac{\omega}{m \omega^2}(- \hat{A} \sin{\omega t} + \hat{B} \cos{\omega t}) + \frac{q \epsilon}{m \omega^2}
    \end{equation*}
    
    si $\hat{p}_{H} (0) = \hat{A}$ y $\hat{x}_{H}(0) = - \frac{\hat{B}}{m \omega} + \frac{q \epsilon}{m \omega^2}$
    
    entonces
    
    \begin{equation*}
        \hat{p}_{H} (t) = \hat{p}_{H}(0) \cos{\omega t} + (\frac{q \epsilon}{\omega}- m \omega \hat{x}_{H}(0)) \sin{\omega t}
    \end{equation*}
    
    y 
    
    \begin{equation*}
        \hat{x}_{H}(t) = \frac{1}{m \omega} (\hat{p}_{H} (0) \sin{\omega t} - (\frac{q\epsilon}{\omega}-m \omega \hat{x}_{H}(0)) \sin{\omega t}) + \frac{q \epsilon}{m \omega^2}
    \end{equation*}



%%%3%%%



\item Muestre que si $\hat{A}$ y $\hat{B}$ son constantes de movimiento entonces

\begin{enumerate}
    \item Si el espectro del hamiltoniano es no degenerado, entonces $[\hat{A}, \hat{B}]= 0$.
    
    \textbf{Sol:}
    
    Por hipótesis $[\hat{A}, \hat{H}] = [\hat{B},\hat{H}] = 0$ y
    
    \begin{equation*}
        \hat{H} \ket{E_n} = E_n \ket{E_n}
    \end{equation*}
    
    así que aplicando $\hat{A}$
    
    \begin{equation*}
        \hat{A} \hat{H} \ket{E_n} = E_n \hat{A} \ket{E_n}
    \end{equation*}
    
    \begin{equation*}
        \hat{H}(\hat{A}  \ket{E_n}) = E_n (\hat{A} \ket{E_n})
    \end{equation*}
    
    entonces como $\hat{A} \ket{E_n}$ es estado propio de $\hat{H}$ con el valor propio $E_n$, no le queda más que ser proporcional al estado $\ket{E_n}$, o bien $\hat{A} \ket{E_n} = a_n \ket{E_n}$
    
    Por la misma razón $\hat{B}\ket{E_n} = b_n \ket{E_n}$
    
    entonces
    
    \begin{equation*}
        [\hat{A}, \hat{B}] \ket{\Psi} = \sum_{n} C_{n} [\hat{A}, \hat{B}] \ket{E_n} = \sum_{n} (\hat{A}\hat{B} - \hat{B}\hat{A}) \ket{E_n}
    \end{equation*}
    
    \begin{equation*}
        = \sum_{n} C_n (a_n b_n - b_n a_n) \ket{E_n} = 0
    \end{equation*}
    
    por lo tanto $[\hat{A},\hat{B}] = 0$
    
    
    \item Si $[\hat{A}, \hat{B}] \neq 0$ entonces $i[\hat{A},\hat{B}]$ es una constante de movimiento
    
    \textbf{Sol:}
    
    \begin{equation*}
        [\hat{H}, \hat{C}] = [\hat{H},i \hat{A}\hat{B}] - [\hat{H}, i \hat{B} \hat{A}]
    \end{equation*}
    
    \begin{equation*}
        = i \hat{A} [\hat{H},\hat{B}] + i [\hat{H},\hat{A}] \hat{B} - i \hat{B} [\hat{H}, \hat{A}] - i [\hat{H}, \hat{B}] \hat{A} = 0
    \end{equation*}
    
    
    \item Si $[\hat{A},\hat{B}] \neq 0$ entonces el hamiltoniano debe tener valores propios degenerados.
    
    
    \textbf{Sol:}
    
    Como $\hat{A}^{\dagger} = \hat{A}$ y $\hat{B}^{\dagger} = \hat{B}$ entonces $\hat{C}^{\dagger} = \hat{C}$, y así
    
    \begin{equation*}
        \hat{C} \hat{H} \ket{E_n}  = E_n (\hat{C}\ket{E_n}) = \hat{H}(\hat{C}\ket{E_n})
    \end{equation*}
    
    y si el espectro es no degenerado
    
    \begin{equation*}
        \hat{C} \ket{E_n} = C_n \ket{E_n} = i(\hat{A}\hat{B} - \hat{B}\hat{A}) \ket{E_n}= i(a_nb_n - b_n a_n) \ket{E_n} = 0
    \end{equation*}
    
    lo que es una contradicción, entonces
    $\hat{C} \ket{E_n} = \sum_{n} d_n \ket{E_n} \neq C_n \ket{E_n}$, y como nos dice que existe un estado $\sum_{n} d_n \ket{E_n}$ distinto de $\ket{E_n}$ con el mismo valor propio $E_n$ el espectro es degenerado.
    
    
\end{enumerate}



%%%4%%%



\item Considere un hamiltoniano de la forma $\hat{H} = \hbar \omega (\hat{a}^{\dagger} \hat{a} + \frac{1}{2})$, con los estados $\ket{n}$ y energías propias $E_n = \hbar \omega (n + \frac{1}{2})$ con $n = 0,1,...$ y los operadores

\begin{equation*}
    \hat{a} = \sqrt{\frac{m\omega}{2 \hbar}} \left(\hat{x}+ i \frac{\hat{p}}{m \omega}\right)
\end{equation*}

\begin{equation*}
    \hat{a}^{\dagger} = \sqrt{\frac{m \omega}{2 \hbar}} \left(\hat{x} - i \frac{\hat{p}}{m \omega}\right)
\end{equation*}

cumplen con las reglas de conmutación

\begin{equation*}
    [\hat{a}, \hat{a}^{\dagger}] = 1
\end{equation*}

\begin{enumerate}
    \item Escriba y resuelva la ecuación de Heisenberg para los operadores $\hat{a}$ y $\hat{a}^{\dagger}$ con las condiciones iniciales generales $\hat{a}_{H}(0)$, $\hat{a}_{H}^{\dagger}(0)$.
    
    \textbf{Sol:}
    
    Noten $[\hat{a}, \hat{a}^{\dagger}] = 1$ entonces $[\hat{a}_{H},\hat{a}_{H}^{\dagger}] = 1$ y así
    
    \begin{equation*}
        i \hbar \frac{d}{dt} \hat{a}_{H} = [\hat{a}_{H}, \hat{H}_{H}] = [\hat{a}_{H}, \hbar \omega \hat{a}_{H}^{\dagger}\hat{a}_{H}]  = \hbar \omega \{\hat{a}_{H}^{\dagger}[\hat{a}_{H},\hat{a}_{H}]+ [\hat{a}_{H}, \hat{a}_{H}^{\dagger}]\}
    \end{equation*}
    
    \begin{equation*}
        = \hbar \omega \hat{a}_{H}
    \end{equation*}
    
    entonces
    
    \begin{equation*}
        \hat{a}_{H}(t) = \hat{a}_{H}(0) e^{-i \omega t} = a_0 e^{-i \omega t}
    \end{equation*}
    
    ahora
    
    \begin{equation*}
        i \hbar \frac{d}{dt} \hat{a}_{H}^{\dagger}= [\hat{a}_{H}^{\dagger}, \hat{H}_{H}] = [\hat{a}_{H}^{\dagger}, \hbar \omega \hat{a}_{H}^{\dagger} \hat{a}_{H}]= \hbar \omega \{\hat{a}_{H}^{\dagger}[\hat{a}_{H}^{\dagger}, \hat{a}_{H}]+ [\hat{a}_{H}^{\dagger}, \hat{a}_{H}^{\dagger}] \hat{a}_{H} \} = -\hbar \omega \hat{a}_{H}
    \end{equation*}
    
    entonces
    
    \begin{equation*}
        \hat{a}_{H}^{\dagger} (t) = \hat{a}_{0}^{\dagger} e^{i \omega t}
    \end{equation*}
    
    \item Calcule $\braket{\psi_{H}|\hat{x}_{H}(t)}{\psi_{H}}$ si $\ket{\psi_{H}} = \ket{1}$
    
    \textbf{Sol:}
    
    \item CAlcule $\braket{\psi_{H}| \hat{x}_{H}(t)}{\psi_{H}}$ si $\ket{\psi_{H}}= e^{-\frac{i l \hat{p}}{\hbar}}\ket{0}$, con $l$ una constante real.¿Qué significado tiene el operador $e^{-\frac{i l \hat{p}}{\hbar}}$? Tal vez la siguiente identidad sea útil
    
    \begin{equation*}
        e^{\frac{i l \hat{p}}{\hbar}} \hat{a} e^{- \frac{i l \hat{p}}{\hbar}} = \hat{a} + l \sqrt{\frac{m \omega}{2 \hbar}}
    \end{equation*}
    
    \textbf{Sol:}
    
\end{enumerate}







\end{enumerate}

\end{document}
